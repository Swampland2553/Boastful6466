\documentclass[12pt,a4paper]{article}

% Packages for math, lists, geometry, and title formatting
\usepackage{amsmath}
\usepackage{amssymb}
\usepackage{enumitem} % For customized lists
\usepackage{geometry}
\geometry{a4paper, margin=2.5cm, top=2cm, bottom=2cm} % Adjust margins as needed

\usepackage{titlesec} % To customize section titles
\titleformat{\section}{\normalfont\Large\bfseries\filcenter}{\thesection}{1em}{}
\titleformat{\subsection}{\normalfont\large\bfseries}{\thesubsection}{1em}{}
\titleformat{\subsubsection}{\normalfont\normalsize\bfseries}{\thesubsubsection}{1em}{}

% Persian language support with xepersian and Amiri font
% As requested, load xepersian after other packages.
% Note: xepersian itself loads fontspec.
\usepackage{xepersian}
\setmainfont{Amiri}
\setdigitfont{Amiri} % Optional: use Amiri for digits as well
\setsansfont{Amiri}  % Optional: use Amiri for sans-serif text
\settextfont{Amiri}  % Set the font for Persian text

% Custom command for study units for better structure (optional)
\newcommand{\studyunit}[1]{\par\medskip\noindent\textbf{#1}\par\nopagebreak}
\newcommand{\topics}{\par\medskip\noindent\textbf{مباحث:}\begin{itemize}[nosep,after=\vspace{-0.5\baselineskip}]}
\newcommand{\activities}{\par\medskip\noindent\textbf{فعالیت:}\begin{itemize}[nosep,after=\vspace{-0.5\baselineskip}]}
% Removed custom end commands - using standard \end{itemize} instead
\newcommand{\breaktime}[1]{\par\smallskip\centerline{\textit{#1}}\smallskip}

\begin{document}

\begin{center}
\Large\textbf{برنامه مطالعاتی دقیق برای کسب نمره کامل ریاضی ۱ (جمعه ۲۶ اردیبهشت تا یکشنبه ۲۸ اردیبهشت)}
\end{center}
\hrulefill
\vspace{1em}

\section*{روز ۱: جمعه ۲۶ اردیبهشت}

\subsection*{فرجه صبح (۲ واحد مطالعاتی):}
    \studyunit{واحد ۱ (۹۰ دقیقه): فصل ۵ - تابع (بخش اول)}
        \topics
            \item مفهوم تابع، بازنمایی‌های مختلف تابع (زوج مرتب، نمودار، ضابطه)، تشخیص تابع (آزمون خط عمودی)، دامنه و برد توابع از روی ضابطه و نمودار (با تأکید بر توابع گویا، رادیکالی و چندجمله‌ای).
        \end{itemize}
        \activities
            \item مرور دقیق درسنامه، حل مثال‌های کلیدی کتاب و "کار در کلاس"ها، حل تمرینات منتخب مربوط به این بخش‌ها از کتاب درسی و یک منبع کمک‌آموزشی قوی.
        \end{itemize}

    \breaktime{استراحت (۱۵ دقیقه)}

    \studyunit{واحد ۲ (۹۰ دقیقه): فصل ۵ - تابع (بخش دوم)}
        \topics
            \item انواع تابع (خطی، ثابت، همانی، قدرمطلق، چندضابطه‌ای، پله‌ای)، رسم نمودار این توابع، اعمال جبری روی توابع (جمع، تفریق، ضرب، تقسیم)، ترکیب توابع.
        \end{itemize}
        \activities
            \item تسلط بر رسم نمودار توابع پایه و انتقال آن‌ها، حل مسائل مربوط به ترکیب توابع و دامنه آن‌ها، حل تمرینات چالشی.
        \end{itemize}

\subsection*{فرجه عصر (۲ واحد مطالعاتی):}
    \studyunit{واحد ۱ (۹۰ دقیقه): فصل ۲ - مثلثات (بخش اول)}
        \topics
            \item نسبت‌های مثلثاتی در مثلث قائم‌الزاویه، زوایای معروف (۳۰، ۴۵، ۶۰ درجه)، دایره مثلثاتی، تعیین علامت نسبت‌ها در نواحی مختلف، محاسبه مقادیر مثلثاتی زوایای خاص روی دایره.
        \end{itemize}
        \activities
            \item مرور دقیق تعاریف و دایره مثلثاتی، حل مسائل محاسباتی و کاربردی (شیب، ارتفاع)، تمرین بر روی پیدا کردن مقادیر مثلثاتی زوایای قرینه و مکمل.
        \end{itemize}

    \breaktime{استراحت (۱۵ دقیقه)}

    \studyunit{واحد ۲ (۹۰ دقیقه): فصل ۲ - مثلثات (بخش دوم)}
        \topics
            \item روابط بین نسبت‌های مثلثاتی (اتحادهای اصلی: $\sin^2\alpha + \cos^2\alpha = 1$, $\tan\alpha = \frac{\sin\alpha}{\cos\alpha}$ و ...)، اثبات اتحادهای مثلثاتی ساده، حل معادلات مثلثاتی مقدماتی (در حد کتاب).
        \end{itemize}
        \activities
            \item تمرین زیاد بر روی اثبات اتحادها، حل مسائل مربوط به پیدا کردن سایر نسبت‌ها با داشتن یکی از آن‌ها و ناحیه زاویه.
        \end{itemize}

\rule{\linewidth}{0.4pt}\vspace{1em} % Horizontal line between days

\section*{روز ۲: شنبه ۲۷ اردیبهشت}

\subsection*{فرجه صبح (۲ واحد مطالعاتی):}
    \studyunit{واحد ۱ (۹۰ دقیقه): فصل ۴ - معادله‌ها و نامعادله‌ها (بخش اول)}
        \topics
            \item حل معادله درجه دوم به روش‌های مختلف (تجزیه، ریشه‌گیری، مربع کامل، فرمول کلی دلتا)، تشخیص تعداد ریشه‌ها با استفاده از دلتا، مجموع و حاصلضرب ریشه‌ها.
        \end{itemize}
        \activities
            \item حل تعداد زیادی معادله درجه دوم با روش‌های مختلف، تمرکز بر مسائلی که نیاز به تشکیل معادله دارند.
        \end{itemize}

    \breaktime{استراحت (۱۵ دقیقه)}

    \studyunit{واحد ۲ (۹۰ دقیقه): فصل ۴ - معادله‌ها و نامعادله‌ها (بخش دوم)}
        \topics
            \item سهمی (رسم، رأس، خط تقارن، جهت تقعر)، تعیین علامت چندجمله‌ای درجه اول و دوم، حل نامعادلات یک متغیره درجه اول و دوم و نامعادلات گویا.
        \end{itemize}
        \activities
            \item تمرین بر روی رسم دقیق سهمی و تحلیل آن، حل انواع نامعادلات با استفاده از جدول تعیین علامت.
        \end{itemize}

\subsection*{فرجه عصر (۲ واحد مطالعاتی):}
    \studyunit{واحد ۱ (۹۰ دقیقه): فصل ۳ - توان‌های گویا و عبارت‌های جبری}
        \topics
            \item ریشه nام، توان‌های کسری و قوانین آن‌ها، ساده‌سازی عبارت‌های رادیکالی، اتحادهای جبری (مربع، مکعب، مزدوج، جمله مشترک)، تجزیه عبارت‌ها، گویا کردن مخرج کسرها.
        \end{itemize}
        \activities
            \item تمرین بر روی ساده‌سازی عبارت‌های پیچیده توانی و رادیکالی، تسلط بر تمام اتحادها و کاربرد آن‌ها در تجزیه، حل مسائل متنوع گویا کردن.
        \end{itemize}

    \breaktime{استراحت (۱۵ دقیقه)}

    \studyunit{واحد ۲ (۹۰ دقیقه): فصل ۶ - شمارش، بدون شمردن و فصل ۷ - آمار و احتمال (بخش اول)}
        \textbf{مباحث فصل ۶:}
        \begin{itemize}[nosep,after=\vspace{-0.5\baselineskip}]
            \item اصول شمارش (اصل جمع و اصل ضرب)، جایگشت (با تکرار و بدون تکرار)، ترکیب.
        \end{itemize}
        \textbf{مباحث فصل ۷ (بخش اول):}
        \begin{itemize}[nosep,after=\vspace{-0.5\baselineskip}]
            \item مفاهیم مقدماتی احتمال، فضای نمونه‌ای، پیشامد، انواع پیشامد (ساده، مرکب، حتمی، محال).
        \end{itemize}
        \activities
            \item حل مسائل متنوع شمارش با تشخیص دقیق استفاده از اصل ضرب، جایگشت یا ترکیب. تمرین بر روی نوشتن فضای نمونه‌ای و شناسایی پیشامدها در آزمایش‌های ساده.
        \end{itemize}

\rule{\linewidth}{0.4pt}\vspace{1em} % Horizontal line between days

\section*{روز ۳: یکشنبه ۲۸ اردیبهشت (روز مرور، جمع‌بندی، حل نمونه سوال و رفع اشکال)}

\subsection*{فرجه صبح (۲ واحد مطالعاتی):}
    \studyunit{واحد ۱ (۹۰ دقیقه): حل نمونه سوال امتحان نهایی (بخش اول)}
        \activities
            \item انتخاب یک یا دو نمونه سوال کامل امتحان نهایی سال‌های گذشته (یا آزمون شبیه‌ساز معتبر). شروع به حل سوالات به ترتیب، با رعایت زمان‌بندی. تمرکز بر سوالات فصولی که در روزهای قبل مطالعه شده‌اند (تابع، مثلثات، معادله‌ها).
        \end{itemize}

    \breaktime{استراحت (۱۵ دقیقه)}

    \studyunit{واحد ۲ (۹۰ دقیقه): حل نمونه سوال امتحان نهایی (بخش دوم) و تحلیل اولیه}
        \activities
            \item ادامه حل نمونه سوال. پس از اتمام، بررسی اولیه پاسخ‌ها و مشخص کردن سوالاتی که در آن‌ها مشکل داشته‌اید یا اشتباه کرده‌اید.
        \end{itemize}

\subsection*{فرجه عصر (۲ واحد مطالعاتی):}
    \studyunit{واحد ۱ (۹۰ دقیقه): مرور، جمع‌بندی و رفع اشکال نقاط ضعف شناسایی شده}
        \activities
            \item بر اساس تحلیل نمونه سوال حل شده و پیش‌آزمونی که قبلاً انجام داده‌اید:
            \begin{itemize}[nosep, labelindent=1em]
                \item \textbf{مرور سریع فرمول‌ها، تعاریف و نکات کلیدی مباحثی که در آن‌ها ضعف دارید.} (مثلاً اگر در اتحادهای مثلثاتی مشکل داشتید، آن‌ها را مرور کنید).
                \item \textbf{حل مجدد سوالاتی که اشتباه کرده‌اید} و سعی در فهمیدن دلیل اشتباه.
                \item \textbf{حل چند تمرین مشابه از مباحثی که در آن‌ها احساس ضعف می‌کنید} از کتاب درسی یا کمک‌آموزشی.
                \item \textbf{تمرکز بر قسمت‌های مهم و پرسوال:} (مثلاً دامنه و برد توابع، حل معادله درجه دوم، کاربرد نسبت‌های مثلثاتی).
            \end{itemize}
        \end{itemize}

    \breaktime{استراحت (۱۵ دقیقه)}

    \studyunit{واحد ۲ (۹۰ دقیقه): جمع‌بندی نهایی و نکات تکمیلی}
        \activities
            \item % Add an item for the outer list
            \begin{itemize}[nosep, labelindent=1em]
                \item \textbf{مرور تیپ سوالات مختلف} از هر فصل.
                \item \textbf{تورق سریع کل کتاب} و یادآوری نکات مهم هر درس.
                \item \textbf{مطالعه "اشتباهات مهلک" و "نکات طلایی"} (که در ادامه ذکر می‌شود).
                \item \textbf{آماده‌سازی ذهنی برای امتحان،} مدیریت استرس و ایجاد اعتماد به نفس.
            \end{itemize}
        \end{itemize}

\rule{\linewidth}{0.4pt}\vspace{1em}

\section*{اشتباهات مهلک در این روزها (تأکید مضاعف):}
\begin{enumerate}[label=\arabic*., itemsep=0.2em, topsep=0.3em]
    \item \textbf{یادگیری مطالب جدید:} این سه روز زمان یادگیری مطالب جدید نیست. اگر مبحثی را اصلاً نخوانده‌اید، سعی نکنید در این زمان کوتاه آن را از صفر یاد بگیرید. تمرکز بر تثبیت آموخته‌ها و رفع اشکالات باشد.
    \item \textbf{حل کردن صرف و بدون تحلیل:} فقط حل کردن نمونه سوال کافی نیست. باید سوالات را تحلیل کنید، اشتباهات خود را پیدا کرده و دلیل آن‌ها را بفهمید.
    \item \textbf{نادیده گرفتن کتاب درسی:} منبع اصلی امتحان نهایی، کتاب درسی است. تمام مثال‌ها، فعالیت‌ها و کار در کلاس‌های آن مهم هستند.
    \item \textbf{استرس و کم‌خوابی:} اضطراب و خستگی زیاد، بازدهی شما را به شدت کاهش می‌دهد. سعی کنید خواب کافی داشته باشید و با آرامش مطالعه کنید.
    \item \textbf{حفظ کردن طوطی‌وار فرمول‌ها بدون درک مفهوم:} باید بدانید هر فرمول از کجا آمده و در چه شرایطی کاربرد دارد.
    \item \textbf{توجه نکردن به قیود سوال:} کلماتی مانند "بدون تکرار ارقام"، "حداقل"، "دقیقاً" و ... در صورت سوال بسیار مهم هستند و بی‌توجهی به آن‌ها منجر به پاسخ اشتباه می‌شود.
    \item \textbf{عجله در محاسبات و پاسخگویی:} با دقت محاسبات را انجام دهید و پاسخ نهایی را کنترل کنید.
\end{enumerate}

\vspace{1em}
\rule{\linewidth}{0.4pt}\vspace{1em}

\section*{نکات طلایی برای این درس و این روزها (تأکید مضاعف):}
\begin{enumerate}[label=\arabic*., itemsep=0.2em, topsep=0.3em]
    \item \textbf{تسلط بر مفاهیم پایه:} مطمئن شوید که مفاهیم اولیه هر فصل (مانند تعریف تابع، دایره مثلثاتی، مفهوم دلتا) را به خوبی درک کرده‌اید.
    \item \textbf{تمرین، تمرین، تمرین:} ریاضی با تمرین زیاد ملکه ذهن می‌شود. تا می‌توانید مسائل متنوع حل کنید.
    \item \textbf{اهمیت فوق‌العاده کتاب درسی:} دوباره تأکید می‌کنم، متن کتاب، مثال‌ها، فعالیت‌ها و کار در کلاس‌ها را به دقت مطالعه و حل کنید.
    \item \textbf{تحلیل نمونه سوالات نهایی سال‌های قبل:} بهترین منبع برای آشنایی با سبک و سطح سوالات، نمونه‌های واقعی امتحانات گذشته است.
    \item \textbf{خلاصه‌نویسی فرمول‌ها و نکات مهم:} یک دفترچه کوچک برای یادداشت فرمول‌ها، تعاریف کلیدی و نکاتی که فراموش می‌کنید تهیه کنید و در این روزها مرور نمایید.
    \item \textbf{دسته‌بندی سوالات:} سعی کنید تیپ‌های مختلف سوالات هر فصل را شناسایی کنید و برای هر کدام روش حل مناسب را بلد باشید.
    \item \textbf{مدیریت زمان در جلسه امتحان:} از قبل برای خودتان مشخص کنید که برای هر سوال یا هر بخش چقدر زمان می‌خواهید اختصاص دهید.
    \item \textbf{شروع با سوالات آسان‌تر:} در جلسه امتحان، ابتدا به سوالاتی که مطمئن هستید پاسخ دهید تا اعتماد به نفستان بیشتر شود.
    \item \textbf{بازخوانی سوال و پاسخ:} پس از حل هر سوال، یک بار دیگر صورت سوال و پاسخ خود را مرور کنید تا از اشتباهات احتمالی جلوگیری شود.
    \item \textbf{مثبت‌اندیشی و باور به توانایی‌ها:} شما برای این امتحان تلاش کرده‌اید، پس به خودتان اعتماد داشته باشید و با دید مثبت در جلسه حاضر شوید.
\end{enumerate}

\end{document}