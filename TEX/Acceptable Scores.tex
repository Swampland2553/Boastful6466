\documentclass[12pt,a4paper]{article}

% Packages for general use
\usepackage{amsmath}
\usepackage{amssymb}
\usepackage{enumitem} % For customizing lists
\usepackage[a4paper, margin=2.5cm]{geometry} % Set margins

% Font and Persian language support
\usepackage{fontspec}
% xepersian must be loaded after fontspec and other packages that might conflict
\usepackage{xepersian}

% Set the main font to Amiri
\settextfont{Amiri}
% Optional: Set a specific font for Persian digits if Amiri doesn't render them as desired
% \setdigitfont{Amiri} % Usually Amiri handles Persian digits well by default

% Document Information
\title{برنامه مطالعاتی دین و زندگی ۳ (ویژه کسب نمره قابل قبول)}
\author{} % You can add an author if you like
\date{}    % To remove the date

% Customizations for lists if needed (Persian style)
% \setlist[itemize,1]{label=\textbullet} % Default
% \setlist[itemize,2]{label=--}

\begin{document}
\maketitle
\begin{center}
\rule{0.8\textwidth}{0.4pt}
\end{center}
\vspace{0.5cm}

\section*{اصول کلی برنامه}
\begin{itemize}
    \item پوشش اکثر مفاهیم کتاب: مطالعه دقیق‌تر و جامع‌تر نسبت به برنامه قبولی.
    \item تمرکز بر درک ارتباط مفاهیم: توانایی ربط دادن مباحث مختلف درون یک درس و بین دروس.
    \item آشنایی با انواع سؤالات: آمادگی برای پاسخ به سؤالات ترجمه، پیام، مفهومی، و تحلیلی ساده.
    \item حل نمونه سؤالات منتخب: تمرین و ارزیابی آموخته‌ها.
\end{itemize}

\hrulefill
\section*{پنجشنبه، ۲۵ اردیبهشت}

\subsection*{فرجه صبح (حدود ۳ ساعت): بخش اول: تفکر و اندیشه - توحید، اختیار و قضا و قدر}

\subsubsection*{۱. واحد اول (۱ ساعت و ۳۰ دقیقه): درس ۱: هستی‌بخش و درس ۲: یگانه بی‌همتا}
\textbf{تمرکز ویژه:}
\begin{itemize}
    \item نیازمندی جهان به خدا در پیدایش و بقا (با استدلال کتاب) (درس ۱).
    \item ناتوانی در شناخت ذات خداوند و دلایل آن (درس ۱).
    \item مراتب توحید و شرک (خالقیت، مالکیت، ولایت، ربوبیت) با تعاریف دقیق و تمایزات (درس ۲).
    \item مرز توحید و شرک در استفاده از اسباب (\textbf{بسیار مهم}، با درک عمیق) (درس ۲).
\end{itemize}
\textbf{روش مطالعه:}
\begin{itemize}
    \item مطالعه دقیق متن، تحلیل مثال‌ها.
    \item بررسی عمیق آیات کلیدی (فاطر/۱۵، الرحمن/۲۹، اخلاص، رعد/۱۶، آل عمران/۲۶، انعام/۱۶۴) و توانایی استخراج پیام و ارتباط با مفاهیم.
    \item حل کامل سؤالات «تدبر در قرآن» و «اندیشه و تحقیق» این دو درس.
\end{itemize}

\subsubsection*{۲. استراحت (۱۵ دقیقه)}

\subsubsection*{۳. واحد دوم (۱ ساعت و ۳۰ دقیقه): درس ۵: قدرت پرواز}
\textbf{تمرکز ویژه:}
\begin{itemize}
    \item نشانه‌های وجدانی اختیار (تفکر و تصمیم، رضایت و پشیمانی، مسئولیت‌پذیری).
    \item تعریف دقیق قضا و قدر الهی و تفاوت آن‌ها.
    \item تحلیل دقیق رابطه اختیار انسان با قضا و قدر الهی (جلوگیری از برداشت جبری یا تفویضی).
    \item تفاوت علم پیشین خداوند با جبر.
    \item داستان امام علی (ع) و دیوار کج و تحلیل عمیق آن.
\end{itemize}
\textbf{روش مطالعه:}
\begin{itemize}
    \item مطالعه تحلیلی و مفهومی بخش‌های مربوط به رابطه اختیار و قضا و قدر.
    \item بررسی آیات کلیدی (انسان/۳، انعام/۱۰۴، آل عمران/۱۸۲) و استخراج نکات مربوط به اختیار و مسئولیت.
    \item پاسخ تشریحی و مستدل به سؤالات «تدبر در قرآن» و «اندیشه و تحقیق».
\end{itemize}

\hrulefill
\subsection*{فرجه عصر (حدود ۳ ساعت): بخش اول: تفکر و اندیشه - سنت‌های الهی و توحید عملی و توبه}

\subsubsection*{۱. واحد اول (۱ ساعت و ۳۰ دقیقه): درس ۶: سنت‌های خداوند در زندگی}
\textbf{تمرکز ویژه:}
\begin{itemize}
    \item ویژگی‌های سنت‌های الهی.
    \item سنت ابتلاء: هدف، گستردگی، و تفاوت با امتحان بشری.
    \item امداد عام و توفیق الهی: تفاوت و شرایط برخورداری از توفیق.
    \item سنت املاء و استدراج: تعریف دقیق، تفاوت، نشانه‌ها و دلیل هر یک (\textbf{بسیار مهم}).
    \item سنت تأثیر اعمال (با مثال‌های متنوع از کتاب).
\end{itemize}
\textbf{روش مطالعه:}
\begin{itemize}
    \item مطالعه دقیق و مفهومی هر سنت، با تمرکز بر مثال‌ها و آیات مرتبط.
    \item بررسی عمیق آیات کلیدی (عنکبوت/۲، اسراء/۲۰، عنکبوت/۶۹، آل عمران/۱۷۸، اعراف/۱۸۲-۱۸۳، اعراف/۹۶) و تحلیل پیام آن‌ها.
    \item حل کامل سؤالات «تدبر در قرآن» و «اندیشه و تحقیق».
\end{itemize}

\subsubsection*{۲. استراحت (۱۵ دقیقه)}

\subsubsection*{۳. واحد دوم (۱ ساعت و ۳۰ دقیقه): درس ۳: توحید و سبک زندگی و درس ۴: فقط برای او}
\textbf{تمرکز ویژه:}
\begin{itemize}
    \item آثار فردی و اجتماعی توحید عملی (درس ۳).
    \item رابطه دوسویه توحید فردی و اجتماعی (درس ۳).
    \item اخلاص و اهمیت آن در پذیرش اعمال (درس ۳).
    \item ارکان توبه (پشیمانی، تصمیم بر ترک، جبران گذشته – با جزئیات حقوق الهی و مردم) (درس ۴).
    \item آثار توبه (آمرزش، تبدیل سیئات به حسنات) (درس ۴).
    \item ریا: تعریف، انواع، آثار و راه‌های مقابله (درس ۴).
\end{itemize}
\textbf{روش مطالعه:}
\begin{itemize}
    \item مطالعه دقیق متن با تمرکز بر مصادیق و تحلیل‌ها.
    \item بررسی آیات و روایات کلیدی و استخراج نکات مهم.
    \item حل کامل سؤالات «تدبر در قرآن» و «اندیشه و تحقیق» این دو درس، به‌ویژه سؤالات کاربردی.
\end{itemize}

\hrulefill
\section*{جمعه، ۲۶ اردیبهشت}

\subsection*{فرجه صبح (حدود ۳ ساعت): بخش دوم: در مسیر - احکام و پایه‌های تمدن}

\subsubsection*{۱. واحد اول (۱ ساعت و ۳۰ دقیقه): درس ۸: احکام الهی در زندگی امروز}
\textbf{تمرکز ویژه:}
\begin{itemize}
    \item حکمت و ضرورت احکام الهی.
    \item احکام اجتماعی در عرصه‌های فرهنگ و ارتباطات، ورزش و بازی، و اقتصاد (با ذکر دلایل و حدود هر حکم).
    \item توجه به استثنائات و شرایط خاص (مثلاً در مورد شرط‌بندی یا موسیقی).
\end{itemize}
\textbf{روش مطالعه:}
\begin{itemize}
    \item مطالعه دقیق متن با تمرکز بر فلسفه و حکمت احکام در کنار خود احکام.
    \item بررسی آیات کلیدی و ارتباط آن‌ها با احکام مطرح شده.
    \item حل کامل سؤالات «تدبر در قرآن» و «اندیشه و تحقیق»، با تمرکز بر کاربرد احکام در موقعیت‌های جدید.
\end{itemize}

\subsubsection*{۲. استراحت (۱۵ دقیقه)}

\subsubsection*{۳. واحد دوم (۱ ساعت و ۳۰ دقیقه): درس ۹: پایه‌های استوار}
\textbf{تمرکز ویژه:}
\begin{itemize}
    \item تعریف دقیق و نقش هر یک از پایه‌های تمدن اسلامی (توحید، نبوت و امامت، عدالت اجتماعی، علم، اخلاق، خانواده، کار و تلاش).
    \item ارتباط این پایه‌ها با یکدیگر و تأثیر متقابل آن‌ها.
\end{itemize}
\textbf{روش مطالعه:}
\begin{itemize}
    \item مطالعه تحلیلی متن با تمرکز بر اهمیت و کارکرد هر پایه.
    \item بررسی آیات کلیدی و استخراج نقش آن‌ها در تبیین این پایه‌ها.
    \item حل کامل سؤالات «تدبر در قرآن» و «اندیشه و تحقیق»، با تمرکز بر چگونگی تقویت این پایه‌ها در جامعه امروز.
\end{itemize}

\hrulefill
\subsection*{فرجه عصر (حدود ۳ ساعت): بخش دوم: در مسیر - تمدن جدید و مرور بخش اول}

\subsubsection*{۱. واحد اول (۱ ساعت و ۳۰ دقیقه): درس ۱۰: مسئولیت ما}
\textbf{تمرکز ویژه:}
\begin{itemize}
    \item شناسایی و تحلیل آثار مثبت و منفی تمدن جدید غرب (با ذکر مثال‌های دقیق از کتاب).
    \item تبیین رویکرد صحیح اسلامی (نقادانه و هوشمندانه) در مواجهه با تمدن جدید.
    \item مسئولیت‌های عملی مسلمانان در این مواجهه.
\end{itemize}
\textbf{روش مطالعه:}
\begin{itemize}
    \item مطالعه دقیق و تحلیلی متن، با تمرکز بر نقد منصفانه و ارائه راهکارهای اسلامی.
    \item بررسی آیات کلیدی و نقش آن‌ها در تبیین مسئولیت اجتماعی.
    \item حل کامل سؤالات «تدبر در قرآن» و «اندیشه و تحقیق»، با تأکید بر ارائه راه‌حل‌های عملی و قابل اجرا.
\end{itemize}

\subsubsection*{۲. استراحت (۱۵ دقیقه)}

\subsubsection*{۳. واحد دوم (۱ ساعت و ۳۰ دقیقه): مرور مفهومی و تست‌زنی از دروس ۱ تا ۶}
\textbf{روش مرور:}
\begin{itemize}
    \item مرور خلاصه‌نویسی‌ها و نکات کلیدی.
    \item حل تعدادی تست مفهومی و ترکیبی از این دروس (از کتاب‌های تست استاندارد یا آزمون‌های آزمایشی معتبر).
    \item تمرکز بر رفع اشکال سؤالاتی که اشتباه پاسخ داده‌اید.
\end{itemize}

\hrulefill
\section*{شنبه، ۲۷ اردیبهشت}

\subsection*{فرجه صبح (حدود ۳ ساعت): مرور بخش دوم و جمع‌بندی آیات}

\subsubsection*{۱. واحد اول (۱ ساعت و ۳۰ دقیقه): مرور مفهومی و تست‌زنی از دروس ۷ تا ۱۰}
\textbf{روش مرور:} مشابه مرور دروس ۱ تا ۶.
\begin{itemize}
    \item تمرکز بر سؤالات کاربردی مربوط به احکام و تحلیل‌های مربوط به تمدن.
\end{itemize}

\subsubsection*{۲. استراحت (۱۵ دقیقه)}

\subsubsection*{۳. واحد دوم (۱ ساعت و ۳۰ دقیقه): جمع‌بندی و دسته‌بندی آیات و روایات مهم کتاب}
\textbf{روش جمع‌بندی:}
\begin{itemize}
    \item مرور تمام آیات و روایاتی که در طول مطالعه مشخص کرده‌اید.
    \item دسته‌بندی موضوعی آیات (مثلاً آیات مربوط به توحید، آیات مربوط به اختیار، آیات مربوط به توبه و...).
    \item تمرین بر یادآوری پیام اصلی و مفهوم کلیدی هر آیه بدون دیدن ترجمه.
\end{itemize}

\hrulefill
\subsection*{فرجه عصر (حدود ۳ ساعت): حل نمونه سؤال جامع، رفع اشکال و آمادگی نهایی}

\subsubsection*{۱. واحد اول و دوم (مجموعاً حدود ۳ ساعت با یک استراحت کوتاه در صورت نیاز):}
\begin{itemize}
    \item حل کامل حداقل ۱ تا ۲ نمونه سؤال امتحان نهایی جامع (با رعایت زمان‌بندی).
    \item بررسی دقیق پاسخنامه و تحلیل اشتباهات.
    \item شناسایی نقاط ضعف نهایی و مراجعه سریع به کتاب برای رفع آن‌ها.
    \item مرور نهایی و بسیار سریع نکات بسیار مهم، آیات کلیدی و تعاریفی که ممکن است فراموش کرده باشید.
    \item ایجاد آمادگی ذهنی و کاهش استرس: به تلاش خود اعتماد کنید و با آرامش وارد جلسه امتحان شوید.
\end{itemize}

\hrulefill
\section*{اشتباهات مهلک در این روزها (برای کسی که نمره قابل قبول می‌خواهد):}
\begin{enumerate}
    \item مطالعه سطحی و روزنامه‌وار: برای نمره قابل قبول نیاز به درک عمیق‌تر مفاهیم است.
    \item نادیده گرفتن اهمیت تحلیل آیات و روایات: بخش قابل توجهی از نمره به این بخش اختصاص دارد.
    \item عدم توجه کافی به سؤالات "تدبر در قرآن" و "اندیشه و تحقیق": این بخش‌ها مهارت تحلیلی شما را می‌سنجند.
    \item عدم حل نمونه سؤال کافی: آشنایی با سبک سؤالات و مدیریت زمان ضروری است.
\end{enumerate}

\section*{نکات طلایی برای کسب نمره قابل قبول:}
\begin{enumerate}
    \item فهم عمیق مفاهیم اصلی و توانایی توضیح آن‌ها به زبان خودتان.
    \item تسلط بر ترجمه روان و پیام اصلی آیات و روایات مهم.
    \item توانایی برقراری ارتباط بین مفاهیم مختلف و تحلیل ساده مسائل.
    \item تمرین پاسخگویی به انواع مختلف سؤالات (کوتاه پاسخ، تشریحی، تطبیقی).
\end{enumerate}

\end{document}