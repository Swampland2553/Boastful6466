\documentclass[aspectratio=169]{beamer}

% ----------------------------------------------------------------------
% THEME & APPEARANCE
% ----------------------------------------------------------------------
\usetheme{Madrid} % More elegant and modern theme
\usecolortheme{dolphin} % Professional color scheme
\useinnertheme{rectangles} % Changed from circles to rectangles for a slightly different look
\useoutertheme{miniframes}

\beamertemplatenavigationsymbolsempty

\definecolor{myblue}{RGB}{0, 75, 150}
\definecolor{mygold}{RGB}{230, 180, 80} % A slightly richer gold

\setbeamercolor{frametitle}{bg=myblue, fg=white}
\setbeamercolor{title}{fg=myblue, bg=} % Ensure title text is blue, no specific background for title itself
\setbeamercolor{subtitle}{fg=myblue!75!black} % Make subtitle a slightly darker shade of blue/black
\setbeamercolor{author}{fg=black}
\setbeamercolor{institute}{fg=black}
\setbeamercolor{date}{fg=black}

\setbeamercolor{structure}{fg=myblue} % For bullets, item numbers etc.

% Make footer elements less prominent if the footer feels too busy
\setbeamercolor{author in foot}{fg=gray}
\setbeamercolor{title in foot}{fg=gray}
\setbeamercolor{date in foot}{fg=gray}
\setbeamercolor{institute in foot}{fg=gray} % Madrid theme might show institute in foot

% ----------------------------------------------------------------------
% PACKAGES
% ----------------------------------------------------------------------
% General LaTeX Packages (load before xepersian if they might interact)
% \usepackage{amsmath} % If you need advanced math
% \usepackage{graphicx} % If you decide to add actual images later
\usepackage{adjustbox} % For auto-scaling content to fit available space
\newsavebox{\mycounselorparttwobox} % For saving complex content before scaling

% XePersian for Persian typesetting
\usepackage{xepersian}
\settextfont{Amiri}

\logo{\textcolor{mygold}{\textsf{\textbf{دبیرستان پیام}}}} % Changed font slightly for logo

% ----------------------------------------------------------------------
% PRESENTATION METADATA
% ----------------------------------------------------------------------
\title{همراهی برای اوج: طرح جامع مشاوره تحصیلی دبیرستان پیام}
\subtitle{سال تحصیلی کنکور - ویژه دانش‌آموزان دوازدهم ریاضی فیزیک}
\author{دبیرستان پیام} % You can change this to the counselor's name or department
\institute{مشاوره تحصیلی دبیرستان پیام}
\date{\today} % You can set a specific date, e.g., "پاییز ۱۴۰۳"

% ----------------------------------------------------------------------
% BEGIN DOCUMENT
% ----------------------------------------------------------------------
\begin{document}

% ----------------------------------------------------------------------
% SLIDE 1: TITLE
% ----------------------------------------------------------------------
\begin{frame}[plain] % Using [plain] to remove header/footer from the title slide for a cleaner look
  \titlepage
  % You can add a small, centered tagline below the titlepage if you wish
  % For example:
  % \vfill % Pushes the tagline to the bottom
  % \begin{center}
  %  \textcolor{mygold}{\large \textit{همراهی برای اوج}}
  % \end{center}
\end{frame}

% ----------------------------------------------------------------------
% SLIDE 2: INTRODUCTION AND COUNSELOR INFO
% ----------------------------------------------------------------------
\begin{frame}
  \frametitle{\textcolor{white}{\textbf{چرا اینجا هستیم؟ هدف مشترک ما}}}

  \begin{block}{اهداف ما}
    \begin{itemize}
      \item موفقیت درخشان در کنکور و ورود به دانشگاه‌های برتر.
      \item رشد همه‌جانبه فردی و کسب مهارت‌های زندگی.
      \item ایجاد یک تجربه مثبت و به‌یادماندنی از سال سرنوشت‌ساز کنکور.
    \end{itemize}
  \end{block}

  \begin{alertblock}{معرفی کوتاه مشاور}
    نام، تخصص، و سابقه (اشاره به ۱۰ سال تجربه).
  \end{alertblock}
\end{frame}

% ----------------------------------------------------------------------
% SLIDE: COUNSELOR INTRODUCTION (NEW SLIDE)
% ----------------------------------------------------------------------
\begin{frame}[fragile]
  \frametitle{همراه متخصص شما در مسیر موفقیت تحصیلی}

  \begin{columns}[T,totalwidth=\textwidth] % Top-aligned columns with explicit total width
    \begin{column}{0.62\textwidth} % Left column for details (slightly wider)
      % Removed block environment to eliminate the gray box
      \small % Slightly smaller text to fit better
      \hfill\textbf{درباره من:}\hfill\mbox{}
      \par
      با بیش از ۷ سال سابقه درخشان در حوزه مشاوره تحصیلی و همکاری با کانون فرهنگی آموزش و مدارس مطرح شیراز، افتخار همراهی با دانش‌آموزان در مسیر موفقیت را داشته‌ام.
      \par
      هدف من، توانمندسازی شما برای دستیابی به بهترین نتایج و رشد همه‌جانبه است.
      \smallskip

      \hfill\textbf{سوابق تحصیلی:}\hfill\mbox{}
      \begin{itemize}\setlength{\itemsep}{0pt}\setlength{\parskip}{0pt} % Tighter spacing
        \item دانشجوی سال آخر دکتری تخصصی مهندسی مکانیک - گرایش تبدیل انرژی
        \item کارشناسی ارشد مهندسی مکانیک - گرایش تبدیل انرژی (دانشگاه بیرجند، ۱۳۹۳-۱۳۹۶)
        \item کارشناسی مهندسی مکانیک - گرایش حرارت و سیالات (دانشگاه شهید باهنر کرمان، ۱۳۸۳-۱۳۸۸)
      \end{itemize}
      \normalsize % Reset to normal size
    \end{column}

    \begin{column}{0.35\textwidth} % Right column for personal info (slightly narrower)
      % Removed block environment to eliminate the gray box
      \centering % Center the content
      \textbf{مسلم ایوبی راد}
      \par\smallskip
      مشاور تحصیلی متخصص و با تجربه
      \par\medskip
      \textbf{اطلاعات تماس:}
      \par\smallskip
      \begin{tabular}{r}
        ایمیل: M.AyubiRad@gmail.com \\
      \end{tabular}
      \par\smallskip
      \footnotesize{(می‌توان به جای اطلاعات تماس شخصی، به راه‌های ارتباطی از طریق مدرسه اشاره کرد)}
    \end{column}
  \end{columns}
\end{frame}

% ----------------------------------------------------------------------
% SLIDE: COUNSELOR INTRODUCTION - PART 2 (NEW SLIDE)
% ----------------------------------------------------------------------
\begin{frame}[fragile]
  \frametitle{همراه متخصص شما (ادامه)}

  \begin{block}{تجربه در عمل و مهارت‌های کاربردی}
    \textbf{سوابق شغلی:}
    \begin{itemize}
      \item مشاور تحصیلی در مدارس و آموزشگاه‌های برتر (طاها، امام رضا، نگرش) (۱۴۰۰-۱۴۰۴)
      \item مدیر گروه کنکور و مدارس در کانون فرهنگی آموزش (۱۳۹۶-۱۳۹۹)
    \end{itemize}
    \textbf{دوره‌ها و گواهینامه‌ها:}
    \begin{itemize}
      \item دارای گواهینامه‌های تخصصی در اصول و فنون مشاوره، برنامه‌ریزی درسی، مهارت‌های مطالعه، کوچینگ تحصیلی و استعدادیابی.
    \end{itemize}
    \textbf{مهارت‌های برجسته:}
    \begin{itemize}
      \item برنامه‌ریزی و مشاوره تخصصی کنکور
      \item مدیریت و راهبری تحصیلی و انگیزشی
      \item فن بیان و سخنوری موثر
      \item افزایش تمرکز و انگیزه در دانش‌آموزان
      \item آموزش روش‌های نوین مطالعه و یادگیری
      \item انتخاب رشته تحصیلی هوشمندانه
    \end{itemize}
  \end{block}

  \begin{block}{زبان انگلیسی}
      \begin{itemize}
          \item خواندن: بسیار خوب
          \item نوشتن: متوسط
          \item صحبت کردن: خوب
          \item گوش دادن: بسیار خوب
      \end{itemize}
  \end{block}
\end{frame}

% ----------------------------------------------------------------------
% SLIDE 3: IMPORTANCE AND PILLARS OF THE PLAN
% ----------------------------------------------------------------------
\begin{frame}[fragile]
  \frametitle{نقشه راه موفقیت: ارکان کلیدی طرح}
  \begin{itemize}
    \item دانش‌آموز (مرکز و محور اصلی)
    \item والدین (حامیان و پشتیبانان)
    \item دبیران (راهنمایان علمی و تخصصی)
    \item مشاور (همراه، تسهیل‌گر و برنامه‌ریز)
    \item برنامه جامع دبیرستان پیام (چارچوب و مسیر)
  \end{itemize}
\end{frame}

% ----------------------------------------------------------------------
% SLIDE 4: PHASE 1: START AND INFRASTRUCTURE (SUMMER & EARLY YEAR)
% ----------------------------------------------------------------------
\begin{frame}[fragile]
  \frametitle{گام‌های نخست: بنا نهادن پایه موفقیت}
  \textbf{تابستان و اوایل سال}
  \begin{itemize}
    \item جلسه توجیهی جامع و انگیزشی (دانش‌آموزان، والدین، دبیران).
    \item مشاوره فردی اولیه: شناخت و تدوین "نقشه راه شخصی".
    \item کارگاه‌های آموزشی برای والدین: ایجاد محیط حمایتی در خانه.
    \item همکاری نزدیک با دبیران: هماهنگی و هم‌افزایی.
  \end{itemize}
\end{frame}

% ----------------------------------------------------------------------
% SLIDE 5: DETAILS OF INITIAL INDIVIDUAL CONSULTATION
% ----------------------------------------------------------------------
\begin{frame}[fragile]
  \frametitle{شناخت عمیق، برنامه دقیق: نقشه راه شخصی شما}
  \textbf{اهداف مشاوره اولیه:}
  \begin{itemize}
    \item شناخت وضعیت تحصیلی، نقاط قوت و ضعف.
    \item تعیین اهداف کنکوری و علایق.
    \item بررسی چالش‌های احتمالی.
  \end{itemize}
  \medskip
  \textbf{خروجی‌ها:}
  \begin{itemize}
    \item اهداف SMART (مشخص، قابل اندازه‌گیری، قابل دستیابی، مرتبط، زمان‌بندی شده).
    \item برنامه مطالعاتی اولیه (با تاکید بر تابستان).
    \item برنامه‌ریزی برای رفع نقاط ضعف پایه‌ای.
  \end{itemize}
\end{frame}

% ----------------------------------------------------------------------
% SLIDE 6: PHASE 2: EXECUTION, MONITORING, AND SUPPORT (THROUGHOUT THE ACADEMIC YEAR)
% ----------------------------------------------------------------------
\begin{frame}[fragile]
  \frametitle{در مسیر پیشرفت: یادگیری، تلاش، و همراهی}
  \textbf{طول سال تحصیلی}
  \begin{itemize}
    \item جلسات مشاوره فردی و گروهی منظم (تحلیل آزمون، رفع اشکال، مهارت‌آموزی).
    \item پایش مستمر عملکرد (آزمون‌های گزینه ۲، گزارش دبیران).
    \item آموزش و پیگیری مهارت‌های تحصیلی (خلاصه‌نویسی، تست‌زنی، مدیریت زمان و...).
    \item ارتباط مستمر و مؤثر با والدین.
    \item همکاری در برنامه‌های ویژه مدرسه (جمع‌بندی، شبیه‌ساز کنکور).
  \end{itemize}
\end{frame}

% ----------------------------------------------------------------------
% SLIDE 7: IMPORTANCE OF STUDY SKILLS
% ----------------------------------------------------------------------
\begin{frame}[fragile]
  \frametitle{ابزارهای موفقیت: فراتر از دانش صرف}
  \textbf{چرا مهارت‌های تحصیلی مهم هستند؟} (افزایش کارایی، کاهش استرس، یادگیری عمیق‌تر).
  \medskip
  \textbf{نمونه‌هایی از مهارت‌های کلیدی که آموزش داده خواهند شد:}
  \begin{itemize}
    \item هدف‌گذاری دقیق کنکوری و مدیریت انتظارات.
    \item بهینه‌سازی زمان (تکنیک پومودورو، مدیریت انرژی).
    \item خواندن تحلیلی و انتقادی.
    \item یکپارچه‌سازی یادداشت‌ها و خلاصه‌نویسی کنکوری.
    \item مدیریت استرس و حفظ تمرکز پایدار.
    \item استراتژی‌های یادگیری تطبیقی.
    \item و...
  \end{itemize}
\end{frame}

% ----------------------------------------------------------------------
% SLIDE 8: PHASE 3: REVIEW AND FINAL PREPARATION (FINAL MONTHS)
% ----------------------------------------------------------------------
\begin{frame}[fragile]
  \frametitle{خط پایان: تمرکز، آرامش، و آمادگی حداکثری}
  \textbf{ماه‌های پایانی}
  \begin{itemize}
    \item مشاوره فشرده و هدفمند (برنامه جمع‌بندی، مرور سریع، رفع اشکال).
    \item حمایت روانی ویژه (مدیریت اضطراب، افزایش اعتماد به نفس).
    \item همراهی تا روز کنکور و پس از آن (انتخاب رشته آگاهانه).
  \end{itemize}
\end{frame}

% ----------------------------------------------------------------------
% SLIDE 9: ROLE OF THE COUNSELOR
% ----------------------------------------------------------------------
\begin{frame}[fragile]
  \frametitle{مشاور شما: تسهیل‌گر، راهنما، پشتیبان}
  \begin{itemize}
    \item \textbf{تسهیل‌گر:} ایجاد ارتباط مؤثر بین دانش‌آموز، والدین و دبیران.
    \item \textbf{راهنما:} ارائه اطلاعات، آموزش مهارت‌ها، کمک در تصمیم‌گیری.
    \item \textbf{همراه و پشتیبان:} حمایت عاطفی، ایجاد انگیزه، کمک به عبور از چالش‌ها.
    \item \textbf{ناظر و تحلیل‌گر:} پایش پیشرفت، تحلیل نتایج، ارائه بازخوردهای سازنده.
    \item \textbf{متخصص:} استفاده از دانش و تجربه برای ارائه بهترین راهکارها.
  \end{itemize}
\end{frame}

% ----------------------------------------------------------------------
% SLIDE 10: KEYS TO EXTRAORDINARY SUCCESS
% ----------------------------------------------------------------------
\begin{frame}[fragile]
  \frametitle{با هم برای بهترین نتیجه}
  \begin{itemize}
    \item شفافیت و ارتباط مستمر: همه در جریان باشیم.
    \item انعطاف‌پذیری: برنامه برای شماست، نه شما برای برنامه.
    \item تأکید بر فرآیند، نه فقط نتیجه: رشد و یادگیری پایدار.
    \item ایجاد فضای مثبت و حمایتی: مدرسه، خانه دوم شما.
    \item جشن گرفتن موفقیت‌های کوچک: هر گام، یک پیروزی.
  \end{itemize}
\end{frame}

% ----------------------------------------------------------------------
% SLIDE 11: Q&A
% ----------------------------------------------------------------------
\begin{frame}[fragile]
  \frametitle{آماده شنیدن شما هستیم}
  \Large % Make text a bit larger for this slide
  \centering
  فرصت برای طرح سوالات و ابهامات.
  \bigskip
  \normalsize % Back to normal size for contact info
  \textbf{اطلاعات تماس مشاور:}
  \begin{itemize}
      \item ایمیل: [آدرس ایمیل مشاور]
      \item شماره تلفن دفتر مشاوره مدرسه: [شماره تلفن]
  \end{itemize}
  \vfill % Pushes content down if there's space
\end{frame}

% ----------------------------------------------------------------------
% SLIDE 12: THANK YOU
% ----------------------------------------------------------------------
\begin{frame}[plain] % Keep plain for a clean final slide
  \centering % Center all content on this slide
  \vspace*{\fill} % Push content towards vertical center

  \textcolor{myblue}{\Huge\textbf{با آرزوی موفقیت برای همه شما}}
  \vspace{0.75cm}

  \textcolor{mygold!90!black}{\Large از حضور و توجه شما سپاسگزاریم.} % Slightly darker gold
  \vspace{1cm}

  \begin{block}{} % Use a Beamer block for the quote; empty title for no explicit block title
    \centering
    \Large\textit{"آینده از آن کسانی است که به زیبایی رویاهایشان ایمان دارند."}
  \end{block}
  \setbeamercolor{block body}{bg=mygold!10,fg=black} % Subtle background for the block
  \setbeamercolor{block title}{bg=myblue,fg=white} % In case you add a title to the block later

  \vspace{1cm}
  \textcolor{mygold}{\Large\textbf{دبیرستان پیام}}
  \vspace{0.25cm}
  \textcolor{myblue!75!black}{\normalsize\today}

  \vspace*{\fill} % Balance vertical spacing
\end{frame}

% ----------------------------------------------------------------------
% END DOCUMENT
% ----------------------------------------------------------------------
\end{document}