\documentclass[12pt,a4paper]{article}

% Packages for math, lists, geometry, and title formatting
\usepackage{amsmath}
\usepackage{amssymb}
\usepackage{enumitem} % For customized lists
\usepackage{geometry}
\geometry{a4paper, margin=2.5cm, top=2cm, bottom=2cm} % Adjust margins as needed

\usepackage{titlesec} % To customize section titles
\titleformat{\section}{\normalfont\Large\bfseries\filcenter}{\thesection}{1em}{}
\titleformat{\subsection}{\normalfont\large\bfseries}{\thesubsection}{1em}{}
\titleformat{\subsubsection}{\normalfont\normalsize\bfseries}{\thesubsubsection}{1em}{}

% Persian language support with xepersian and Amiri font
\usepackage{xepersian}
\setmainfont{Amiri}
\setdigitfont{Amiri} % Optional: use Amiri for digits as well
\setsansfont{Amiri}  % Optional: use Amiri for sans-serif text
\settextfont{Amiri}  % Set the font for Persian text

% Custom command for study units for better structure (optional)
\newcommand{\studyunit}[1]{\par\medskip\noindent\textbf{#1}\par\nopagebreak}
\newcommand{\topics}{\par\medskip\noindent\textbf{مباحث:}\begin{itemize}[nosep,after=\vspace{-0.5\baselineskip}]}
\newcommand{\activities}{\par\medskip\noindent\textbf{فعالیت:}\begin{itemize}[nosep,after=\vspace{-0.5\baselineskip}]}
\newcommand{\breaktime}[1]{\par\smallskip\centerline{\textit{#1}}\smallskip}

\begin{document}

\begin{center}
\Large\textbf{برنامه مطالعاتی دقیق برای کسب نمره قبولی ریاضی ۱ (جمعه ۲۶ اردیبهشت تا یکشنبه ۲۸ اردیبهشت)}
\end{center}
\hrulefill
\vspace{1em}

\section*{اولویت‌بندی مباحث بر اساس سادگی و احتمال بالای کسب نمره حداقلی:}
\begin{enumerate}[label=\arabic*., itemsep=0.2em, topsep=0.3em]
    \item \textbf{فصل ۱ (مجموعه، الگو و دنباله):} تشخیص مجموعه و بازه، نوشتن چند جمله اول دنباله حسابی.
    \item \textbf{فصل ۲ (مثلثات):} حفظ مقادیر سینوس و کسینوس زوایای ۳۰، ۴۵، ۶۰.
    \item \textbf{فصل ۴ (معادله‌ها و نامعادله‌ها):} حل معادله درجه دوم فقط با فرمول دلتا (در حد جایگذاری و محاسبه).
    \item \textbf{فصل ۵ (تابع):} تشخیص تابع از روی زوج مرتب و نمودار، پیدا کردن مقدار تابع از ضابطه ساده.
    \item \textbf{فصل ۳ (توان‌های گویا و عبارت‌های جبری):} قوانین ساده توان و اتحاد مربع.
    \item \textbf{فصل ۶ و ۷ (شمارش و احتمال):} مسائل بسیار ساده اصل ضرب و محاسبه احتمال در پرتاب سکه/تاس.
\end{enumerate}
\vspace{1em}

\section*{برنامه مطالعاتی دقیق (جمعه ۲۶ اردیبهشت تا یکشنبه ۲۸ اردیبهشت):}
\rule{\linewidth}{0.4pt}\vspace{1em}

\section*{روز ۱: جمعه ۲۶ اردیبهشت}

\subsection*{فرجه صبح (۲ واحد مطالعاتی):}
    \studyunit{واحد ۱ (۹۰ دقیقه): فصل ۱ - مجموعه، الگو و دنباله (فقط مباحث خیلی پایه)}
        \topics
            \item تعریف مجموعه، نمایش بازه‌ها روی محور، نوشتن ۲-۳ جمله اول یک دنباله حسابی با داشتن جمله اول و قدر نسبت (فقط با جمع کردن ساده).
        \end{itemize}
        \activities
            \item خواندن تعاریف از روی کتاب، حل ۲-۳ مثال بسیار ساده از کتاب برای هر مبحث.
        \end{itemize}

    \breaktime{استراحت (۱۵ دقیقه)}

    \studyunit{واحد ۲ (۹۰ دقیقه): فصل ۴ - معادله‌ها و نامعادله‌ها (فقط حل معادله درجه دوم با دلتا)}
        \topics
            \item حفظ فرمول دلتا ($\Delta = b^2 - 4ac$) و فرمول ریشه‌ها ($x = \frac{-b \pm \sqrt{\Delta}}{2a}$).
        \end{itemize}
        \activities
            \item حل ۴-۵ مثال ساده کتاب که مستقیماً با جایگذاری در فرمول دلتا حل می‌شوند. تمرکز فقط بر محاسبه صحیح دلتا و ریشه‌ها. (از مسائل کاربردی و پیچیده صرف نظر شود).
        \end{itemize}

\subsection*{فرجه عصر (۲ واحد مطالعاتی):}
    \studyunit{واحد ۱ (۹۰ دقیقه): فصل ۲ - مثلثات (فقط مقادیر زوایای خاص و تشخیص علامت)}
        \topics
            \item حفظ جدول مقادیر سینوس، کسینوس و تانژانت برای زوایای ۳۰، ۴۵ و ۶۰ درجه. تشخیص علامت نسبت‌های مثلثاتی در چهار ناحیه دایره مثلثاتی (فقط حفظ کردن "همه مثبت، سینوس مثبت، تانژانت و کتانژانت مثبت، کسینوس مثبت").
        \end{itemize}
        \activities
            \item چندین بار نوشتن جدول مقادیر و تکرار آن. کشیدن دایره مثلثاتی و مشخص کردن نواحی و علامت‌ها.
        \end{itemize}

    \breaktime{استراحت (۱۵ دقیقه)}

    \studyunit{واحد ۲ (۹۰ دقیقه): فصل ۵ - تابع (فقط تشخیص تابع از زوج مرتب/نمودار و مقدار تابع)}
        \topics
            \item تعریف تابع به زبان ساده (به هر ورودی دقیقاً یک خروجی نسبت داده شود). تشخیص تابع بودن از روی مجموعه زوج‌های مرتب (عدم وجود زوج مرتب با مؤلفه اول یکسان و مؤلفه دوم متفاوت). آزمون خط عمودی برای نمودار. محاسبه مقدار تابع $y=f(x)$ برای یک $x$ عددی ساده (مثلاً $f(x)=2x+1$، مقدار $f(2)$ را بیابید).
        \end{itemize}
        \activities
            \item حل چند مثال از کتاب برای تشخیص تابع. حل ۲-۳ تمرین ساده برای محاسبه مقدار تابع.
        \end{itemize}

\rule{\linewidth}{0.4pt}\vspace{1em}

\section*{روز ۲: شنبه ۲۷ اردیبهشت}

\subsection*{فرجه صبح (۱ واحد مطالعاتی + ۱ واحد مرور):}
    \studyunit{واحد ۱ (۹۰ دقیقه): فصل ۳ - توان‌های گویا و عبارت‌های جبری (فقط قوانین ساده توان و اتحاد مربع)}
        \topics
            \item یادآوری قوانین ضرب و تقسیم توان‌ها با پایه‌های یکسان یا توان‌های یکسان (فقط اعداد صحیح). اتحاد مربع دوجمله‌ای $(a \pm b)^2 = a^2 \pm 2ab + b^2$.
        \end{itemize}
        \activities
            \item حل چند مثال بسیار ساده برای کاربرد قوانین توان. باز و بسته کردن چند عبارت ساده با اتحاد مربع.
        \end{itemize}

    \breaktime{استراحت (۱۵ دقیقه)}

    \studyunit{واحد ۲ (۹۰ دقیقه): مرور مطالب روز گذشته}
        \activities
            \item خیلی سریع مباحثی که دیروز خوانده‌اید را از روی یادداشت‌ها یا کتاب مرور کنید. سعی کنید یکی دو مثال از هر مبحث را دوباره برای خودتان حل کنید (دنباله حسابی، دلتا، مقادیر مثلثاتی، تشخیص تابع).
        \end{itemize}

\subsection*{فرجه عصر (۱ واحد مطالعاتی + ۱ واحد مرور):}
    \studyunit{واحد ۱ (۹۰ دقیقه): فصل ۶ و ۷ - شمارش و احتمال (فقط مفاهیم بسیار اولیه و مثال‌های ساده)}
        \par\medskip\noindent\textbf{مباحث فصل ۶:}
        \begin{itemize}[nosep,after=\vspace{-0.5\baselineskip}]
            \item اصل ضرب در شمارش تعداد حالت‌ها در ۱-۲ مرحله (مثلاً انتخاب لباس با داشتن چند پیراهن و چند شلوار).
        \end{itemize}
        \par\medskip\noindent\textbf{مباحث فصل ۷:}
        \begin{itemize}[nosep,after=\vspace{-0.5\baselineskip}]
            \item نوشتن تمام حالت‌های ممکن (فضای نمونه‌ای) در پرتاب ۱ یا ۲ سکه، یا ۱ تاس. محاسبه احتمال یک پیشامد ساده (تعداد حالات مطلوب تقسیم بر تعداد کل حالات).
        \end{itemize}
        \activities
            \item حل ۲-۳ مثال بسیار ساده از کتاب برای اصل ضرب و محاسبه احتمال.
        \end{itemize}

    \breaktime{استراحت (۱۵ دقیقه)}

    \studyunit{واحد ۲ (۹۰ دقیقه): مرور کلی تمام مباحث خوانده شده تا اینجا}
        \activities
            \item یک دور خیلی سریع تمام فرمول‌ها و مثال‌های ساده‌ای که در این دو روز کار کرده‌اید را مرور کنید.
        \end{itemize}

\rule{\linewidth}{0.4pt}\vspace{1em}

\section*{روز ۳: یکشنبه ۲۸ اردیبهشت (روز مرور نهایی و حل چند سوال کلیدی)}

\subsection*{فرجه صبح (۲ واحد مطالعاتی):}
    \studyunit{واحد ۱ (۹۰ دقیقه): حل چند سوال بسیار ساده و پرتکرار از نمونه سوالات نهایی}
        \activities
            \item از یک نمونه سوال امتحان نهایی، فقط سوالات بسیار ساده و مستقیمی که مربوط به مباحث خوانده شده هستند را پیدا کنید و سعی کنید حل کنید (مثلاً: یک معادله درجه دوم بدهند و فقط ریشه‌ها را با دلتا بخواهند، مقدار سینوس ۳۰ درجه را بپرسند، یک مجموعه زوج مرتب بدهند و بگویند تابع است یا نه).
        \end{itemize}

    \breaktime{استراحت (۱۵ دقیقه)}

    \studyunit{واحد ۲ (۹۰ دقیقه): ادامه حل سوالات ساده و مرور فرمول‌ها}
        \activities
            \item ادامه حل سوالات بسیار ساده از نمونه‌های دیگر. مرور فرمول دلتا، جدول مقادیر مثلثاتی، و روش اصل ضرب.
        \end{itemize}

\subsection*{فرجه عصر (فقط ۱ واحد مرور سبک):}
    \studyunit{واحد ۱ (۹۰ دقیقه): مرور نهایی و آماده‌سازی برای امتحان}
        \activities
            \item یک بار دیگر خیلی سریع فقط تیترها، فرمول‌های اصلی و مثال‌های خیلی ساده‌ای که علامت زده‌اید را نگاه کنید.
            \item \textbf{به هیچ عنوان سراغ مطلب جدید یا سوال سخت نروید.}
            \item سعی کنید اعتماد به نفس خود را حفظ کنید. هدف شما کسب نمره قبولی است و با تمرکز روی همین مطالب ساده، امکان‌پذیر است.
            \item وسایل لازم برای امتحان را آماده کنید.
        \end{itemize}

\rule{\linewidth}{0.4pt}\vspace{1em}

\section*{اشتباهات مهلک برای دانش‌آموزان با هدف نمره قبولی:}
\begin{enumerate}[label=\arabic*., itemsep=0.2em, topsep=0.3em]
    \item \textbf{تلاش برای یادگیری تمام مطالب:} این کار در زمان کوتاه و با پایه ضعیف، فقط باعث سردرگمی و استرس می‌شود. روی همین مباحث محدود و ساده تمرکز کنید.
    \item \textbf{حفظ کردن فرمول بدون اینکه بدانید کجا و چطور استفاده کنید:} حداقل کاربرد فرمول در یک مثال ساده را یاد بگیرید.
    \item \textbf{وحشت از ریاضی و ناامیدی:} با تمرکز روی بخش‌های آسان‌تر، می‌توانید نمره قبولی را کسب کنید.
    \item \textbf{صرف وقت زیاد روی یک سوال سخت در امتحان:} اگر سوالی سخت به نظر می‌رسد، از آن عبور کنید و به سوالات ساده‌تری که بلد هستید پاسخ دهید.
\end{enumerate}

\vspace{1em}
\rule{\linewidth}{0.4pt}\vspace{1em}

\section*{نکات طلایی برای دانش‌آموزان با هدف نمره قبولی:}
\begin{enumerate}[label=\arabic*., itemsep=0.2em, topsep=0.3em]
    \item \textbf{تمرکز، تمرکز، تمرکز:} فقط روی مباحثی که در این برنامه آمده و مثال‌های مشابه کتاب تمرکز کنید.
    \item \textbf{مثال‌های حل شده کتاب:} این مثال‌ها بهترین راهنما برای شما هستند. سعی کنید آن‌ها را بفهمید.
    \item \textbf{تکرار و تمرین محدود:} همان چند مثال ساده از هر مبحث را چندین بار حل کنید تا روش آن را یاد بگیرید.
    \item \textbf{حفظ کردن دقیق موارد مشخص شده:} جدول مقادیر مثلثاتی و فرمول دلتا باید دقیق حفظ شوند.
    \item \textbf{مثبت‌اندیشی:} باور داشته باشید که می‌توانید نمره قبولی را کسب کنید.
\end{enumerate}

\end{document}