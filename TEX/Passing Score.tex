\documentclass[12pt,a4paper]{article}

% ---------------- Basic Packages (before XePersian) ----------------
\usepackage{amsmath} % For mathematical expressions (though not heavily used here)
\usepackage[right=2.5cm, left=2.5cm, top=2.5cm, bottom=2.5cm]{geometry} % Page margins
\usepackage{enumitem} % For customizing lists
\usepackage{titlesec} % For customizing section titles
\usepackage{setspace} % For line spacing, if needed (e.g., \onehalfspacing)
\usepackage{ragged2e} % For \Centering, \RaggedRight, \RaggedLeft, \justifying

% ---------------- XePersian and Font Setup (XePersian LAST) ----------------
\usepackage{xepersian}
\settextfont[Scale=1.1]{Amiri} % Set Amiri as the main font, slightly larger
\setdigitfont[Scale=1.1]{Amiri} % Set Amiri for digits as well

% ---------------- Customizations ----------------
% Title formatting
\titleformat{\section}
  {\normalfont\Large\bfseries\centering}{\thesection}{1em}{}
\titlespacing*{\section}{0pt}{3.5ex plus 1ex minus .2ex}{2.3ex plus .2ex}

\titleformat{\subsection}
  {\normalfont\large\bfseries\RaggedRight}{\thesubsection}{1em}{} % Subsections not centered
\titlespacing*{\subsection}{0pt}{2.5ex plus 1ex minus .2ex}{1.5ex plus .2ex}

% List settings for more compact lists
\setlist{nosep, itemsep=0.2em, topsep=0.3em}
\setlist[itemize,1]{label=\textbullet}
\setlist[itemize,2]{label=--}

% Command for Quranic verses (simple italic for now)
\newcommand{\quranverse}[1]{\textit{#1}}
\newcommand{\hadith}[1]{\textit{#1}}

% Horizontal rule
\newcommand{\sectionbreak}{\noindent\hrulefill\par\vspace{0.5em}}

% No page numbering for a plan like this
\pagestyle{empty}

% ---------------- Document ----------------
\begin{document}

\begin{center}
    \bfseries\Huge برنامه فشرده دین و زندگی ۳ (ویژه کسب نمره قبولی)
\end{center}
\vspace{1em}

\textbf{اصول کلی برنامه:}
\begin{itemize}
    \item تمرکز بر مفاهیم پایه و پرتکرار: شناسایی و مطالعه دقیق بخش‌هایی که معمولاً بیشترین سهم را در سؤالات امتحانات نهایی دارند.
    \item اولویت با درک کلی، نه جزئیات پیچیده: هدف، فهمیدن پیام اصلی دروس و آیات است، نه درگیر شدن با نکات بسیار ریز و تحلیلی.
    \item استفاده بهینه از زمان: تمرکز بر مطالعه مؤثر و پرهیز از اتلاف وقت.
    \item حل نمونه سؤالات ساده‌تر: آشنایی با حداقل‌های مورد نیاز برای پاسخگویی.
\end{itemize}

\sectionbreak
\vspace{1em}

\section*{پنجشنبه، 25 اردیبهشت}

\subsection*{فرجه صبح (حدود 3 ساعت): بخش اول: تفکر و اندیشه - پایه‌های توحید و اختیار}

\textbf{1. واحد اول (1 ساعت و 30 دقیقه): درس ۱: هستی‌بخش و نکات کلیدی درس ۲: یگانه بی‌همتا}
\begin{itemize}
    \item \textbf{تمرکز ویژه:}
    \begin{itemize}
        \item مفهوم ساده نیازمندی جهان به خدا در پیدایش (درس 1).
        \item تعریف ساده توحید در خالقیت و توحید در ربوبیت (درس 2).
        \item پیام اصلی سوره توحید (اخلاص) (درس 2).
    \end{itemize}
    \item \textbf{روش مطالعه:}
    \begin{itemize}
        \item خواندن روان متن درس 1 با تمرکز بر مثال‌ها.
        \item از درس 2، فقط تعاریف توحید در خالقیت و ربوبیت را به زبان ساده یاد بگیرید.
        \item ترجمه و پیام ساده سوره توحید را به خاطر بسپارید.
        \item به سؤالات ساده «تفکر در متن» پاسخ دهید.
    \end{itemize}
\end{itemize}

\textbf{2. استراحت (15 دقیقه)}
\vspace{0.5em}

\textbf{3. واحد دوم (1 ساعت و 30 دقیقه): درس ۵: قدرت پرواز (بخش‌های کلیدی)}
\begin{itemize}
    \item \textbf{تمرکز ویژه:}
    \begin{itemize}
        \item مفهوم اختیار و مثال‌هایی که نشان‌دهنده وجود اختیار در انسان است (تفکر و تصمیم، پشیمانی).
        \item تعریف ساده قضا و قدر الهی.
        \item نکته اصلی: قضا و قدر الهی با اختیار انسان منافات ندارد (در حد درک کلی این جمله).
    \end{itemize}
    \item \textbf{روش مطالعه:}
    \begin{itemize}
        \item بخش‌های ابتدایی درس در مورد نشانه‌های اختیار را بخوانید.
        \item تعریف قضا و قدر را یاد بگیرید.
        \item فقط این نکته را به خاطر بسپارید که اعتقاد به قضا و قدر به معنای مجبور بودن ما نیست.
        \item آیه ۳ سوره انسان (\quranverse{إِنَّا هَدَیْنَاهُ السَّبِیلَ...}) را با ترجمه و پیام کلی آن (خدا راه را نشان داده و انسان در انتخاب آزاد است) یاد بگیرید.
    \end{itemize}
\end{itemize}

\sectionbreak
\vspace{1em}

\subsection*{فرجه عصر (حدود 3 ساعت): بخش اول: تفکر و اندیشه - سنت‌های الهی (مهم‌ترین‌ها) و بخش دوم: در مسیر - توبه}

\textbf{1. واحد اول (1 ساعت و 30 دقیقه): درس ۶: سنت‌های خداوند در زندگی (مهم‌ترین سنت‌ها)}
\begin{itemize}
    \item \textbf{تمرکز ویژه:}
    \begin{itemize}
        \item تعریف ساده سنت الهی.
        \item سنت ابتلاء (امتحان): اینکه همه انسان‌ها امتحان می‌شوند.
        \item تأثیر اعمال انسان در زندگی: اینکه اعمال خوب و بد ما در زندگی‌مان اثر دارد.
    \end{itemize}
    \item \textbf{روش مطالعه:}
    \begin{itemize}
        \item تعریف سنت الهی را یاد بگیرید.
        \item بخش مربوط به سنت ابتلاء را بخوانید و بفهمید که هدف از امتحان الهی چیست.
        \item بخش مربوط به تأثیر اعمال را بخوانید و یک مثال برای آن به خاطر بسپارید.
        \item آیه ۱۱ سوره رعد (\quranverse{إِنَّ اللَّهَ لَا یُغَیِّرُ مَا بِقَوْمٍ...}) را با ترجمه و پیام کلی آن (تغییر سرنوشت به دست خود انسان‌هاست) یاد بگیرید.
    \end{itemize}
\end{itemize}

\textbf{2. استراحت (15 دقیقه)}
\vspace{0.5em}

\textbf{3. واحد دوم (1 ساعت و 30 دقیقه): درس ۷: بازگشت (مباحث کلیدی توبه)}
\begin{itemize}
    \item \textbf{تمرکز ویژه:}
    \begin{itemize}
        \item حقیقت توبه (پشیمانی از گناه و تصمیم بر ترک آن).
        \item ضرورت توبه و اینکه خداوند توبه را می‌پذیرد.
        \item آثار توبه (آمرزش گناهان، ایجاد امید).
    \end{itemize}
    \item \textbf{روش مطالعه:}
    \begin{itemize}
        \item تعریف توبه و ارکان اصلی آن را یاد بگیرید.
        \item بدانید که خداوند توبه‌پذیر است.
        \item آیه ۵۳ سوره زمر (\quranverse{قُلْ یَا عِبَادِیَ الَّذِینَ أَسْرَفُوا...}) را با ترجمه و پیام کلی آن (ناامید نشدن از رحمت خدا و آمرزش همه گناهان با توبه) به خوبی یاد بگیرید.
        \item حدیث پیامبر (ص) (\hadith{"التَّائِبُ مِنَ الذَّنْبِ کَمَنْ لَا ذَنْبَ لَهُ"}) را به خاطر بسپارید.
    \end{itemize}
\end{itemize}

\pagebreak % Start new page for next day
\section*{جمعه، 26 اردیبهشت}

\subsection*{فرجه صبح (حدود 3 ساعت): بخش دوم: در مسیر - احکام (منتخب) و مرور}

\textbf{1. واحد اول (1 ساعت و 30 دقیقه): درس ۸: احکام الهی در زندگی امروز (بخش‌های قابل فهم‌تر)}
\begin{itemize}
    \item \textbf{تمرکز ویژه:}
    \begin{itemize}
        \item حکمت کلی احکام الهی (برای مصلحت انسان و جامعه).
        \item حکم موسیقی حرام (موسیقی لهوی و مناسب مجالس گناه).
        \item حرمت شرط‌بندی.
        \item حرمت ربا.
    \end{itemize}
    \item \textbf{روش مطالعه:}
    \begin{itemize}
        \item بفهمید که احکام دینی برای خیر و صلاح ماست.
        \item فقط بدانید چه نوع موسیقی حرام است.
        \item بدانید شرط‌بندی حرام است (جز موارد استثناء که نیازی به حفظ دقیق آن‌ها نیست).
        \item بدانید ربا حرام است و یکی از دلایل آن (مثلاً جلوگیری از ظلم اقتصادی) را به زبان ساده یاد بگیرید.
        \item آیه ۲۱۹ سوره بقره (\quranverse{یَسْأَلُونَکَ عَنِ الْخَمْرِ وَالْمَیْسِرِ...}) و پیام کلی آن (شراب و قمار گناه بزرگی دارند) را بدانید.
    \end{itemize}
\end{itemize}

\textbf{2. استراحت (15 دقیقه)}
\vspace{0.5em}

\textbf{3. واحد دوم (1 ساعت و 30 دقیقه): مرور سریع دروس ۱، ۲، ۵ و ۶ (بخش‌های خوانده شده)}
\begin{itemize}
    \item \textbf{روش مرور:}
    \begin{itemize}
        \item مرور نکات کلیدی و تعاریف ساده‌ای که یادداشت کرده‌اید.
        \item یادآوری پیام اصلی آیاتی که مشخص شده است.
        \item سعی کنید مفاهیم را یک بار دیگر برای خودتان به زبان ساده توضیح دهید.
    \end{itemize}
\end{itemize}

\sectionbreak
\vspace{1em}

\subsection*{فرجه عصر (حدود 3 ساعت): بخش دوم: در مسیر - پایه‌های تمدن (مهم‌ترین‌ها) و مرور}

\textbf{1. واحد اول (1 ساعت و 30 دقیقه): درس ۹: پایه‌های استوار (مهم‌ترین پایه‌ها)}
\begin{itemize}
    \item \textbf{تمرکز ویژه:}
    \begin{itemize}
        \item توحید به عنوان زیربنا.
        \item عدالت اجتماعی و اهمیت آن.
        \item علم و دانش و تأکید اسلام بر آن.
        \item خانواده و نقش آن.
    \end{itemize}
    \item \textbf{روش مطالعه:}
    \begin{itemize}
        \item برای هر یک از این چهار پایه، یک توضیح کوتاه و ساده یاد بگیرید.
        \item آیه ۲۵ سوره حدید (\quranverse{لَقَدْ أَرْسَلْنَا رُسُلَنَا بِالْبَیِّنَاتِ...}) و پیام اصلی آن (هدف از ارسال رسولان، برپایی قسط و عدالت است) را به خوبی یاد بگیرید.
    \end{itemize}
\end{itemize}

\textbf{2. استراحت (15 دقیقه)}
\vspace{0.5em}

\textbf{3. واحد دوم (1 ساعت و 30 دقیقه): مرور سریع دروس ۳، ۴، ۷ و ۸ (بخش‌های خوانده شده)}
\begin{itemize}
    \item \textbf{روش مرور:} مشابه مرور قبلی.
    \begin{itemize}
        \item تمرکز بر یادآوری تعاریف (اخلاص، ریا، توبه) و احکام اصلی.
    \end{itemize}
\end{itemize}

\pagebreak % Start new page for next day
\section*{شنبه، 27 اردیبهشت}

\subsection*{فرجه صبح (حدود 3 ساعت): مرور کلی و حل نمونه سؤالات ساده}

\textbf{1. واحد اول (1 ساعت و 30 دقیقه): مرور بسیار سریع تمام مباحث کلیدی خوانده شده از همه دروس.}
\begin{itemize}
    \item \textbf{روش مرور:} یک نگاه گذرا به تمام یادداشت‌ها و نکات مهم. سعی کنید ارتباط کلی بین دروس را پیدا کنید (مثلاً چگونه توحید بر همه جنبه‌های زندگی اثر می‌گذارد).
\end{itemize}

\textbf{2. استراحت (15 دقیقه)}
\vspace{0.5em}

\textbf{3. واحد دوم (1 ساعت و 30 دقیقه): حل نمونه سؤالات ساده و پرتکرار}
\begin{itemize}
    \item \textbf{روش:} از انتهای کتاب درسی یا نمونه سؤالات سال‌های قبل، سؤالات کوتاه پاسخ، جای خالی و صحیح/غلط مربوط به مباحثی که خوانده‌اید را پیدا کرده و حل کنید.
    \item به سؤالات مربوط به ترجمه و پیام آیات مهم (که در برنامه مشخص شده) پاسخ دهید.
\end{itemize}

\sectionbreak
\vspace{1em}

\subsection*{فرجه عصر (حدود 3 ساعت): جمع‌بندی نهایی و رفع اشکال}

\textbf{1. واحد اول و دوم (مجموعاً حدود 3 ساعت با یک استراحت کوتاه در صورت نیاز):}
\begin{itemize}
    \item مرور نهایی تمام تعاریف و پیام آیات مهم.
    \item تمرکز بر نقاطی که در حین مطالعه یا حل نمونه سؤال احساس ضعف کرده‌اید. آن بخش‌ها را یک بار دیگر به سرعت بخوانید.
    \item سعی کنید برای هر مفهوم کلیدی، یک مثال ساده در ذهن داشته باشید.
    \item مطمئن شوید که ترجمه روان آیات مشخص شده را به خوبی بلدید.
    \item آمادگی ذهنی برای امتحان: با یادآوری اینکه مطالب اصلی و پایه‌ای را خوانده‌اید، با آرامش و اعتماد به نفس برای امتحان آماده شوید.
\end{itemize}

\sectionbreak
\vspace{1em}

\textbf{اشتباهات مهلک در این روزها (برای کسی که فقط نمره قبولی می‌خواهد):}
\begin{enumerate}
    \item درگیر شدن با جزئیات و مباحث پیچیده: هدف شما فهم کلیات و پاسخ به سؤالات پایه‌ای است.
    \item تلاش برای حفظ کردن تمام آیات و روایات: فقط آیات و روایات کلیدی و پرتکرار را با ترجمه و پیام اصلی یاد بگیرید.
    \item صرف وقت زیاد برای دروس کم‌اهمیت‌تر (از منظر بارم قبولی): اولویت با مباحثی است که احتمال سؤال از آن‌ها بیشتر است.
    \item ایجاد استرس و نگرانی از عدم مطالعه کامل کتاب: به یاد داشته باشید که هدف شما قبولی است و با مطالعه بخش‌های کلیدی می‌توانید به این هدف برسید.
\end{enumerate}
\vspace{1em}

\textbf{نکات طلایی برای کسب نمره قبولی:}
\begin{enumerate}
    \item تمرکز، تمرکز، تمرکز: بر روی مباحثی که در این برنامه مشخص شده، تمرکز کامل داشته باشید.
    \item یادگیری به زبان ساده: سعی کنید مفاهیم را برای خودتان ساده‌سازی کنید.
    \item تکرار نکات کلیدی: تعاریف و پیام آیات مهم را چندین بار تکرار کنید.
    \item مثبت‌اندیشی و اعتماد به نفس: باور داشته باشید که می‌توانید نمره قبولی را کسب کنید.
\end{enumerate}

\end{document}