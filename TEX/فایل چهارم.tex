\documentclass[12pt]{article}
\usepackage{geometry}
\geometry{a4paper, margin=1in} % Adjust margins as needed

\usepackage{amsmath} % xepersian recommends loading amsmath before it
\usepackage{enumitem} % For more list control, good to have
\usepackage{xepersian}
\settextfont{Amiri} % Or any other Persian font you have installed and prefer
\SepMark{/} % Set slash as decimal separator
% \settextfont[Scale=1.1]{XB Niloofar} % Example of another font with scaling

% For unnumbered paragraph-like headings for units
\newcommand{\unithead}[1]{\par\vspace{1ex}\noindent\textbf{#1}\par\nopagebreak[4]\vspace{0.5ex}}
\newcommand{\休息}[1]{\par\centering\textit{#1}\par\vspace{1ex}} % For rest periods

\begin{document}

\section*{برنامه‌ریزی درسی برای نمره کامل امتحان فیزیک ۳ تجربی}
\addcontentsline{toc}{section}{برنامه‌ریزی درسی برای نمره کامل امتحان فیزیک ۳ تجربی}

\subsection*{پنجشنبه ۱ خرداد (عصر)}
\addcontentsline{toc}{subsection}{پنجشنبه ۱ خرداد (عصر) - نمره کامل تجربی}
فرجه عصر (حدود ۳ تا ۴ ساعت مطالعه مفید):

\unithead{واحد ۱ (۱ ساعت و ۳۰ دقیقه): فصل ۲ - دینامیک}
\noindent\textbf{مرور سریع: }قوانین نیوتون، انواع نیروها (وزن، عمودی، اصطکاک، کشش).
\par\noindent\textbf{حل مسئله: }تمرکز بر مسائل قانون دوم نیوتون (با نیروهای در یک راستا و عمود بر هم)، مسائل مربوط به نیروی اصطکاک (ایستایی و جنبشی). حداقل ۱۰-۱۲ مسئله متنوع حل شود.

\休息{استراحت (۱۵ دقیقه)}

\unithead{واحد ۲ (۱ ساعت و ۳۰ دقیقه): فصل ۲ - دینامیک (ادامه)}
\noindent\textbf{مرور سریع: }تکانه و قانون دوم بر حسب تکانه، نیروی گرانشی.
\par\noindent\textbf{حل مسئله: }مسائل مربوط به تکانه و برخورد، مسائل ساده نیروی گرانشی (محاسبه نیروی گرانش بین دو جسم). حداقل ۸-۱۰ مسئله.
\par\noindent\textbf{نکته مهم: }در این فرجه، بیشتر تمرکز روی یادآوری فرمول‌ها و روش حل مسائل پایه دینامیک است.

\subsection*{جمعه ۲ خرداد (صبح و عصر)}
\addcontentsline{toc}{subsection}{جمعه ۲ خرداد (صبح و عصر) - نمره کامل تجربی}
فرجه صبح (حدود ۳ تا ۴ ساعت مطالعه مفید):

\unithead{واحد ۱ (۱ ساعت و ۳۰ دقیقه): فصل ۱ - حرکت بر خط راست}
\noindent\textbf{مرور سریع: }مفاهیم سرعت و شتاب متوسط و لحظه‌ای، معادلات حرکت با شتاب ثابت، نمودارهای حرکت.
\par\noindent\textbf{حل مسئله: }تمرکز بر مسائل حرکت با شتاب ثابت (مسائل چند مرحله‌ای)، تحلیل نمودارهای سرعت-زمان (محاسبه جابجایی، شتاب، مسافت طی شده). حداقل ۱۰-۱۲ مسئله.

\休息{استراحت (۱۵ دقیقه)}

\unithead{واحد ۲ (۱ ساعت و ۳۰ دقیقه): فصل ۳ - نوسان و امواج}
\noindent\textbf{مرور سریع: }حرکت هماهنگ ساده (معادله مکان-زمان، دوره، بسامد)، انرژی در حرکت هماهنگ ساده.
\par\noindent\textbf{حل مسئله: }مسائل مربوط به محاسبه پارامترهای نوسان (دوره، بسامد، بسامد زاویه‌ای، دامنه) از روی معادله یا اطلاعات داده شده، مسائل انرژی مکانیکی نوسانگر. حداقل ۱۰-۱۲ مسئله.

\vspace{1em}
فرجه عصر (حدود ۳ تا ۴ ساعت مطالعه مفید):

\unithead{واحد ۳ (۱ ساعت و ۳۰ دقیقه): فصل ۳ - نوسان و امواج (ادامه)}
\noindent\textbf{مرور سریع: }مشخصه‌های موج (طول موج، دامنه، سرعت انتشار)، بازتاب و شکست موج.
\par\noindent\textbf{حل مسئله: }مسائل مربوط به رابطه سرعت، طول موج و بسامد موج، مسائل مربوط به قانون اسنل (شکست نور). حداقل ۸-۱۰ مسئله.

\休息{استراحت (۱۵ دقیقه)}

\unithead{واحد ۴ (۱ ساعت و ۳۰ دقیقه): فصل ۴ - آشنایی با فیزیک اتمی و هسته‌ای (بخش اتمی)}
\noindent\textbf{مرور سریع: }اثر فوتوالکتریک (تابع کار، بسامد آستانه)، انرژی فوتون، مدل اتم رادرفورد-بور (ترازهای انرژی، گسیل و جذب فوتون).
\par\noindent\textbf{حل مسئله: }مسائل مربوط به معادله فوتوالکتریک، محاسبه انرژی و طول موج فوتون‌های گسیل شده یا جذب شده. حداقل ۸-۱۰ مسئله.

\subsection*{شنبه ۳ خرداد (صبح و عصر) - فرجه ویژه نمونه سوال و جمع‌بندی}
\addcontentsline{toc}{subsection}{شنبه ۳ خرداد (صبح و عصر) - نمره کامل تجربی}
فرجه صبح (حدود ۳ تا ۴ ساعت مطالعه مفید):

\unithead{واحد ۱ (۱ ساعت و ۳۰ دقیقه): حل نمونه سوال امتحان نهایی (بخش اول)}
یک نمونه سوال کامل امتحان نهایی (ترجیحاً از سال‌های اخیر) را در شرایط آزمون حل کنید.
\par زمان‌بندی را رعایت کنید.

\休息{استراحت (۱۵ دقیقه)}

\unithead{واحد ۲ (۱ ساعت و ۳۰ دقیقه): تحلیل نمونه سوال و رفع اشکال (بخش اول)}
پاسخ‌های خود را با راهنمای تصحیح مقایسه کنید.
\par اشکالات و نقاط ضعف خود را شناسایی کنید.
\par به مباحثی که در آن‌ها مشکل داشته‌اید، مراجعه و مرور کنید. (اینجا محل رفع اشکال نقاط ضعف اعلام شده است)

\vspace{1em}
فرجه عصر (حدود ۳ تا ۴ ساعت مطالعه مفید):

\unithead{واحد ۳ (۱ ساعت و ۳۰ دقیقه): حل نمونه سوال امتحان نهایی (بخش دوم) یا تست‌های جامع}
یک نمونه سوال دیگر یا مجموعه‌ای از تست‌های جامع از فصول مختلف را حل کنید.

\休息{استراحت (۱۵ دقیقه)}

\unithead{واحد ۴ (۱ ساعت و ۳۰ دقیقه): مرور نهایی و جمع‌بندی قسمت‌های مهم}
\noindent\textbf{مرور سریع فرمول‌ها: }تمام فرمول‌های مهم هر فصل را یک بار دیگر مرور کنید.
\par\noindent\textbf{مرور نکات کلیدی: }نکاتی که در طول مطالعه یادداشت کرده‌اید یا در حاشیه کتاب نوشته‌اید را مرور کنید.
\par\noindent\textbf{تورق سریع کتاب: }یک بار دیگر به سرعت صفحات کتاب را ورق بزنید و به تیترها، شکل‌ها و نمودارهای مهم توجه کنید.
\par\noindent\textbf{رفع اشکال نهایی: }اگر هنوز در مبحثی احساس ضعف می‌کنید، آخرین تلاش خود را برای رفع آن انجام دهید.

\subsection*{اولویت‌بندی بر اساس بارم‌بندی و اهمیت}
\addcontentsline{toc}{subsection}{اولویت‌بندی - نمره کامل تجربی}
\begin{itemize}
    \item فصل ۲ - دینامیک: به دلیل بارم بالا و تنوع سوالات، اولویت اصلی با این فصل است. تسلط بر قوانین نیوتون و حل مسائل مربوط به نیروها بسیار مهم است.
    \item فصل ۱ - حرکت بر خط راست: مفاهیم پایه و نمودارها در این فصل اهمیت زیادی دارند و معمولاً سوالات محاسباتی از آن طرح می‌شود.
    \item فصل ۳ - نوسان و امواج: درک مفاهیم حرکت هماهنگ ساده و مشخصه‌های موج ضروری است. مسائل مربوط به انرژی و شکست موج نیز پرتکرار هستند.
    \item فصل ۴ - آشنایی با فیزیک اتمی و هسته‌ای: بخش اثر فوتوالکتریک و مدل بور معمولاً سوالات محاسباتی دارند. بخش هسته‌ای نیز شامل تعاریف و معادلات واپاشی است.
\end{itemize}

\subsection*{اشتباهات مهلک در این روزها}
\addcontentsline{toc}{subsection}{اشتباهات مهلک - نمره کامل تجربی}
\begin{itemize}
    \item شروع مطالعه مباحث جدید: در این زمان کوتاه، به هیچ وجه سراغ یادگیری مباحثی که قبلاً نخوانده‌اید نروید. تمرکز فقط باید روی مرور و تثبیت آموخته‌ها باشد.
    \item حل نکردن نمونه سوالات نهایی: عدم آشنایی با سبک سوالات امتحان نهایی می‌تواند بسیار آسیب‌زا باشد.
    \item وسواس بیش از حد روی یک مبحث: اگر در مبحثی مشکل دارید، زمان زیادی را صرف آن نکنید. سعی کنید به سایر مباحث مسلط شوید و در فرجه آخر به رفع اشکال آن بپردازید.
    \item عدم توجه به استراحت: مطالعه پیوسته و بدون استراحت باعث کاهش بازدهی می‌شود. استراحت‌های کوتاه و منظم ضروری هستند.
    \item ایجاد استرس و اضطراب بیهوده: به خودتان اعتماد داشته باشید. شما درس را خوانده‌اید و این فرجه برای مرور و آمادگی نهایی است.
    \item مقایسه خود با دیگران: هر کس روش مطالعه و سرعت یادگیری خاص خود را دارد. روی برنامه خودتان تمرکز کنید.
    \item نادیده گرفتن نقاط ضعف: حتماً نقاط ضعف خود را شناسایی و برای رفع آن‌ها برنامه‌ریزی کنید. فرجه آخر فرصت خوبی برای این کار است.
\end{itemize}

\subsection*{نکات طلایی برای این درس}
\addcontentsline{toc}{subsection}{نکات طلایی - نمره کامل تجربی}
\begin{itemize}
    \item تسلط بر مفاهیم پایه: فیزیک درسی مفهومی است. قبل از حفظ کردن فرمول‌ها، سعی کنید مفاهیم را به خوبی درک کنید.
    \item یادداشت‌برداری و خلاصه‌نویسی: در حین مرور، نکات مهم و فرمول‌ها را یادداشت کنید تا در روزهای آخر به راحتی مرور کنید.
    \item توجه به یکاها: در حل مسائل، حتماً به یکاها توجه کنید و در صورت نیاز تبدیل یکا انجام دهید.
    \item تجسم فیزیکی مسائل: سعی کنید مسائل را در ذهن خود تجسم کنید. این کار به درک بهتر مسئله و یافتن راه‌حل کمک می‌کند.
    \item مرور مثال‌های حل شده کتاب: مثال‌های کتاب درسی بهترین نمونه برای یادگیری روش حل مسائل هستند.
    \item اهمیت به بخش "خوب است بدانید" و "فعالیت‌ها": گاهی اوقات نکات مهمی در این بخش‌ها وجود دارد که ممکن است مورد سوال قرار گیرند.
    \item تمرکز بر روی سوالات پرتکرار: با بررسی نمونه سوالات سال‌های گذشته، سوالات پرتکرار را شناسایی و روی آن‌ها تمرکز بیشتری داشته باشید.
    \item حفظ آرامش در جلسه امتحان: با آمادگی کامل و اعتماد به نفس در جلسه امتحان حاضر شوید و آرامش خود را حفظ کنید.
    \item بازخوانی سوالات قبل از پاسخ‌دهی: قبل از شروع به حل هر سوال، آن را به دقت بخوانید تا مطمئن شوید صورت سوال را به درستی متوجه شده‌اید.
\end{itemize}

\newpage
\section*{برنامه‌ریزی درسی برای نمره قابل قبول فیزیک ۳ تجربی}
\addcontentsline{toc}{section}{برنامه‌ریزی درسی برای نمره قابل قبول فیزیک ۳ تجربی}

\subsection*{پنجشنبه ۱ خرداد (عصر)}
\addcontentsline{toc}{subsection}{پنجشنبه ۱ خرداد (عصر) - نمره قابل قبول تجربی}
فرجه عصر (حدود 2/5 تا ۳ ساعت مطالعه مفید):

\unithead{واحد ۱ (۱ ساعت و ۳۰ دقیقه): فصل ۱ - حرکت بر خط راست}
\noindent\textbf{مرور: }مفاهیم اولیه (مکان، جابجایی، مسافت)، تندی و سرعت متوسط و لحظه‌ای، نمودارهای مکان-زمان و سرعت-زمان.
\par\noindent\textbf{حل مسئله: }حل مثال‌های کتاب درسی مربوط به این مباحث، و تعدادی از تمرینات ساده و متوسط انتهای فصل.

\休息{استراحت (۱۵ دقیقه)}

\unithead{واحد ۲ (۱ ساعت تا ۱ ساعت و ۱۵ دقیقه): فصل ۱ - حرکت بر خط راست (ادامه)}
\noindent\textbf{مرور: }حرکت با شتاب ثابت و معادلات آن.
\par\noindent\textbf{حل مسئله: }حل مثال‌های کتاب درسی و چند تمرین منتخب مربوط به حرکت با شتاب ثابت.

\subsection*{جمعه ۲ خرداد (صبح و عصر)}
\addcontentsline{toc}{subsection}{جمعه ۲ خرداد (صبح و عصر) - نمره قابل قبول تجربی}
فرجه صبح (حدود ۳ ساعت مطالعه مفید):

\unithead{واحد ۱ (۱ ساعت و ۳۰ دقیقه): فصل ۲ - دینامیک}
\noindent\textbf{مرور: }قوانین اول و دوم نیوتون، مفهوم نیرو، نیروی وزن و نیروی عمودی سطح.
\par\noindent\textbf{حل مسئله: }حل مسائل ساده و متوسط کتاب درسی با تمرکز بر کاربرد قانون دوم نیوتون (در یک راستا).

\休息{استراحت (۱۵ دقیقه)}

\unithead{واحد ۲ (۱ ساعت و ۳۰ دقیقه): فصل ۲ - دینامیک (ادامه)}
\noindent\textbf{مرور: }نیروی اصطکاک (ایستایی و جنبشی)، قانون سوم نیوتون.
\par\noindent\textbf{حل مسئله: }حل مسائل مربوط به نیروی اصطکاک و چند مثال از کاربرد قانون سوم.

\vspace{1em}
فرجه عصر (حدود ۳ ساعت مطالعه مفید):

\unithead{واحد ۳ (۱ ساعت و ۳۰ دقیقه): فصل ۳ - نوسان و امواج (بخش نوسان)}
\noindent\textbf{مرور: }حرکت هماهنگ ساده (معادله مکان-زمان، دوره، بسامد)، انرژی در حرکت هماهنگ ساده.
\par\noindent\textbf{حل مسئله: }حل مثال‌های کتاب و چند تمرین منتخب برای محاسبه پارامترهای نوسان و انرژی.

\休息{استراحت (۱۵ دقیقه)}

\unithead{واحد ۴ (۱ ساعت و ۳۰ دقیقه): فصل ۳ - نوسان و امواج (بخش موج)}
\noindent\textbf{مرور: }مشخصه‌های موج (طول موج، دامنه، سرعت انتشار)، رابطه سرعت، طول موج و بسامد.
\par\noindent\textbf{حل مسئله: }حل مسائل مربوط به محاسبه مشخصه‌های موج.

\subsection*{شنبه ۳ خرداد (صبح و عصر) - فرجه مرور و نمونه سوال}
\addcontentsline{toc}{subsection}{شنبه ۳ خرداد (صبح و عصر) - نمره قابل قبول تجربی}
فرجه صبح (حدود ۳ ساعت مطالعه مفید):

\unithead{واحد ۱ (۱ ساعت و ۳۰ دقیقه): فصل ۴ - آشنایی با فیزیک اتمی و هسته‌ای}
\noindent\textbf{مرور: }اثر فوتوالکتریک و فوتون، مدل اتم رادرفورد-بور (اصول اولیه)، انواع واپاشی و مفهوم نیمه‌عمر.
\par\noindent\textbf{حل مسئله: }حل چند مثال ساده از معادله فوتوالکتریک و مسائل مربوط به نیمه‌عمر.

\休息{استراحت (۱۵ دقیقه)}

\unithead{واحد ۲ (۱ ساعت و ۳۰ دقیقه): حل نمونه سوال منتخب (بخش‌هایی از فصول ۱ و ۲)}
تعدادی از سوالات منتخب امتحانات نهایی سال‌های گذشته مربوط به فصل ۱ و ۲ را حل کنید.
\par روی سوالاتی تمرکز کنید که در آن‌ها احساس ضعف بیشتری دارید.

\vspace{1em}
فرجه عصر (حدود 2/5 تا ۳ ساعت مطالعه مفید):

\unithead{واحد ۳ (۱ ساعت و ۱۵ دقیقه تا ۱ ساعت و ۳۰ دقیقه): حل نمونه سوال منتخب (بخش‌هایی از فصول ۳ و ۴)}
تعدادی از سوالات منتخب امتحانات نهایی سال‌های گذشته مربوط به فصل ۳ و ۴ را حل کنید.

\休息{استراحت (۱۵ دقیقه)}

\unithead{واحد ۴ (۱ ساعت تا ۱ ساعت و ۱۵ دقیقه): مرور نهایی و جمع‌بندی}
\noindent\textbf{مرور سریع فرمول‌ها و تعاریف مهم: }یک بار دیگر فرمول‌ها و تعاریف کلیدی تمام فصول را مرور کنید.
\par\noindent\textbf{بررسی اشکالات: }به سوالاتی که در طول حل نمونه سوالات مشکل داشته‌اید، دوباره نگاه کنید و سعی کنید راه‌حل آن‌ها را بفهمید.
\par\noindent\textbf{تمرکز بر نقاط قوت: }مباحثی را که در آن‌ها قوی‌تر هستید، یک بار دیگر مرور کنید تا در پاسخ به سوالات آن‌ها مطمئن باشید.

\subsection*{اولویت‌بندی برای دانش‌آموز متوسط}
\addcontentsline{toc}{subsection}{اولویت‌بندی - نمره قابل قبول تجربی}
\begin{itemize}
    \item تسلط بر مفاهیم و مسائل پایه کتاب درسی: این مهم‌ترین بخش است. مطمئن شوید که تمام مثال‌ها و تمرینات اصلی کتاب را به خوبی بلد هستید.
    \item فصول با بارم بالاتر: معمولاً فصل‌های حرکت و دینامیک بارم بیشتری دارند. سعی کنید در این فصول به تسلط خوبی برسید.
    \item مباحثی که در آن‌ها قوی‌تر هستید: ابتدا روی این مباحث کار کنید تا اعتماد به نفس شما افزایش یابد و نمره تضمین شده‌ای داشته باشید.
    \item مباحثی که کمی در آن‌ها ضعف دارید: پس از تسلط بر نقاط قوت، به سراغ مباحثی بروید که کمی در آن‌ها مشکل دارید و سعی کنید با حل تمرین آن‌ها را بهبود ببخشید.
\end{itemize}

\subsection*{اشتباهات مهلک برای دانش‌آموز متوسط}
\addcontentsline{toc}{subsection}{اشتباهات مهلک - نمره قابل قبول تجربی}
\begin{itemize}
    \item نادیده گرفتن کتاب درسی: کتاب درسی بهترین منبع برای یادگیری مفاهیم و حل مسائل پایه است.
    \item حفظ کردن فرمول‌ها بدون درک مفهومی: این کار باعث می‌شود در مواجهه با سوالات کمی متفاوت‌تر، دچار مشکل شوید.
    \item صرف زمان بیش از حد روی یک مبحث دشوار: اگر مبحثی برای شما خیلی سخت است و زمان زیادی از شما می‌گیرد، بهتر است ابتدا روی مباحثی که راحت‌تر یاد می‌گیرید تمرکز کنید.
    \item عدم حل نمونه سوالات نهایی: آشنایی با سبک سوالات امتحان نهایی بسیار مهم است.
    \item استرس و نگرانی بیش از حد: با برنامه‌ریزی و مطالعه منظم، می‌توانید نمره خوبی کسب کنید. به خودتان اعتماد داشته باشید.
\end{itemize}

\subsection*{نکات طلایی برای دانش‌آموز متوسط}
\addcontentsline{toc}{subsection}{نکات طلایی - نمره قابل قبول تجربی}
\begin{itemize}
    \item تمرکز بر یادگیری مفاهیم کلیدی: سعی کنید مفاهیم اصلی هر فصل را به خوبی درک کنید.
    \item حل مثال‌های کتاب درسی با دقت: مثال‌های کتاب بهترین راهنما برای یادگیری روش حل مسائل هستند.
    \item تمرین مرحله به مرحله: ابتدا مسائل ساده‌تر را حل کنید و سپس به سراغ مسائل کمی پیچیده‌تر بروید.
    \item استفاده از خلاصه‌نویسی: نکات مهم و فرمول‌ها را برای مرور سریع‌تر یادداشت کنید.
    \item پرسیدن سوال: اگر در مبحثی مشکل دارید، از معلم یا دوستان خود کمک بگیرید.
    \item مرور منظم: مطالب خوانده شده را در فواصل زمانی منظم مرور کنید.
\end{itemize}

\newpage
\section*{برنامه‌ریزی درسی برای نمره قبولی فیزیک ۳ تجربی}
\addcontentsline{toc}{section}{برنامه‌ریزی درسی برای نمره قبولی فیزیک ۳ تجربی}

\subsection*{پنجشنبه ۱ خرداد (عصر)}
\addcontentsline{toc}{subsection}{پنجشنبه ۱ خرداد (عصر) - نمره قبولی تجربی}
فرجه عصر (حدود ۲ تا 2/5 ساعت مطالعه مفید):

\unithead{واحد ۱ (۱ ساعت تا ۱ ساعت و ۱۵ دقیقه): فصل ۱ - حرکت بر خط راست (مفاهیم اولیه)}
\noindent\textbf{مرور: }تعاریف مکان، جابجایی، مسافت، تندی متوسط و سرعت متوسط. نحوه خواندن نمودار مکان-زمان.
\par\noindent\textbf{حل مسئله: }حل چند مثال بسیار ساده از کتاب درسی برای محاسبه مسافت، جابجایی، تندی و سرعت متوسط.

\休息{استراحت (۱۵ دقیقه)}

\unithead{واحد ۲ (۱ ساعت): فصل ۲ - دینامیک (قوانین نیوتون و وزن)}
\noindent\textbf{مرور: }بیان ساده قوانین اول، دوم و سوم نیوتون. مفهوم نیروی وزن و رابطه آن با جرم ($W=mg$).
\par\noindent\textbf{حل مسئله: }حل چند مثال ساده برای محاسبه نیروی وزن.

\subsection*{جمعه ۲ خرداد (صبح و عصر)}
\addcontentsline{toc}{subsection}{جمعه ۲ خرداد (صبح و عصر) - نمره قبولی تجربی}
فرجه صبح (حدود 2/5 ساعت مطالعه مفید):

\unithead{واحد ۱ (۱ ساعت و ۱۵ دقیقه): فصل ۱ - حرکت بر خط راست (حرکت با سرعت ثابت)}
\noindent\textbf{مرور: }مفهوم حرکت با سرعت ثابت و نمودارهای آن.
\par\noindent\textbf{حل مسئله: }حل چند تمرین ساده از کتاب درسی مربوط به حرکت با سرعت ثابت.

\休息{استراحت (۱۵ دقیقه)}

\unithead{واحد ۲ (۱ ساعت و ۱۵ دقیقه): فصل ۲ - دینامیک (نیروی عمودی سطح و اصطکاک ساده)}
\noindent\textbf{مرور: }مفهوم نیروی عمودی سطح در سطوح افقی. مفهوم کلی نیروی اصطکاک (بدون ورود به جزئیات پیچیده).
\par\noindent\textbf{حل مسئله: }شناسایی نیروی عمودی سطح در مثال‌های ساده.

\vspace{1em}
فرجه عصر (حدود 2/5 ساعت مطالعه مفید):

\unithead{واحد ۳ (۱ ساعت و ۱۵ دقیقه): فصل ۳ - نوسان و امواج (مفاهیم اولیه موج)}
\noindent\textbf{مرور: }تعریف موج، انواع موج (طولی و عرضی به صورت کلی)، مشخصه‌های موج (طول موج و دامنه به صورت تصویری).
\par\noindent\textbf{حل مسئله: }شناسایی طول موج و دامنه از روی شکل‌های ساده کتاب درسی.

\休息{استراحت (۱۵ دقیقه)}

\unithead{واحد ۴ (۱ ساعت و ۱۵ دقیقه): فصل ۴ - آشنایی با فیزیک اتمی و هسته‌ای (مفاهیم اولیه اتمی و پرتوزایی)}
\noindent\textbf{مرور: }تعریف فوتون و اثر فوتوالکتریک (به صورت کلی). انواع اصلی واپاشی (آلفا، بتا، گاما) و نمادهای آن‌ها.
\par\noindent\textbf{حل مسئله: }شناسایی نوع واپاشی از روی معادله ساده.

\subsection*{شنبه ۳ خرداد (صبح و عصر) - فرجه مرور و نکات کلیدی}
\addcontentsline{toc}{subsection}{شنبه ۳ خرداد (صبح و عصر) - نمره قبولی تجربی}
فرجه صبح (حدود 2/5 ساعت مطالعه مفید):

\unithead{واحد ۱ (۱ ساعت و ۱۵ دقیقه): مرور فصل ۱ و ۲ (نکات بسیار مهم)}
مرور سریع تعاریف و فرمول‌های بسیار اساسی فصل ۱ و ۲.
\par حل چند تمرین بسیار ساده و پرتکرار از این دو فصل از نمونه سوالات سال‌های گذشته (در صورت امکان).

\休息{استراحت (۱۵ دقیقه)}

\unithead{واحد ۲ (۱ ساعت و ۱۵ دقیقه): مرور فصل ۳ و ۴ (نکات بسیار مهم)}
مرور سریع تعاریف و نکات کلیدی فصل ۳ و ۴.
\par حل چند سوال جای خالی یا صحیح/غلط از این دو فصل.

\vspace{1em}
فرجه عصر (حدود ۲ تا 2/5 ساعت مطالعه مفید):

\unithead{واحد ۳ (۱ ساعت): حل چند سوال بسیار ساده از نمونه سوالات نهایی}
تمرکز بر سوالات تعریفی، جای خالی و مسائل تک مرحله‌ای بسیار ساده از نمونه سوالات سال‌های گذشته.

\休息{استراحت (۱۵ دقیقه)}

\unithead{واحد ۴ (۱ ساعت تا ۱ ساعت و ۱۵ دقیقه): مرور نهایی و جمع‌بندی نکات طلایی برای قبولی}
یک بار دیگر تمام تعاریف و فرمول‌های اصلی که یادداشت کرده‌اید را مرور کنید.
\par به خودتان برای یادآوری مطالب ساده و کسب نمره قبولی انگیزه بدهید.

\subsection*{اولویت‌بندی برای نمره قبولی}
\addcontentsline{toc}{subsection}{اولویت‌بندی - نمره قبولی تجربی}
\begin{itemize}
    \item تسلط بر تعاریف و مفاهیم بسیار پایه: مطمئن شوید تعاریف کلیدی هر فصل را بلد هستید.
    \item یادگیری فرمول‌های اصلی و کاربرد ساده آن‌ها: فرمول‌های اساسی مانند سرعت متوسط، وزن، و رابطه اصلی موج را یاد بگیرید و بتوانید در مسائل تک مرحله‌ای از آن‌ها استفاده کنید.
    \item تمرکز بر سوالات مستقیم و بدون پیچیدگی: سوالاتی که مستقیماً از متن کتاب یا تمرینات ساده آن طرح می‌شوند، هدف اصلی شما هستند.
    \item فصولی که درک آن‌ها برایتان ساده‌تر است: روی این فصول تمرکز بیشتری داشته باشید.
\end{itemize}

\subsection*{اشتباهات مهلک برای دانش‌آموزی که هدفش قبولی است}
\addcontentsline{toc}{subsection}{اشتباهات مهلک - نمره قبولی تجربی}
\begin{itemize}
    \item صرف زمان روی مباحث بسیار دشوار و کم‌اهمیت: وقت خود را برای یادگیری مباحثی که پیچیده هستند و بارم کمی دارند، تلف نکنید.
    \item حفظ کردن بدون فهم: حتی برای نمره قبولی هم نیاز به حداقل درک از مفاهیم دارید.
    \item ناامیدی و استرس: باور داشته باشید که با تلاش و تمرکز بر مباحث اصلی، می‌توانید نمره قبولی را کسب کنید.
    \item رها کردن کامل برخی فصول: سعی کنید از هر فصل حداقل مفاهیم بسیار پایه را یاد بگیرید.
\end{itemize}

\subsection*{نکات طلایی برای کسب نمره قبولی}
\addcontentsline{toc}{subsection}{نکات طلایی - نمره قبولی تجربی}
\begin{itemize}
    \item کتاب درسی، بهترین دوست شما: تمام تمرکز خود را روی یادگیری مفاهیم و مثال‌های ساده کتاب درسی بگذارید.
    \item فلش کارت برای تعاریف و فرمول‌ها: تعاریف و فرمول‌های مهم را روی فلش کارت بنویسید و مدام مرور کنید.
    \item حل تمرینات "در حد فهم": تمریناتی را حل کنید که مفهوم آن‌ها را متوجه می‌شوید و راه‌حل آن‌ها برایتان قابل درک است.
    \item مثبت‌اندیشی و اعتماد به نفس: با باور به توانایی خودتان و تمرکز بر یادگیری، می‌توانید موفق شوید.
    \item درخواست کمک: اگر در مبحثی ساده مشکل دارید، از معلم یا دوستانتان کمک بگیرید.
\end{itemize}

\end{document}