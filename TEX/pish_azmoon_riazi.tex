\documentclass[12pt,a4paper]{article}
\usepackage{amsmath}
\usepackage{amssymb}
\usepackage{amsfonts}
\usepackage[right=2cm, left=2cm, top=2.5cm, bottom=2.5cm]{geometry} % Adjust margins as needed
\usepackage{enumitem} % For custom list labels

% Load fontspec before xepersian
\usepackage{fontspec}
\setmainfont{Amiri} % Use Amiri font for the main text

% Load xepersian last
\usepackage{xepersian}
\settextfont{Amiri} % Use Amiri font for Persian text

\title{
    \vspace{-2em} % Adjust vertical space if needed
    \textbf{پیش‌آزمون تشخیصی ریاضی ۱ پایه دهم (مشابه امتحان نهایی)}
}
\author{} % No author needed for a test
\date{} % No date needed, or you can put custom info here

\begin{document}

\maketitle
\thispagestyle{empty} % No page number on the first page

\begin{center}
    \textbf{زمان پاسخگویی:} ۱۲۰ دقیقه \quad \textbf{بارم کل:} ۲۰ نمره
\end{center}

\noindent
این پیش‌آزمون سعی دارد با پوشش مباحث کلیدی و سوالاتی که معمولاً چالش‌برانگیزتر هستند یا نیاز به درک عمیق‌تری دارند، به دانش‌آموز در شناسایی دقیق‌تر مشکلاتش کمک کند.

\vspace{0.5em}
\noindent
\textbf{توجه:} هدف این آزمون، شناسایی نقاط ضعف است. پس با دقت و بدون نگرانی از نمره، به سوالات پاسخ دهید.

\hrulefill
\vspace{1em}

\section*{\textbf{فصل ۱: مجموعه، الگو و دنباله (حدود 2/5 نمره)}}

\begin{enumerate}[label=\arabic*., rightmargin=1em, itemsep=1em]
    \item الف) مجموعه $A = \{x \in \mathbb{R} \mid -2 \le x < 3\}$ و $B = \{x \in \mathbb{R} \mid x \ge 1\}$ را در نظر بگیرید. مجموعه $A \cap B$ و $A \cup B'$ را با استفاده از بازه‌ها نمایش دهید و روی محور اعداد رسم کنید. (۱ نمره)
    \begin{itemize}[label=$\circ$, rightmargin=2em]
        \item \textit{نقطه ضعف احتمالی: درک عملیات روی بازه‌ها، مفهوم متمم، نمایش روی محور.}
    \end{itemize}

    \item در یک دنباله حسابی، جمله چهارم برابر ۱۱ و جمله هفتم برابر ۲۰ است.
    \begin{enumerate}[label=\abjad*), itemsep=0.5em]
        \item قدر نسبت و جمله اول این دنباله را بیابید.
        \item جمله عمومی این دنباله چیست؟
        \item مجموع ۱۰ جمله اول این دنباله را محاسبه کنید. (1/5 نمره)
    \end{enumerate}
    \begin{itemize}[label=$\circ$, rightmargin=2em]
        \item \textit{نقطه ضعف احتمالی: استفاده از فرمول‌های دنباله حسابی، تشکیل دستگاه معادلات، محاسبه مجموع.}
    \end{itemize}
\end{enumerate}

\hrulefill
\vspace{1em}

\section*{\textbf{فصل ۲: مثلثات (حدود 3/5 نمره)}}

\begin{enumerate}[label=\arabic*., start=3, rightmargin=1em, itemsep=1em]
    \item اگر $\sin \alpha = \frac{3}{5}$ و انتهای کمان $\alpha$ در ربع دوم باشد، سایر نسبت‌های مثلثاتی زاویه $\alpha$ را بیابید. (1/5 نمره)
    \begin{itemize}[label=$\circ$, rightmargin=2em]
        \item \textit{نقطه ضعف احتمالی: استفاده از اتحاد $\sin^2 \alpha + \cos^2 \alpha = 1$, توجه به علامت نسبت‌ها در نواحی مختلف، محاسبه تانژانت و کتانژانت.}
    \end{itemize}

    \item معادله خطی را بنویسید که از نقطه $A(2, -1)$ گذشته و با جهت مثبت محور x ها زاویه $135^\circ$ می‌سازد. (۱ نمره)
    \begin{itemize}[label=$\circ$, rightmargin=2em]
        \item \textit{نقطه ضعف احتمالی: ارتباط شیب خط با تانژانت زاویه، استفاده از فرمول معادله خط.}
    \end{itemize}

    \item درستی یا نادرستی عبارت زیر را با ذکر دلیل بررسی کنید: $\tan 75^\circ > \sin 150^\circ$ (۱ نمره)
    \begin{itemize}[label=$\circ$, rightmargin=2em]
        \item \textit{نقطه ضعف احتمالی: مقایسه مقادیر مثلثاتی، استفاده از دایره مثلثاتی، تبدیل زوایا به ربع اول.}
    \end{itemize}
\end{enumerate}

\hrulefill
\vspace{1em}

\section*{\textbf{فصل ۳: توان‌های گویا و عبارت‌های جبری (حدود ۳ نمره)}}

\begin{enumerate}[label=\arabic*., start=6, rightmargin=1em, itemsep=1em]
    \item عبارت زیر را ساده کنید: (1/5 نمره)
    \[
    (\sqrt[3]{a^2b} \times \sqrt{ab^3}) \div \sqrt[6]{a^5b^7}
    \]
    (فرض کنید همه عبارت‌ها تعریف شده باشند)
    \begin{itemize}[label=$\circ$, rightmargin=2em]
        \item \textit{نقطه ضعف احتمالی: تبدیل رادیکال به توان گویا، استفاده از قوانین توان، ساده‌سازی.}
    \end{itemize}

    \item الف) عبارت $8x^3 - 27y^3$ را تجزیه کنید. \\
    ب) مخرج کسر $\frac{x}{\sqrt{x} - 2}$ را گویا کنید. (1/5 نمره)
    \begin{itemize}[label=$\circ$, rightmargin=2em]
        \item \textit{نقطه ضعف احتمالی: استفاده از اتحاد مکعب دوجمله‌ای، روش گویا کردن مخرج (ضرب در مزدوج).}
    \end{itemize}
\end{enumerate}

\hrulefill
\vspace{1em}

\section*{\textbf{فصل ۴: معادله‌ها و نامعادله‌ها (حدود 3/5 نمره)}}
\begin{enumerate}[label=\arabic*., start=8, rightmargin=1em, itemsep=1em]
    \item معادله $2x^2 - 5x + 2 = 0$ را به روش‌های زیر حل کنید:
    \begin{enumerate}[label=\abjad*), itemsep=0.5em]
        \item تجزیه
        \item فرمول کلی (دلتا) (1/5 نمره)
    \end{enumerate}
    \begin{itemize}[label=$\circ$, rightmargin=2em]
        \item \textit{نقطه ضعف احتمالی: تسلط بر روش‌های مختلف حل معادله درجه دوم.}
    \end{itemize}

    \item سهمی $y = -x^2 + 4x - 3$ داده شده است.
    \begin{enumerate}[label=\abjad*), itemsep=0.5em]
        \item مختصات رأس و معادله خط تقارن آن را بیابید.
        \item نمودار سهمی را رسم کرده و محل برخورد آن با محورهای مختصات را مشخص کنید. (1/25 نمره)
    \end{enumerate}
    \begin{itemize}[label=$\circ$, rightmargin=2em]
        \item \textit{نقطه ضعف احتمالی: پیدا کردن رأس و خط تقارن، رسم دقیق سهمی، پیدا کردن نقاط برخورد.}
    \end{itemize}

    \item مجموعه جواب نامعادله $\frac{x-1}{x+2} \le 0$ را با استفاده از جدول تعیین علامت بیابید. (0/75 نمره)
    \begin{itemize}[label=$\circ$, rightmargin=2em]
        \item \textit{نقطه ضعف احتمالی: تعیین علامت عبارت‌های کسری، توجه به ریشه‌های مخرج.}
    \end{itemize}
\end{enumerate}

\hrulefill
\vspace{1em}

\section*{\textbf{فصل ۵: تابع (حدود 4/5 نمره)}}

\begin{enumerate}[label=\arabic*., start=11, rightmargin=1em, itemsep=1em]
    \item کدام یک از روابط زیر تابع است؟ در مورد توابع، دامنه و برد را مشخص کنید. (1/5 نمره)
    \begin{enumerate}[label=\abjad*)]
        \item $f = \{(1,2), (2,3), (1,4), (3,5)\}$
        \item $g(x) = \sqrt{x-2}$
        \item نموداری که از نقاط $(0,1), (1,2), (2,1), (1,0)$ می‌گذرد.
    \end{enumerate}
    \begin{itemize}[label=$\circ$, rightmargin=2em]
        \item \textit{نقطه ضعف احتمالی: تعریف تابع، تشخیص تابع از روی زوج مرتب، ضابطه و نمودار، تعیین دامنه و برد.}
    \end{itemize}

    \item اگر $f(x) = 2x-1$ و $g(x) = x^2 + 1$ باشند، موارد زیر را محاسبه کنید:
    \begin{enumerate}[label=\abjad*), itemsep=0.5em]
        \item $f(g(0))$
        \item ضابطه تابع $f(x) + g(x-1)$ (1/25 نمره)
    \end{enumerate}
    \begin{itemize}[label=$\circ$, rightmargin=2em]
        \item \textit{نقطه ضعف احتمالی: محاسبه مقدار تابع ترکیبی، عملیات روی توابع و تغییر متغیر.}
    \end{itemize}

    \item نمودار تابع $y = |x-1| + 2$ را با استفاده از انتقال رسم کنید. (۱ نمره)
    \begin{itemize}[label=$\circ$, rightmargin=2em]
        \item \textit{نقطه ضعف احتمالی: رسم تابع قدرمطلق، اعمال انتقال‌های افقی و عمودی.}
    \end{itemize}

    \item ضابطه یک تابع خطی را بنویسید که از نقاط $(1,3)$ و $(-1, -1)$ می‌گذرد. (0/75 نمره)
    \begin{itemize}[label=$\circ$, rightmargin=2em]
        \item \textit{نقطه ضعف احتمالی: پیدا کردن معادله خط با داشتن دو نقطه، مفهوم تابع خطی.}
    \end{itemize}
\end{enumerate}

\hrulefill
\vspace{1em}

\section*{\textbf{فصل ۶: شمارش، بدون شمردن (حدود 2/5 نمره)}}

\begin{enumerate}[label=\arabic*., start=15, rightmargin=1em, itemsep=1em]
    \item با ارقام $0, 1, 2, 3, 4, 5$ چند عدد سه رقمی \textbf{بدون تکرار ارقام} می‌توان نوشت به طوری که:
    \begin{enumerate}[label=\abjad*), itemsep=0.5em]
        \item زوج باشد؟
        \item از ۳۰۰ بزرگتر باشد؟ (1/5 نمره)
    \end{enumerate}
    \begin{itemize}[label=$\circ$, rightmargin=2em]
        \item \textit{نقطه ضعف احتمالی: اصل ضرب، توجه به محدودیت‌ها (عدم تکرار، رقم اول صفر نباشد، شرایط زوج بودن یا بزرگتر بودن).}
    \end{itemize}

    \item از بین ۵ ریاضیدان و ۴ فیزیکدان، به چند طریق می‌توان یک کمیته ۳ نفره انتخاب کرد به طوری که:
    \begin{enumerate}[label=\abjad*), itemsep=0.5em]
        \item دقیقاً ۲ نفر ریاضیدان باشند؟
        \item حداقل ۱ نفر فیزیکدان باشد؟ (۱ نمره)
    \end{enumerate}
    \begin{itemize}[label=$\circ$, rightmargin=2em]
        \item \textit{نقطه ضعف احتمالی: تشخیص بین جایگشت و ترکیب، استفاده از فرمول ترکیب، اصل متمم.}
    \end{itemize}
\end{enumerate}

\hrulefill
\vspace{1em}

\section*{\textbf{فصل ۷: آمار و احتمال (حدود 2/5 نمره)}}

\begin{enumerate}[label=\arabic*., start=17, rightmargin=1em, itemsep=1em]
    \item دو تاس را همزمان پرتاب می‌کنیم. مطلوب است احتمال آنکه:
    \begin{enumerate}[label=\abjad*), itemsep=0.5em]
        \item مجموع اعداد رو شده برابر ۷ باشد.
        \item اعداد رو شده هر دو اول باشند.
        \item حداقل یکی از اعداد رو شده مضرب ۳ باشد. (1/5 نمره)
    \end{enumerate}
    \begin{itemize}[label=$\circ$, rightmargin=2em]
        \item \textit{نقطه ضعف احتمالی: تشکیل فضای نمونه‌ای، شناسایی پیشامدهای مطلوب، محاسبه احتمال، استفاده از اعمال روی پیشامدها.}
    \end{itemize}

    \item در یک کیسه ۳ مهره قرمز، ۴ مهره آبی و ۲ مهره سبز وجود دارد. دو مهره به تصادف و \textbf{بدون جایگذاری} از کیسه خارج می‌کنیم. احتمال آنکه هر دو مهره همرنگ باشند چقدر است؟ (۱ نمره)
    \begin{itemize}[label=$\circ$, rightmargin=2em]
        \item \textit{نقطه ضعف احتمالی: احتمال شرطی، انتخاب بدون جایگذاری، استفاده از اصل ضرب یا ترکیب در احتمال.}
    \end{itemize}
\end{enumerate}

\hrulefill
\vspace{1em}

\section*{\textbf{پس از آزمون:}}

\noindent
دانش‌آموز عزیز، پس از اتمام این پیش‌آزمون، پاسخ‌های خود را با یک پاسخنامه معتبر (مثلاً پاسخ‌های معلم یا حل تشریحی کتاب کمک‌آموزشی) مقایسه کن.

\begin{itemize}[rightmargin=1em, itemsep=0.5em]
    \item \textbf{سوالاتی که کاملاً درست حل کرده‌ای:} نشان‌دهنده نقاط قوت شماست.
    \item \textbf{سوالاتی که با اشتباهات جزئی (مثلاً محاسباتی) حل کرده‌ای:} نیاز به دقت بیشتر و تمرین در آن مباحث دارید.
    \item \textbf{سوالاتی که اصلاً نتوانسته‌ای حل کنی یا راه‌حل اشتباهی ارائه داده‌ای:} این‌ها \textbf{مهم‌ترین نقاط ضعف} شما هستند. دقیقاً مشخص کن که مشکل از کدام بخش مفهوم، فرمول یا روش حل بوده است.
    \item \textbf{فصل‌هایی که بیشترین مشکل را داشته‌ای:} این فصول نیاز به مطالعه و تمرین مجدد و عمیق‌تری دارند.
    \item \textbf{نوع سوالاتی که در آن‌ها مشکل داری (مفهومی، محاسباتی، کاربردی):} این به شما کمک می‌کند تا روش مطالعه خود را اصلاح کنی.
\end{itemize}

\vspace{1em}
\noindent
\textbf{تحلیل نتایج برای روشن شدن مسیر:}

\begin{itemize}[rightmargin=1em, itemsep=0.5em]
    \item \textbf{مباحث پرتکرار و مهم:} معمولاً سوالات از بخش‌های اصلی هر فصل مانند حل معادله درجه دوم، دایره مثلثاتی، مفهوم تابع و اصول شمارش حتماً در امتحان خواهد بود. این پیش‌آزمون سعی کرده این مباحث را پوشش دهد.
    \item \textbf{ارتباط بین فصول:} برخی سوالات ممکن است نیازمند استفاده از مفاهیم چند فصل باشند (مثلاً ترکیب تابع با مثلثات یا احتمال با شمارش). اگر در این سوالات مشکل دارید، باید روی پیوند بین مفاهیم کار کنید.
    \item \textbf{مهارت‌های پایه:} اشتباهات محاسباتی، ضعف در ساده‌سازی عبارت‌ها یا عدم تسلط بر مفاهیم پایه‌ای سال‌های قبل می‌تواند مشکل‌ساز باشد.
\end{itemize}

\vspace{1em}
\noindent
با استفاده از نتایج این پیش‌آزمون، یک برنامه مطالعاتی هدفمند برای رفع نقاط ضعف خود تهیه کن. روی مباحثی که مشکل بیشتری داری، زمان بیشتری بگذار و تمرینات متنوع‌تری حل کن.

\vspace{1em}
\begin{center}
    موفق باشی!
\end{center}

\end{document}