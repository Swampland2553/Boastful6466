\documentclass{article}

% General packages
\usepackage{fontspec} % For \settextfont
\usepackage{array}    % For m{} column type in tables
\usepackage{geometry} % For adjusting page margins
\geometry{a4paper, margin=1in} % Sensible default margins

% TikZ package and libraries
\usepackage{tikz}
\usetikzlibrary{
    shapes.geometric, % For shapes like rectangle
    arrows.meta,      % For arrow tips like Stealth
    positioning,      % For relative positioning (right=of, etc.)
    fit,              % For drawing nodes that fit around other nodes (subgraphs)
    calc              % For coordinate calculations if needed
}

% Xepersian - load after other general packages
\usepackage{xepersian}
\settextfont{Amiri} % Set Amiri as the main font
                    % Note: Amiri might not support all emojis.
                    % You might need a font like "XB Niloofar", "Vazirmatn",
                    % or a comprehensive Unicode font for full emoji support.

% Define colors (from your TikZ code)
\definecolor{merFillAF}{HTML}{FF99FF} % #f9f
\definecolor{merFillCD}{HTML}{CCCCFF} % #ccf
\definecolor{merFillEF}{HTML}{99CCFF} % #9cf
\definecolor{merStroke}{HTML}{333333} % #333

% Define TikZ styles (from your TikZ code)
\tikzset{
    % Base style for all flowchart nodes
    base_node/.style={
        rectangle,
        draw=merStroke,
        line width=1.5pt, % Approximates stroke-width:2px
        text centered,
        minimum height=3.2cm, % Further increased height for more wrapped lines
        text width=2.7cm,   % Further reduced text width
        font=\footnotesize % Changed to smaller font
    },
    % Specific styles from Mermaid
    nodeA/.style={base_node, fill=merFillAF},
    nodeB/.style={base_node, fill=merFillAF},
    nodeC/.style={base_node, fill=merFillCD},
    nodeD/.style={base_node, fill=merFillCD},
    nodeE/.style={base_node, fill=merFillEF},
    nodeF/.style={base_node, fill=merFillEF},
    % Arrow style
    arrow/.style={
        -Stealth, % Arrow tip
        thick,
        draw=merStroke
    },
    % Subgraph box style
    subgraph_box/.style={
        draw=black!60,
        line width=1pt,
        rounded corners,
        inner sep=0.15cm, % Further reduced padding inside the subgraph box
        label={[font=\footnotesize\bfseries, anchor=south, yshift=0.1em]above:#1} % Changed label font size
    }
}

\begin{document}

\begin{center}
\begin{tabular}{|c|m{0.15\textwidth}|m{0.15\textwidth}|m{0.20\textwidth}|m{0.2\textwidth}|m{0.2\textwidth}|}
\hline
\textbf{گام} & \textbf{زمان برنامه‌ریزی} & \textbf{نحوه گزارش} & \textbf{متغیر کلیدی (تغییر اصلی)} & \textbf{هدف اصلی مرحله} & \textbf{پیش‌نیاز/مهارت کسب‌شده} \\
\hline
 \textit{۱} & گزارش نویسی (پس از انجام) & یک کاغذ معمولی &  \textit{آغاز گزارش‌نویسی} (فعالیت دلخواه) & ایجاد عادت پایش فعالیت‌ها & - \\
\hline
 \textit{۲} & از شب قبل & برگ برنامه ریزی سطح ۲ &  \textit{قالب‌بندی} (مکتوب کردن برنامه) & ایجاد نظم و ساختار اولیه & عادت به گزارش نویسی \\
\hline
 \textit{۳} & از شب قبل & برگ برنامه ریزی سطح ۳ &  \textit{اولویت‌بندی و لیست کارها} & آموزش تعادل و انتخاب هوشمندانه & برنامه‌ریزی ساختار یافته \\
\hline
 \textit{۴} & سه روزه & برگه برنامه ریزی سطح ۴ &  \textit{درج ساعت و زمان‌بندی دقیق‌تر} & آغاز برنامه‌ریزی میان‌مدت & اولویت‌بندی و لیست کارها \\
\hline
️ \textit{۵} & یک هفته & برگه برنامه ریزی سطح ۵ &  \textit{توجه به محدودیت زمان} & ادامه برنامه‌ریزی میان‌مدت & زمان‌بندی دقیق \\
\hline
 \textit{۶} & دو هفته & برگه برنامه ریزی سطح ۶ &  \textit{توجه به پروژه‌ها و مباحث} & برنامه‌ریزی بلندمدت‌تر، چینش مباحث & مدیریت زمان در بازه طولانی \\
\hline
\end{tabular}
\end{center}

\vspace{2ex} % Add some vertical space after the table

\noindent\textbf{چرا این جدول مؤثرتر است؟}

\begin{itemize}
    \item \textit{آیکون‌ها:} به سرعت مفهوم گام یا متغیر کلیدی را منتقل می‌کنند.
    \item \textit{تأکید بر متغیر کلیدی:} ستون "متغیر کلیدی" به وضوح نشان می‌دهد که در هر مرحله چه مهارت یا مفهوم جدیدی اضافه می‌شود.
    \item \textit{هدف اصلی مرحله:} به طور خلاصه بیان می‌کند که دانش‌آموز در هر گام چه چیزی را یاد می‌گیرد.
    \item \textit{پیش‌نیاز/مهارت کسب‌شده:} ارتباط بین مراحل را نشان می‌دهد.
\end{itemize}

\vspace{4ex} % Space after the "چرا این جدول مؤثرتر است؟" itemize

\noindent\textbf{گزینه ۲: دیاگرام روند یا فلوچارت (Roadmap/Flowchart)}

\vspace{1ex} % Little space after this new title
این روش برای نمایش پیشرفت مرحله به مرحله بسیار عالی است.
\vspace{2ex} % Space before the flowchart figure

\begin{figure}[htbp]
\centering
\makebox[\textwidth][c]{%
\begin{tikzpicture}[
    node distance = 0.7cm and 0.25cm, % Further reduced horizontal distance
    scale=0.85, % Scaling to fit
    transform shape, % Ensures text and node contents also scale
    % every node/.style={on grid} % Optional: for easier alignment if needed
  ]

    % Define the nodes
    \node (A) [nodeA] {گام ۱: گزارش‌نویسی پایه \\(فعالیت دلخواه، کاغذ معمولی)};
    \node (B) [nodeB, right=of A] {گام ۲: ساختاردهی برنامه \\(قالب مکتوب، از شب قبل)};
    \node (C) [nodeC, right=of B] {گام ۳: سازماندهی و اولویت‌بندی \\(لیست کارها، تعادل مطالعاتی)};
    \node (D) [nodeD, right=of C] {گام ۴: مدیریت زمان اولیه \\(درج ساعت، برنامه‌ریزی سه‌روزه)};
    \node (E) [nodeE, right=of D] {گام ۵: برنامه‌ریزی هفتگی ️\\(توجه به محدودیت زمان)};
    \node (F) [nodeF, right=of E] {گام ۶: برنامه‌ریزی جامع دوهفته‌ای \\(توجه به پروژه‌ها و مباحث)};

    % Draw the arrows
    \draw [arrow] (A) -- (B);
    \draw [arrow] (B) -- (C);
    \draw [arrow] (C) -- (D);
    \draw [arrow] (D) -- (E);
    \draw [arrow] (E) -- (F);

    % Define the subgraphs using the 'fit' library
    % These are drawn on a background layer by default, so nodes appear on top.
    % We draw them after the main nodes and arrows.
    \node[subgraph_box="مهارت‌های پایه‌ای", fit=(A) (B)] {};
    \node[subgraph_box="مهارت‌های پیشرفته", fit=(C) (D)] {};
    \node[subgraph_box="تسلط و کاربرد", fit=(E) (F)] {};

\end{tikzpicture}%
}
\caption{فلوچارت مراحل برنامه‌ریزی}
\label{fig:flowchart_planning}
\end{figure}

\vspace{2ex} % Optional: add some vertical space after the figure

\noindent\textbf{نکات کلیدی برای فلوچارت:}

\begin{itemize}
    \item \textit{مسیر واضح:} فلش‌ها به وضوح مسیر پیشرفت را نشان می‌دهند.
    \item \textit{خلاصه‌سازی:} هر مرحله با یک عنوان کوتاه و آیکون مرتبط مشخص می‌شود.
    \item \textit{گروه‌بندی (اختیاری):} می‌توان مراحل را بر اساس سطح مهارت (مثلاً پایه‌ای، متوسط، پیشرفته) گروه‌بندی کرد.
    \item \textit{رنگ‌بندی:} استفاده از رنگ‌های مختلف برای تمایز مراحل یا گروه‌ها.
\end{itemize}

\vspace{4ex} % Add some vertical space
\newpage
\noindent\textbf{گزینه ۳: اینفوگرافیک پله‌ای یا هرمی}

\vspace{1ex} % Little space after the title
این روش رشد و ارتقا را به صورت بصری نشان می‌دهد.
\vspace{2ex} % Space before the pyramid

\begin{flushright} % To align the pyramid text to the right
سطح ۶: برنامه‌ریزی جامع دوهفته‌ای (پروژه‌ها) \\
-------------------------------------------------- \\
سطح ۵: برنامه‌ریزی هفتگی (محدودیت زمان) \\
------------------------------------------------ \\
سطح ۴: مدیریت زمان اولیه (درج ساعت، سه‌روزه) \\
---------------------------------------------- \\
سطح ۳: سازماندهی و اولویت‌بندی (لیست کارها) \\
-------------------------------------------- \\
سطح ۲: ساختاردهی برنامه (قالب مکتوب) \\
------------------------------------------ \\
سطح ۱: گزارش‌نویسی پایه (فعالیت دلخواه) \\
---------------------------------------- \\
بنیان: ایجاد عادت \\
\end{flushright}

\end{document}