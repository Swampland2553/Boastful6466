\documentclass[12pt]{article}
\usepackage{geometry}
\geometry{a4paper, margin=1in} % Adjust margins as needed

\usepackage{amsmath} % xepersian recommends loading amsmath before it
\usepackage{xepersian}
\settextfont{Amiri} % Or any other Persian font you have installed and prefer
% \settextfont[Scale=1.1]{XB Niloofar} % Example of another font with scaling

\begin{document}

\section{ساختار کتاب و بارم‌بندی (بر اساس تخمین و اهمیت مباحث)}

کتاب فیزیک (۳) شامل چهار فصل اصلی است. بارم‌بندی در امتحانات نهایی می‌تواند هر سال تغییرات جزئی داشته باشد، اما توزیع تقریبی نمرات (از ۲۰ نمره) معمولاً به شکل زیر است:

\begin{enumerate}
    \item \textbf{فصل ۱: حرکت بر خط راست (حدود ۴ نمره)}
    \begin{enumerate}
        \item درس اول: شناخت حرکت (مفاهیم اولیه، مسافت، جابجایی، تندی و سرعت متوسط)
        \item درس دوم: حرکت با سرعت ثابت (معادلات و نمودارها)
        \item درس سوم: حرکت با شتاب ثابت (معادلات، نمودارها، حرکت سقوط آزاد)
    \end{enumerate}

    \item \textbf{فصل ۲: دینامیک (حدود ۶ نمره)}
    \begin{enumerate}
        \item درس اول: قوانین حرکت نیوتون (قانون اول، دوم و سوم)
        \item درس دوم: معرفی برخی از نیروهای خاص (وزن، نیروی عمودی سطح، اصطکاک، مقاومت هوا، نیروی کشش)
        \item درس سوم: تکانه و قانون دوم نیوتون (تکانه، رابطه تکانه و ضربه، قانون پایستگی تکانه)
        \item درس چهارم: نیروی گرانشی (قانون جهانی گرانش، میدان گرانشی، وزن)
    \end{enumerate}

    \item \textbf{فصل ۳: نوسان و امواج (حدود ۶ نمره)}
    \begin{enumerate}
        \item درس اول: نوسان دوره‌ای (مفاهیم اولیه، دوره، بسامد)
        \item درس دوم: حرکت هماهنگ ساده (معادلات مکان، سرعت و شتاب، انرژی)
        \item درس سوم: انرژی در حرکت هماهنگ ساده
        \item درس چهارم: تشدید
        \item درس پنجم: موج و انواع آن (موج عرضی و طولی)
        \item درس ششم: مشخصه‌های موج (طول موج، دامنه، سرعت، دوره، بسامد)
        \item درس هفتم: بازتاب موج
        \item درس هشتم: شکست موج
    \end{enumerate}

    \item \textbf{فصل ۴: آشنایی با فیزیک اتمی و هسته‌ای (حدود ۴ نمره)}
    \begin{enumerate}
        \item درس اول: اثر فوتوالکتریک و فوتون (مفهوم فوتون، انرژی فوتون، اثر فوتوالکتریک)
        \item درس دوم: طیف خطی (طیف گسیلی و جذبی)
        \item درس سوم: مدل اتم رادرفورد - بور (مدل‌های اتمی، ترازهای انرژی، گذارهای الکترونی)
        \item درس چهارم: لیزر (اساس کار لیزر و کاربردها - در حد مفاهیم اولیه)
        \item درس پنجم: ساختار هسته (نوکلئون‌ها، عدد اتمی و جرمی، ایزوتوپ‌ها)
        \item درس ششم: پرتوزایی طبیعی و نیمه‌عمر (انواع پرتوها، نیمه‌عمر، فعالیت)
    \end{enumerate}
\end{enumerate}

\section{انواع سوالات در امتحان نهایی}

سوالات امتحان نهایی فیزیک معمولاً ترکیبی از موارد زیر است:
\begin{itemize}
    \item مفاهیم و تعاریف: سوالاتی که دانش شما را از تعاریف کلیدی (مانند تعریف تکانه، نوسان هماهنگ ساده، فوتون و...) می‌سنجند.
    \item سوالات تشریحی و مفهومی: سوالاتی که درک شما را از پدیده‌ها، قوانین و اصول فیزیکی ارزیابی می‌کنند (مثلاً توضیح قانون سوم نیوتون با یک مثال، یا شرح اثر فوتوالکتریک).
    \item مسائل حل کردنی: بخش قابل توجهی از نمره به حل مسائل عددی اختصاص دارد. این مسائل می‌توانند از یک یا چند مفهوم ترکیبی باشند.
    \item تحلیل نمودارها: توانایی شما در خواندن، تفسیر و رسم نمودارهای مرتبط با حرکت، نوسان و...
    \item سوالات کامل کردنی یا صحیح/غلط: برای سنجش نکات ریز و کلیدی.
\end{itemize}

\section{راهنمای مطالعه برای سطوح مختلف نمره}

\subsection{الف) برای گرفتن حداقل نمره قبولی (نمره ۱۰ تا ۱۲)}

\textbf{تمرکز اصلی:} مفاهیم پایه، تعاریف، فرمول‌های اصلی و حل مثال‌ها و تمرینات ساده کتاب درسی.

\textbf{چگونه بخوانیم؟}
\begin{itemize}
    \item کتاب درسی: متن کتاب را با دقت بخوانید. تعاریف مهم، قوانین و فرمول‌های اصلی هر درس را حفظ کرده و درک کنید.
    \item مثال‌های حل شده: مثال‌های داخل متن و پایان هر درس را به دقت بررسی کنید و سعی کنید خودتان آنها را حل کنید.
    \item تمرینات ساده: تمرینات و پرسش‌های مفهومی انتهای هر فصل که ساده‌تر هستند را حتماً حل کنید.
\end{itemize}

\textbf{مباحث کلیدی برای این سطح:}
\begin{itemize}
    \item \textbf{فصل ۱:} تعاریف (مسافت، جابجایی، سرعت، شتاب)، فرمول‌های حرکت با سرعت ثابت و شتاب ثابت (بدون مسائل پیچیده)، تشخیص نوع حرکت از روی نمودار.
    \item \textbf{فصل ۲:} بیان قوانین نیوتون، تعاریف نیروهای خاص (وزن، عمودی، اصطکاک)، فرمول تکانه.
    \item \textbf{فصل ۳:} تعاریف (دوره، بسامد، طول موج، دامنه)، انواع موج، معادله اصلی موج ($v=f\lambda$).
    \item \textbf{فصل ۴:} تعریف فوتون و اثر فوتوالکتریک، تعریف نیمه‌عمر، انواع پرتوها.
\end{itemize}
میزان مطالعه: حداقل ۱.۵ تا ۲ ساعت مطالعه روزانه و منظم روی مفاهیم و حل تمرینات مشخص شده.

\subsection{ب) برای گرفتن نمره قابل قبول (نمره ۱۳ تا ۱۷)}

\textbf{تمرکز اصلی:} تمام موارد بند الف + تسلط کامل بر حل تمام مسائل کتاب درسی (متوسط و دشوار)، درک عمیق‌تر مفاهیم و توانایی تحلیل نمودارها.

\textbf{چگونه بخوانیم؟}
\begin{itemize}
    \item تسلط بر کتاب درسی: علاوه بر خواندن متن، تمام مسائل و تمرینات کتاب (از جمله مسائل ترکیبی و مفهومی‌تر) را حل کنید. روی "پرسش‌ها و مسئله‌های فصل" تمرکز ویژه داشته باشید.
    \item نمودارها: توانایی رسم و تفسیر نمودارهای مکان-زمان، سرعت-زمان، شتاب-زمان و نمودارهای مربوط به نوسان و انرژی را کسب کنید.
    \item نمونه سوالات نهایی: حداقل ۳-۲ دوره از سوالات امتحانات نهایی سال‌های قبل را حل کنید تا با سبک سوالات آشنا شوید.
    \item کتاب کمک درسی: می‌توانید از یک کتاب کمک درسی معتبر برای تمرین بیشتر و دیدن مثال‌های متنوع‌تر استفاده کنید، اما کتاب درسی اولویت دارد.
\end{itemize}

\textbf{مباحث کلیدی برای این سطح (علاوه بر موارد قبل):}
\begin{itemize}
    \item \textbf{فصل ۱:} حل تمام مسائل حرکت با شتاب ثابت (از جمله سقوط آزاد و مسائل ترکیبی)، تبدیل نمودارها به یکدیگر.
    \item \textbf{فصل ۲:} کاربرد قوانین نیوتون در مسائل مختلف (سطح شیبدار ساده، آسانسور)، مسائل تکانه و قانون پایستگی تکانه، مسائل نیروی گرانش.
    \item \textbf{فصل ۳:} معادلات حرکت هماهنگ ساده (مکان، سرعت، شتاب)، مسائل انرژی در نوسان، تشدید، مسائل بازتاب و شکست موج (با استفاده از قانون اسنل در حد کتاب).
    \item \textbf{فصل ۴:} مسائل مربوط به انرژی فوتون و اثر فوتوالکتریک، گذارهای الکترونی در مدل بور و محاسبه طول موج فوتون گسیلی/جذبی، مسائل نیمه‌عمر و فعالیت.
\end{itemize}
میزان مطالعه: ۲ تا ۳ ساعت مطالعه روزانه همراه با حل مسئله زیاد و مرور نمونه سوال.

\subsection{ج) برای گرفتن نمره کامل (نمره ۱۸ تا ۲۰)}

\textbf{تمرکز اصلی:} تسلط فوق‌العاده بر تمام جزئیات کتاب درسی، توانایی حل خلاقانه مسائل چالشی و ترکیبی، درک مفهومی بسیار عمیق و قدرت تجزیه و تحلیل بالا.

\textbf{چگونه بخوانیم؟}
\begin{itemize}
    \item استاد کتاب درسی شوید: تمام خطوط کتاب، فعالیت‌ها، خوب است بدانیدها و پاورقی‌های مهم را بفهمید. هیچ نکته‌ای نباید از دید شما پنهان بماند.
    \item حل مسائل متنوع و چالشی: علاوه بر تمام مسائل کتاب، به سراغ سوالات مفهومی و چالشی‌تر از کتاب‌های کمک درسی معتبر و آزمون‌های آزمایشی استاندارد بروید.
    \item تسلط بر مفاهیم: صرفاً به حفظ فرمول اکتفا نکنید. باید بدانید هر فرمول از کجا آمده و در چه شرایطی کاربرد دارد. بتوانید مفاهیم را به زبان خودتان توضیح دهید.
    \item مدیریت زمان: در حل سوالات سرعتی و دقیق باشید.
    \item مرور و جمع‌بندی: با برنامه‌ریزی دقیق، تمام مباحث را چندین بار مرور کنید و نقاط ضعف خود را برطرف نمایید.
\end{itemize}

\textbf{مباحث کلیدی برای این سطح (علاوه بر موارد قبل):}
\begin{itemize}
    \item تسلط بر تمام زیر و بم هر چهار فصل. توانایی ترکیب مفاهیم فصول مختلف در یک مسئله.
    \item درک عمیق "چرا"ی پدیده‌ها، نه فقط "چگونه"ی حل مسائل.
    \item توانایی پاسخگویی به سوالات مفهومی که نیاز به استدلال چند مرحله‌ای دارند.
\end{itemize}
میزان مطالعه: حداقل ۳ ساعت مطالعه عمیق و فعال روزانه، همراه با حل تعداد زیادی مسئله متنوع و مرورهای منظم.

\section{نکات کلیدی برای موفقیت}
\begin{itemize}
    \item برنامه‌ریزی: یک برنامه مطالعاتی منظم و واقع‌بینانه داشته باشید و به آن پایبند بمانید.
    \item مطالعه فعال: فقط روخوانی نکنید. هنگام مطالعه یادداشت‌برداری کنید، سوال طرح کنید و سعی کنید مفاهیم را به زبان خودتان بازگو نمایید.
    \item تمرین، تمرین، تمرین: فیزیک یک درس مهارتی است. هرچه بیشتر مسئله حل کنید، تسلط شما بیشتر می‌شود.
    \item مرور: مطالب خوانده شده را در فواصل زمانی منظم مرور کنید تا فراموش نشوند.
    \item رفع اشکال: سوالات و اشکالات خود را از معلم یا دوستانتان بپرسید.
    \item خواب و تغذیه کافی: برای عملکرد بهتر مغز، استراحت و تغذیه مناسب ضروری است.
    \item اعتماد به نفس: با آمادگی کامل و نگرش مثبت در امتحان حاضر شوید.
\end{itemize}

\end{document}