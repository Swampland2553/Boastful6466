\documentclass[12pt]{article}
\usepackage{geometry}
\geometry{a4paper, margin=1in} % Adjust margins as needed

\usepackage{amsmath} % xepersian recommends loading amsmath before it
\usepackage{enumitem} % For more list control, good to have
\usepackage{xepersian}
\settextfont{Amiri} % Or any other Persian font you have installed and prefer
\SepMark{/} % Set slash as decimal separator
% \settextfont[Scale=1.1]{XB Niloofar} % Example of another font with scaling

% For unnumbered paragraph-like headings for units
\newcommand{\unithead}[1]{\par\vspace{1ex}\noindent\textbf{#1}\par\nopagebreak[4]\vspace{0.5ex}}
\newcommand{\休息}[1]{\par\centering\textit{#1}\par\vspace{1ex}} % For rest periods, using a more distinct command name

\begin{document}

\section*{برنامه ریزی درسی فشرده فیزیک ۳ رشته ریاضی (ویژه نمره کامل)}
\addcontentsline{toc}{section}{برنامه ریزی درسی فشرده فیزیک ۳ رشته ریاضی (ویژه نمره کامل)}


\subsection*{پنجشنبه ۱ خرداد (عصر)}
\addcontentsline{toc}{subsection}{پنجشنبه ۱ خرداد (عصر) - ویژه نمره کامل}
فرجه عصر (حدود ۴ ساعت مطالعه مفید):

\unithead{واحد ۱ (1/5 ساعت): مرور دینامیک و حرکت دایره‌ای (فصل ۲)}
\noindent\textbf{اولویت: }قوانین نیوتون و کاربرد آن‌ها در مسائل مختلف (سطح شیبدار، آسانسور، سیستم‌های چند جسمی ساده)، نیروی اصطکاک (ایستایی و جنبشی)، حرکت دایره‌ای یکنواخت (مخصوصاً مسائل مربوط به نیروی مرکزگرا در شرایط مختلف مانند حرکت ماهواره، خودرو در پیچ).
\par\noindent\textbf{روش: }مرور سریع جزوه و خلاصه‌نویسی‌ها، حل تعدادی تست مفهومی و مسئله تشریحی از این بخش‌ها.

\休息{استراحت (۱۵ دقیقه)}

\unithead{واحد ۲ (1/5 ساعت): ادامه مرور دینامیک و حل تست‌های چالشی}
\noindent\textbf{اولویت: }تکانه و قانون دوم نیوتون بر حسب تکانه، مسائل ترکیبی دینامیک که با سینماتیک ادغام می‌شوند.
\par\noindent\textbf{روش: }حل تست‌های سطح بالاتر و مسائل چالشی از منابع معتبر، بررسی دقیق راه‌حل‌ها و یادگیری نکات جدید.

\休息{استراحت (۱۵ دقیقه)}

\unithead{واحد ۳ (۱ ساعت): مرور سینماتیک (فصل ۱) - بخش‌های پرسوال}
\noindent\textbf{اولویت: }حرکت با شتاب ثابت (معادلات حرکت)، تحلیل نمودارهای سرعت-زمان و شتاب-زمان، سقوط آزاد.
\par\noindent\textbf{روش: }مرور سریع فرمول‌ها و نکات کلیدی، حل چند مسئله نمونه از هر بخش.

\subsection*{جمعه ۲ خرداد (صبح و عصر)}
\addcontentsline{toc}{subsection}{جمعه ۲ خرداد (صبح و عصر) - ویژه نمره کامل}
فرجه صبح (حدود ۴ ساعت مطالعه مفید):

\unithead{واحد ۱ (1/5 ساعت): مرور نوسان و موج (فصل ۳)}
\noindent\textbf{اولویت: }حرکت هماهنگ ساده (معادله مکان-زمان، سرعت و شتاب)، انرژی در حرکت هماهنگ ساده، مشخصه‌های موج (طول موج، بسامد، سرعت انتشار).
\par\noindent\textbf{روش: }مرور دقیق مفاهیم و فرمول‌ها، حل تست‌های مفهومی و محاسباتی.

\休息{استراحت (۱۵ دقیقه)}

\unithead{واحد ۲ (1/5 ساعت): ادامه مرور نوسان و موج و حل تست‌های چالشی}
\noindent\textbf{اولویت: }تشدید، انواع موج (مکانیکی و الکترومغناطیسی، طولی و عرضی)، رابطه بین سرعت، طول موج و بسامد.
\par\noindent\textbf{روش: }حل تست‌های ترکیبی و مفهومی‌تر، توجه به کاربردهای تشدید.

\休息{استراحت (۱۵ دقیقه)}

\unithead{واحد ۳ (۱ ساعت): مرور برهمکنش‌های موج (فصل ۴) - بخش‌های مهم}
\noindent\textbf{اولویت: }قانون شکست (اسنل) و مسائل مربوط به آن، تداخل امواج (سازنده و ویرانگر)، مفاهیم اولیه موج ایستاده (گره و شکم).
\par\noindent\textbf{روش: }مرور فرمول‌ها و نکات، حل چند مسئله نمونه از هر بخش.

\vspace{1em} % Add some space before the next "فرجه"
فرجه عصر (حدود ۴ ساعت مطالعه مفید):

\unithead{واحد ۱ (1/5 ساعت): مرور فیزیک اتمی (فصل ۵)}
\noindent\textbf{اولویت: }اثر فوتوالکتریک (معادله فوتوالکتریک، تابع کار، بسامد آستانه)، مدل اتم رادرفورد-بور (ترازهای انرژی، گسیل و جذب فوتون).
\par\noindent\textbf{روش: }مرور دقیق مفاهیم و روابط، حل مسائل محاسباتی مربوط به اثر فوتوالکتریک و گذارهای الکترونی.

\休息{استراحت (۱۵ دقیقه)}

\unithead{واحد ۲ (1/5 ساعت): ادامه مرور فیزیک اتمی و حل تست‌های چالشی}
\noindent\textbf{اولویت: }طیف خطی (گسیلی و جذبی) و ارتباط آن با مدل بور، مفاهیم اولیه لیزر.
\par\noindent\textbf{روش: }حل تست‌های مفهومی و ترکیبی، توجه به نکات مربوط به انواع طیف‌ها.

\休息{استراحت (۱۵ دقیقه)}

\unithead{واحد ۳ (۱ ساعت): مرور فیزیک هسته‌ای (فصل ۶) - بخش‌های اصلی}
\noindent\textbf{اولویت: }ساختار هسته (ایزوتوپ‌ها)، انواع واپاشی (آلفا، بتا، گاما) و نوشتن معادلات آن‌ها، نیمه‌عمر.
\par\noindent\textbf{روش: }مرور تعاریف و معادلات، حل چند مسئله نمونه از هر بخش.

\subsection*{شنبه ۳ خرداد (صبح و عصر) - فرجه آخر: حل نمونه سوال، مرور و رفع اشکال}
\addcontentsline{toc}{subsection}{شنبه ۳ خرداد (صبح و عصر) - ویژه نمره کامل}
فرجه صبح (حدود ۴ ساعت مطالعه مفید):

\unithead{واحد ۱ (1/5 ساعت): حل کامل یک نمونه سوال امتحان نهایی جامع (ترجیحاً سال‌های اخیر)}
\noindent\textbf{روش: }امتحان را در شرایط آزمون واقعی (با زمان‌بندی) حل کنید.

\休息{استراحت (۱۵ دقیقه)}

\unithead{واحد ۲ (1/5 ساعت): تحلیل دقیق نمونه سوال حل شده}
\noindent\textbf{روش: }پاسخ‌های خود را با راهنمای تصحیح مقایسه کنید. اشتباهات خود را شناسایی کرده و علت آن‌ها را بررسی کنید. نکات مهم سوالات را یادداشت کنید.

\休息{استراحت (۱۵ دقیقه)}

\unithead{واحد ۳ (۱ ساعت): مرور سریع و جمع‌بندی فصل‌های ۱ و ۲}
\noindent\textbf{اولویت: }فرمول‌های کلیدی، نکات مهم، نمودارها، قسمت‌هایی که در نمونه سوال مشکل داشتید.

\vspace{1em}
فرجه عصر (حدود ۴ ساعت مطالعه مفید):

\unithead{واحد ۱ (1/5 ساعت): حل بخش‌هایی از یک نمونه سوال دیگر (تمرکز بر نقاط ضعف شناسایی شده)}
\noindent\textbf{روش: }سوالاتی را انتخاب کنید که در آزمون صبح یا در طول مرورها در آن‌ها مشکل داشته‌اید.

\休息{استراحت (۱۵ دقیقه)}

\unithead{واحد ۲ (1/5 ساعت): مرور سریع و جمع‌بندی فصل‌های ۳ و ۴}
\noindent\textbf{اولویت: }فرمول‌های کلیدی، نکات مهم، نمودارها، قسمت‌هایی که در نمونه سوال مشکل داشتید.

\休息{استراحت (۱۵ دقیقه)}

\unithead{واحد ۳ (۱ ساعت): مرور سریع و جمع‌بندی فصل‌های ۵ و ۶ و مرور کلی نهایی}
\noindent\textbf{اولویت: }فرمول‌های کلیدی، نکات مهم، قسمت‌هایی که در نمونه سوال مشکل داشتید. نگاهی گذرا به تمام خلاصه‌نویسی‌ها و نکات مهم.

\subsection*{اشتباهات مهلک در این روزها (حتماً پرهیز کن!)}
\addcontentsline{toc}{subsection}{اشتباهات مهلک - ویژه نمره کامل}
\begin{itemize}
    \item شروع مطالعه مبحث جدید: این روزها زمان یادگیری مطلب جدید نیست. فقط مرور و تثبیت آموخته‌ها.
    \item وسواس بیش از حد روی یک مبحث: اگر در قسمتی مشکل اساسی داری، بیش از حد روی آن وقت نگذار. سعی کن به بقیه مباحث مسلط شوی. می‌توانی از معلم یا دوستانت برای رفع اشکال کمک بگیری.
    \item نادیده گرفتن خواب و استراحت کافی: مغز برای تثبیت اطلاعات و عملکرد بهینه نیاز به خواب کافی دارد. کم‌خوابی تمرکزت را به شدت کاهش می‌دهد.
    \item استرس و اضطراب بیش از حد: به خودت اعتماد داشته باش. با برنامه‌ریزی و تلاش، بهترین نتیجه را خواهی گرفت.
    \item حل نکردن نمونه سوال امتحان نهایی: آشنایی با سبک سوالات و مدیریت زمان در جلسه امتحان بسیار حیاتی است.
    \item مقایسه خود با دیگران: روی برنامه و پیشرفت خودت تمرکز کن.
    \item رها کردن مباحثی که فکر می‌کنی بلد هستی: یک مرور سریع هم برای این مباحث لازم است تا نکات ریز فراموش نشوند.
\end{itemize}

\subsection*{نکات طلایی برای نمره کامل در فیزیک ۳}
\addcontentsline{toc}{subsection}{نکات طلایی - ویژه نمره کامل}
\begin{itemize}
    \item تسلط بر مفاهیم پایه: مطمئن شو که تعاریف و اصول اولیه هر مبحث را به خوبی درک کرده‌ای.
    \item مهارت در حل مسئله: توانایی تجزیه و تحلیل مسئله، انتخاب فرمول مناسب و انجام محاسبات دقیق بسیار مهم است.
    \item توجه به جزئیات و یکاها: در سوالات فیزیک، جزئیات کوچک می‌توانند تفاوت ایجاد کنند. همیشه به یکاها توجه کن و در صورت نیاز تبدیل واحد انجام بده.
    \item تحلیل دقیق نمودارها: نمودارها زبان فیزیک هستند. شیب، سطح زیر نمودار و نقاط خاص روی نمودار اطلاعات مهمی به ما می‌دهند.
    \item مرور فرمول‌ها و نکات کلیدی به طور مداوم: یک برگه خلاصه از فرمول‌ها و نکات مهم تهیه کن و مدام آن را مرور کن.
    \item مدیریت زمان در جلسه امتحان: قبل از شروع به پاسخگویی، یک نگاه کلی به سوالات بینداز و زمان خود را برای هر بخش تقسیم کن. از سوالات ساده‌تر شروع کن.
    \item بازخوانی سوالات و پاسخ‌ها: اگر وقت اضافه آوردی، حتماً سوالات و پاسخ‌های خود را یک بار دیگر مرور کن تا از بروز اشتباهات سهوی جلوگیری کنی.
    \item حفظ آرامش و تمرکز: حتی اگر با سوال سختی مواجه شدی، آرامش خود را حفظ کن و سعی کن با تمرکز بیشتر آن را حل کنی یا به سراغ سوال بعدی بروی و بعداً به آن برگردی.
    \item نوشتن راه‌حل کامل و خوانا: در امتحان نهایی، مراحل حل مسئله نیز نمره دارد. راه‌حل خود را به طور کامل، منظم و خوانا بنویس.
    \item اعتماد به نفس: تو درس را کامل خوانده‌ای و برای نمره کامل تلاش می‌کنی. به توانایی‌های خودت ایمان داشته باش.
\end{itemize}

\newpage % Start the next plan on a new page for clarity
\section*{برنامه ریزی درسی فیزیک ۳ ریاضی (ویژه دانش‌آموز متوسط)}
\addcontentsline{toc}{section}{برنامه ریزی درسی فیزیک ۳ ریاضی (ویژه دانش‌آموز متوسط)}

\subsection*{پنجشنبه ۱ خرداد (عصر)}
\addcontentsline{toc}{subsection}{پنجشنبه ۱ خرداد (عصر) - ویژه دانش‌آموز متوسط}
فرجه عصر (حدود ۳ تا 3/5 ساعت مطالعه مفید):

\unithead{واحد ۱ (1/5 ساعت): مرور فصل ۲ (دینامیک و حرکت دایره‌ای) - بخش‌های پایه و پرسوال}
\noindent\textbf{اولویت: }قوانین نیوتون (به خصوص قانون دوم)، محاسبه برایند نیروها، نیروی وزن و نیروی عمودی سطح، آشنایی با نیروی اصطکاک. مفاهیم اولیه حرکت دایره‌ای (شتاب و نیروی مرکزگرا).
\par\noindent\textbf{روش: }مرور خلاصه‌نویسی‌ها یا نکات مهم کتاب درسی، حل مثال‌های کلیدی کتاب و چند تمرین ساده از این بخش‌ها.

\休息{استراحت (۱۵ دقیقه)}

\unithead{واحد ۲ (1/5 ساعت): مرور فصل ۱ (حرکت بر خط راست) - تمرکز بر معادلات و نمودارها}
\noindent\textbf{اولویت: }معادلات حرکت با شتاب ثابت، تحلیل نمودارهای مکان-زمان و سرعت-زمان (تشخیص نوع حرکت، محاسبه سرعت و شتاب از روی نمودار).
\par\noindent\textbf{روش: }مرور فرمول‌ها و روش استفاده از آن‌ها، حل تمرینات کتاب درسی مربوط به نمودارها و معادلات حرکت.

\subsection*{جمعه ۲ خرداد (صبح و عصر)}
\addcontentsline{toc}{subsection}{جمعه ۲ خرداد (صبح و عصر) - ویژه دانش‌آموز متوسط}
فرجه صبح (حدود ۳ تا 3/5 ساعت مطالعه مفید):

\unithead{واحد ۱ (1/5 ساعت): مرور فصل ۳ (نوسان و موج) - مفاهیم اصلی و روابط پایه}
\noindent\textbf{اولویت: }تعریف حرکت هماهنگ ساده، دوره، بسامد، دامنه، معادله مکان-زمان (درک کلی)، مشخصه‌های موج (طول موج، دامنه، تندی انتشار) و رابطه بین آن‌ها.
\par\noindent\textbf{روش: }مرور تعاریف و فرمول‌های اصلی، حل چند مثال ساده از هر بخش.

\休息{استراحت (۱۵ دقیقه)}

\unithead{واحد ۲ (1/5 ساعت): مرور فصل ۴ (برهمکنش‌های موج) - تمرکز بر بازتاب و شکست}
\noindent\textbf{اولویت: }قانون بازتاب، قانون شکست (اسنل) و حل مسائل ساده مربوط به آن، آشنایی با مفاهیم پراش و تداخل (در حد تعاریف و تشخیص).
\par\noindent\textbf{روش: }مرور قوانین و حل تمرینات کتاب درسی مربوط به شکست نور.

\vspace{1em}
فرجه عصر (حدود ۳ تا 3/5 ساعت مطالعه مفید):

\unithead{واحد ۱ (1/5 ساعت): مرور فصل ۵ (فیزیک اتمی) - بخش‌های مهم و پرتکرار}
\noindent\textbf{اولویت: }اثر فوتوالکتریک (معادله فوتوالکتریک، تابع کار، بسامد آستانه)، مدل اتمی بور (ترازهای انرژی، گسیل و جذب فوتون).
\par\noindent\textbf{روش: }مرور مفاهیم و فرمول‌ها، حل مسائل محاسباتی ساده مربوط به اثر فوتوالکتریک و گذارهای الکترونی.

\休息{استراحت (۱۵ دقیقه)}

\unithead{واحد ۲ (1/5 ساعت): مرور فصل ۶ (فیزیک هسته‌ای) - تعاریف و معادلات واپاشی}
\noindent\textbf{اولویت: }ساختار هسته (عدد اتمی و جرمی، ایزوتوپ)، انواع واپاشی (آلفا، بتا، گاما) و نوشتن معادلات آن‌ها، مفهوم نیمه‌عمر.
\par\noindent\textbf{روش: }مرور تعاریف و یادگیری نحوه نوشتن معادلات واپاشی، حل یک یا دو مسئله ساده نیمه‌عمر.

\subsection*{شنبه ۳ خرداد (صبح و عصر) - فرجه آخر: حل نمونه سوال، مرور و رفع اشکال}
\addcontentsline{toc}{subsection}{شنبه ۳ خرداد (صبح و عصر) - ویژه دانش‌آموز متوسط}
فرجه صبح (حدود ۳ تا 3/5 ساعت مطالعه مفید):

\unithead{واحد ۱ (1/5 ساعت): حل یک نمونه سوال امتحان نهایی (بخش‌های منتخب یا سوالات ساده‌تر)}
\noindent\textbf{روش: }سعی کنید سوالاتی را که مربوط به مباحث مرور شده و آسان‌تر به نظر می‌رسند، ابتدا حل کنید. هدف اصلی آشنایی با سبک سوالات و مدیریت زمان است.

\休息{استراحت (۱۵ دقیقه)}

\unithead{واحد ۲ (1/5 ساعت): تحلیل نمونه سوال حل شده و مرور فصل‌های ۱ و ۲}
\noindent\textbf{روش: }پاسخ‌های خود را بررسی کنید. قسمت‌هایی که مشکل داشتید را دوباره از کتاب یا جزوه مرور کنید. به بارم‌بندی سوالات توجه کنید.

\vspace{1em}
فرجه عصر (حدود ۳ تا 3/5 ساعت مطالعه مفید):

\unithead{واحد ۱ (1/5 ساعت): حل بخش‌های دیگری از نمونه سوال یا تمرکز بر رفع اشکال}
\noindent\textbf{روش: }اگر در مباحث خاصی ضعف دارید، روی سوالات مربوط به آن مباحث تمرکز کنید. می‌توانید از پاسخنامه تشریحی برای یادگیری روش حل استفاده کنید.

\休息{استراحت (۱۵ دقیقه)}

\unithead{واحد ۲ (1/5 ساعت): مرور فصل‌های ۳ و ۴ و سپس ۵ و ۶ (مرور سریع و نکات کلیدی)}
\noindent\textbf{روش: }فرمول‌های اصلی و تعاریف مهم را یک بار دیگر مرور کنید. به سوالاتی که در نمونه سوالات تکرار شده‌اند بیشتر توجه کنید.

\subsection*{اشتباهات مهلک برای دانش‌آموز متوسط (باید ازشون دوری کنی!)}
\addcontentsline{toc}{subsection}{اشتباهات مهلک - ویژه دانش‌آموز متوسط}
\begin{itemize}
    \item حفظ کردن طوطی‌وار فرمول‌ها بدون درک مفهوم: سعی کن بفهمی هر فرمول از کجا آمده و در چه شرایطی کاربرد دارد.
    \item ترس از مسائل و رها کردن آن‌ها: حتی اگر مسئله‌ای سخت به نظر می‌رسد، سعی کن قسمتی از آن را حل کنی. گاهی با شروع کردن، راه‌حل به ذهنت می‌رسد.
    \item اهمیت ندادن به تمرینات کتاب درسی: بسیاری از سوالات امتحان نهایی مشابه یا الهام گرفته از تمرینات کتاب هستند.
    \item مطالعه سطحی و رد شدن سریع از مباحث: سعی کن هر مبحثی را که می‌خوانی، به خوبی یاد بگیری.
    \item عدم مرور مطالب قبلی: مرور باعث تثبیت مطالب در ذهن می‌شود.
\end{itemize}

\subsection*{نکات طلایی برای دانش‌آموز متوسط (کمکت می‌کنه بهتر نتیجه بگیری!)}
\addcontentsline{toc}{subsection}{نکات طلایی - ویژه دانش‌آموز متوسط}
\begin{itemize}
    \item تمرکز بر مفاهیم اصلی و پرکاربرد: لازم نیست تمام جزئیات ریز را حفظ کنی. مفاهیمی که بیشتر در امتحانات تکرار می‌شوند را خوب یاد بگیر.
    \item حل مثال‌های کتاب درسی: مثال‌های حل شده در کتاب بهترین الگو برای یادگیری روش حل مسائل هستند.
    \item استفاده از خلاصه‌نویسی و فلش‌کارت: برای مرور سریع فرمول‌ها و تعاریف، خلاصه‌نویسی و فلش‌کارت بسیار مفید است.
    \item حل نمونه سوالات امتحانات نهایی سال‌های گذشته: این کار به تو کمک می‌کند با سبک سوالات آشنا شوی و نقاط ضعف خودت را پیدا کنی.
    \item پرسیدن سوال و رفع اشکال: اگر در قسمتی مشکل داری، از معلم، دوستان یا منابع آنلاین کمک بگیر.
    \item اعتماد به نفس و مثبت‌اندیشی: به توانایی‌های خودت ایمان داشته باش و با دید مثبت به سمت امتحان برو.
    \item مدیریت استرس: با تنفس عمیق و تمرکز بر روی مطالب، استرس خودت را کنترل کن.
\end{itemize}

\newpage % Start the next plan on a new page
\section*{برنامه ریزی درسی فیزیک ۳ ریاضی (ویژه کسب نمره قبولی)}
\addcontentsline{toc}{section}{برنامه ریزی درسی فیزیک ۳ ریاضی (ویژه کسب نمره قبولی)}

\subsection*{پنجشنبه ۱ خرداد (عصر)}
\addcontentsline{toc}{subsection}{پنجشنبه ۱ خرداد (عصر) - ویژه کسب نمره قبولی}
فرجه عصر (حدود ۲ تا 2/5 ساعت مطالعه مفید):

\unithead{واحد ۱ (۱ ساعت): مرور فصل ۱ (حرکت بر خط راست) - بخش‌های خیلی مهم و ساده}
\noindent\textbf{اولویت: }تعاریف مسافت، جابجایی، تندی متوسط و سرعت متوسط. آشنایی با نمودار مکان-زمان و سرعت-زمان (فقط تشخیص نوع حرکت و خواندن مقادیر ساده).
\par\noindent\textbf{روش: }مرور سریع تعاریف از روی کتاب یا جزوه، حل چند مثال خیلی ساده از کتاب.

\休息{استراحت (۱۰ دقیقه)}

\unithead{واحد ۲ (۱ ساعت): مرور فصل ۲ (دینامیک) - قوانین نیوتون و نیروهای پایه}
\noindent\textbf{اولویت: }بیان قوانین اول، دوم و سوم نیوتون. تشخیص و رسم نیروی وزن و نیروی عمودی سطح در حالت‌های ساده (سطح افقی). آشنایی با مفهوم نیروی اصطکاک (بدون ورود به محاسبات پیچیده).
\par\noindent\textbf{روش: }خواندتن متن کتاب در مورد قوانین نیوتون، نگاه کردن به شکل‌ها و مثال‌های ساده مربوط به نیروها.

\subsection*{جمعه ۲ خرداد (صبح و عصر)}
\addcontentsline{toc}{subsection}{جمعه ۲ خرداد (صبح و عصر) - ویژه کسب نمره قبولی}
فرجه صبح (حدود ۲ تا 2/5 ساعت مطالعه مفید):

\unithead{واحد ۱ (۱ ساعت): مرور فصل ۳ (نوسان و موج) - تعاریف اولیه و مشخصه‌های موج}
\noindent\textbf{اولویت: }تعریف نوسان دوره‌ای و حرکت هماهنگ ساده (در حد آشنایی). تعریف موج، طول موج، دامنه، دوره و بسامد موج.
\par\noindent\textbf{روش: }مرور تعاریف اصلی، یادگیری شکل کلی یک موج و مشخصه‌های آن.

\休息{استراحت (۱۰ دقیقه)}

\unithead{واحد ۲ (۱ ساعت): مرور فصل ۵ (فیزیک اتمی) - اثر فوتوالکتریک و مدل بور (مفاهیم اصلی)}
\noindent\textbf{اولویت: }تعریف اثر فوتوالکتریک و فوتون. آشنایی کلی با مدل اتمی بور و مفهوم ترازهای انرژی (بدون ورود به محاسبات پیچیده گذارها).
\par\noindent\textbf{روش: }خواندن توضیحات کتاب در مورد این پدیده‌ها، درک کلی اینکه نور می‌تواند ذره‌ای رفتار کند.

\vspace{1em}
فرجه عصر (حدود ۲ تا 2/5 ساعت مطالعه مفید):

\unithead{واحد ۱ (۱ ساعت): مرور فصل ۶ (فیزیک هسته‌ای) - واپاشی‌ها و نیمه‌عمر (تعاریف)}
\noindent\textbf{اولویت: }تعریف ساختار هسته (پروتون و نوترون). آشنایی با انواع واپاشی (آلفا، بتا، گاما) و اینکه چه چیزی از هسته خارج می‌شود. تعریف نیمه‌عمر (درک مفهومی).
\par\noindent\textbf{روش: }خواندن تعاریف و نگاه کردن به معادلات ساده واپاشی (بدون نیاز به حفظ کردن اعداد اتمی و جرمی خاص).

\休息{استراحت (۱۰ دقیقه)}

\unithead{واحد ۲ (۱ ساعت): حل چند نمونه سوال خیلی ساده از امتحانات نهایی (تمرکز بر سوالات تعریفی و جایگذاری در فرمول‌های ساده)}
\noindent\textbf{روش: }از نمونه سوالات سال‌های قبل، سوالات تعریفی و سوالاتی که فقط نیاز به جایگذاری در یک فرمول ساده دارند را پیدا و حل کنید. به بارم‌بندی توجه کنید تا ببینید کدام مباحث وزن بیشتری دارند.

\subsection*{شنبه ۳ خرداد (صبح و عصر) - فرجه آخر: مرور نهایی و تمرکز بر سوالات پرتکرار}
\addcontentsline{toc}{subsection}{شنبه ۳ خرداد (صبح و عصر) - ویژه کسب نمره قبولی}
فرجه صبح (حدود ۲ تا 2/5 ساعت مطالعه مفید):

\unithead{واحد ۱ (۱ ساعت): مرور سریع تمام تعاریف و فرمول‌های اصلی که در روزهای قبل خوانده‌اید.}
\noindent\textbf{روش: }از روی خلاصه‌نویسی‌ها (اگر دارید) یا هایلایت‌های کتاب، یک دور سریع تمام مطالب مهم را مرور کنید.

\休息{استراحت (۱۰ دقیقه)}

\unithead{واحد ۲ (۱ ساعت): حل مجدد مثال‌های ساده کتاب درسی از فصل‌هایی که احساس می‌کنید ضعیف‌تر هستید.}
\noindent\textbf{روش: }تمرکز بر روی درک روش حل مثال‌ها.

\vspace{1em}
فرجه عصر (حدود ۲ تا 2/5 ساعت مطالعه مفید):

\unithead{واحد ۱ (۱ ساعت): بررسی سوالات پرتکرار امتحانات نهایی (فقط خواندن صورت سوال و راه حل، برای آشنایی)}
\noindent\textbf{روش: }هدف این است که با تیپ سوالاتی که احتمال آمدنشان بیشتر است آشنا شوید. لازم نیست همه را کامل حل کنید.

\休息{استراحت (۱۰ دقیقه)}

\unithead{واحد ۲ (۱ ساعت): مرور نهایی نکات خیلی مهم و فرمول‌هایی که حتما باید بلد باشید.}
\noindent\textbf{روش: }یک لیست از مهم‌ترین فرمول‌ها و تعاریف تهیه کنید و آن‌ها را چند بار تکرار کنید.

\subsection*{اشتباهات مهلک برای کسی که دنبال نمره قبولی است (حتماً از این‌ها پرهیز کن!)}
\addcontentsline{toc}{subsection}{اشتباهات مهلک - ویژه کسب نمره قبولی}
\begin{itemize}
    \item تلاش برای یادگیری تمام جزئیات: وقت محدود است، روی کلیات و مفاهیم اصلی تمرکز کن.
    \item صرف وقت زیاد روی مباحث سخت و کم سوال: مباحثی را انتخاب کن که ساده‌تر هستند و احتمال سوال آمدن از آن‌ها بیشتر است.
    \item ایجاد استرس و ناامیدی: با یک برنامه‌ریزی سبک و واقع‌بینانه، می‌توانی نمره قبولی را کسب کنی.
    \item رها کردن کامل برخی فصول: سعی کن از هر فصل حداقل تعاریف و مفاهیم پایه را بلد باشی.
\end{itemize}

\subsection*{نکات طلایی برای کسب نمره قبولی}
\addcontentsline{toc}{subsection}{نکات طلایی - ویژه کسب نمره قبولی}
\begin{itemize}
    \item تمرکز بر تعاریف و مفاهیم پایه: بسیاری از سوالات امتحان نهایی شامل تعاریف و مفاهیم اولیه است.
    \item یادگیری فرمول‌های اصلی و پرکاربرد: چند فرمول کلیدی در هر فصل وجود دارد که با یادگیری آن‌ها می‌توانی به بخشی از سوالات پاسخ دهی.
    \item حل مثال‌های ساده کتاب درسی: این مثال‌ها به تو کمک می‌کنند تا با کاربرد فرمول‌ها آشنا شوی.
    \item بررسی نمونه سوالات امتحانات نهایی (به خصوص سوالات ساده و تعریفی): این کار به تو دید می‌دهد که چه نوع سوالاتی بیشتر تکرار می‌شوند.
    \item مثبت‌اندیشی و عدم ترس از امتحان: با آمادگی در حد توان، می‌توانی از پس امتحان بربیایی.
    \item نوشتن هر آنچه که بلدی: حتی اگر جواب کامل یک سوال را نمی‌دانی، هر قسمتی از آن را که بلد هستی بنویس. گاهی به مراحل حل هم نمره تعلق می‌گیرد.
\end{itemize}

\end{document}