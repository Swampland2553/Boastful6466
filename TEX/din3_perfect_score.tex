\documentclass{article}

% Load other packages you might need (e.g., amsmath, graphicx) BEFORE xepersian
\usepackage{fontspec}
% \usepackage{some_other_package} % Example: if you had other packages

\usepackage{xepersian} % Load xepersian AFTER all other packages

\settextfont{Amiri} % Set the main font to Amiri

\title{برنامه دینی ۳ \\ \mdseries\large برنامه فشرده و هدفمند دین و زندگی ۳ (ویژه کسب نمره کامل)}
\author{} % Optional: You can put your name here or leave it empty
\date{}   % Optional: Leave empty to not display a date

\begin{document}

\maketitle

\section*{اصول کلی برنامه:}

\begin{itemize}
    \item اولویت‌بندی: بر اساس بارم‌بندی تقریبی و اهمیت مفاهیم در امتحانات نهایی.
    \item تمرکز بر نقاط ضعف رایج: توجه ویژه به مباحثی که معمولاً برای دانش‌آموزان چالش‌برانگیزتر است.
    \item مطالعه فعال: تأکید بر درک عمیق، تحلیل و توانایی کاربرد مفاهیم.
    \item مرور و جمع‌بندی: تثبیت مطالب خوانده شده.
\end{itemize}
\section*{پنجشنبه، 25 اردیبهشت}

\subsection*{فرجه صبح (حدود 3 ساعت): بخش اول: تفکر و اندیشه - مباحث بنیادی توحید و اختیار}

\subsubsection*{1. واحد اول (1 ساعت و 30 دقیقه): درس ۱: هستی‌بخش و درس ۲: یگانه بی‌همتا}
\begin{itemize}
    \item تمرکز ویژه:
    \begin{itemize}
        \item مفهوم نیازمندی جهان به خدا در پیدایش و بقا (درس 1).
        \item مراتب توحید (خالقیت، مالکیت، ولایت، ربوبیت) و شرک متناظر با آن‌ها (درس 2).
        \item مرز دقیق توحید و شرک در استفاده از اسباب (بسیار مهم و سؤال‌خیز) (درس 2).
    \end{itemize}
    \item روش مطالعه:
    \begin{itemize}
        \item مطالعه دقیق متن کتاب با تمرکز بر تعاریف و تمایزات.
        \item بررسی عمیق آیات کلیدی (فاطر/15، الرحمن/29، اخلاص، رعد/16، آل عمران/26، انعام/164) و ارتباط آن‌ها با مفاهیم.
        \item حل سؤالات «تدبر در قرآن» و «اندیشه و تحقیق» این دو درس.
    \end{itemize}
\end{itemize}

\vspace{\baselineskip} % Adds a small vertical space before the rest
\noindent
2. استراحت (15 دقیقه)
\vspace{\baselineskip} % Adds a small vertical space after the rest

\subsubsection*{3. واحد دوم (1 ساعت و 30 دقیقه): درس ۵: قدرت پرواز}
\begin{itemize}
    \item تمرکز ویژه:
    \begin{itemize}
        \item مفهوم اختیار به عنوان حقیقتی وجدانی (با مثال‌های کتاب).
        \item تعریف دقیق قضا و قدر الهی.
        \item رابطه صحیح اختیار انسان با قضا و قدر الهی (بسیار بسیار مهم و کلیدی - باید کاملاً مسلط شوی).
        \item تفاوت علم پیشین خداوند با جبر.
        \item داستان امام علی (ع) و دیوار کج و تحلیل آن.
    \end{itemize}
    \item روش مطالعه:
    \begin{itemize}
        \item مطالعه مفهومی و تحلیلی بخش رابطه اختیار و قضا و قدر.
        \item بررسی آیات کلیدی (انسان/3، انعام/104، آل عمران/182).
        \item پاسخ تشریحی به سؤالات «تدبر در قرآن» و «اندیشه و تحقیق».
    \end{itemize}
\end{itemize}

\bigskip
\hrulefill
\bigskip

\subsection*{فرجه عصر (حدود 3 ساعت): بخش اول: تفکر و اندیشه - سنت‌های الهی و توحید عملی}

\subsubsection*{1. واحد اول (1 ساعت و 30 دقیقه): درس ۶: سنت‌های خداوند در زندگی}
\begin{itemize}
    \item تمرکز ویژه:
    \begin{itemize}
        \item تعریف سنت الهی و ویژگی‌های آن.
        \item سنت ابتلاء (امتحان): هدف و گستردگی آن.
        \item امداد عام و امداد خاص (توفیق الهی): تفاوت و مصادیق.
        \item سنت املاء و استدراج: تعریف دقیق، تفاوت و نشانه‌های هر یک (بسیار مهم و پرتکرار).
        \item سنت تأثیر اعمال در زندگی.
    \end{itemize}
    \item روش مطالعه:
    \begin{itemize}
        \item مطالعه دقیق متن با تمرکز بر تعاریف و مثال‌های هر سنت.
        \item بررسی عمیق آیات کلیدی (عنکبوت/2، اسراء/20، عنکبوت/69، آل عمران/178، اعراف/182-183، اعراف/96).
        \item حل سؤالات «تدبر در قرآن» و «اندیشه و تحقیق».
    \end{itemize}
\end{itemize}

\vspace{\baselineskip}
\noindent
2. استراحت (15 دقیقه)
\vspace{\baselineskip}

\subsubsection*{3. واحد دوم (1 ساعت و 30 دقیقه): درس ۳: توحید و سبک زندگی و درس ۴: فقط برای او}
\begin{itemize}
    \item تمرکز ویژه:
    \begin{itemize}
        \item مفهوم توحید عملی و ابعاد فردی و اجتماعی آن (درس 3).
        \item اخلاص در نیت و عمل و اهمیت آن (درس 3).
        \item حقیقت توبه و ارکان آن (به‌ویژه جبران گذشته: حقوق الهی و حقوق مردم) (درس 4).
        \item ریا (شرک خفی): تعریف، انواع و آثار آن (درس 4).
        \item راه‌های مقابله با ریا (درس 4).
    \end{itemize}
    \item روش مطالعه:
    \begin{itemize}
        \item مطالعه متن با تمرکز بر مصادیق عملی توحید و توبه، و نشانه‌های ریا.
        \item بررسی آیات و روایات کلیدی (آل عمران/51، حدیث "نیت مؤمن"، حدیث "کمال اخلاص"، شمس/9، توبه/109، حدیث "کفی بالندم"، حدیث "شرک اصغر").
        \item حل سؤالات «تدبر در قرآن» و «اندیشه و تحقیق» این دو درس.
    \end{itemize}
\end{itemize}

\bigskip
\hrulefill
\bigskip

\section*{جمعه، 26 اردیبهشت}

\subsection*{فرجه صبح (حدود 3 ساعت): بخش دوم: در مسیر - احکام و پایه‌های تمدن}

\subsubsection*{1. واحد اول (1 ساعت و 30 دقیقه): درس ۸: احکام الهی در زندگی امروز}
\begin{itemize}
    \item تمرکز ویژه:
    \begin{itemize}
        \item ضرورت و حکمت کلی احکام الهی.
        \item احکام اجتماعی در عرصه‌های فرهنگ و ارتباطات (موسیقی، مجالس شادی، تولید محتوا).
        \item احکام اجتماعی در عرصه ورزش و بازی (شرط‌بندی).
        \item احکام اجتماعی در عرصه اقتصاد (ربا، کالای ایرانی، مفاسد اقتصادی).
    \end{itemize}
    \item روش مطالعه:
    \begin{itemize}
        \item مطالعه دقیق متن با تمرکز بر حدود و شرایط هر حکم و دلایل کلی آن.
        \item بررسی آیات کلیدی (توبه/109، اسراء/32، بقره/219).
        \item حل سؤالات «تدبر در قرآن» و «اندیشه و تحقیق».
    \end{itemize}
\end{itemize}

\vspace{\baselineskip}
\noindent
2. استراحت (15 دقیقه)
\vspace{\baselineskip}

\subsubsection*{3. واحد دوم (1 ساعت و 30 دقیقه): درس ۹: پایه‌های استوار}
\begin{itemize}
    \item تمرکز ویژه:
    \begin{itemize}
        \item تعریف و اهمیت هر یک از پایه‌های تمدن اسلامی: توحید، نبوت و امامت، عدالت اجتماعی، علم و دانش، اخلاق و معنویت، خانواده، کار و تلاش.
        \item نقش هر یک از این پایه‌ها در استحکام و رشد جامعه.
    \end{itemize}
    \item روش مطالعه:
    \begin{itemize}
        \item مطالعه دقیق متن با تمرکز بر توضیح هر پایه و مثال‌های آن.
        \item بررسی آیات کلیدی (حدید/25، نساء/59).
        \item حل سؤالات «تدبر در قرآن» و «اندیشه و تحقیق»، به‌ویژه تحلیل نقش این پایه‌ها در جامعه امروز.
    \end{itemize}
\end{itemize}

\bigskip
\hrulefill
\bigskip

\subsection*{فرجه عصر (حدود 3 ساعت): بخش دوم: در مسیر - تمدن جدید و مرور}

\subsubsection*{1. واحد اول (1 ساعت و 30 دقیقه): درس ۱۰: مسئولیت ما}
\begin{itemize}
    \item تمرکز ویژه:
    \begin{itemize}
        \item آثار مثبت و منفی تمدن جدید غرب (با ذکر مصادیق).
        \item موضع صحیح مسلمانان در برابر تمدن جدید (رویکرد نقادانه و هوشمندانه).
        \item مسئولیت‌های ما (احیای تمدن اسلامی، بهره‌گیری هوشمندانه، مقابله با آثار منفی، ارائه الگو).
    \end{itemize}
    \item روش مطالعه:
    \begin{itemize}
        \item مطالعه تحلیلی متن با تمرکز بر چالش‌ها و راهکارها.
        \item بررسی آیات کلیدی (رعد/11، آل عمران/104).
        \item حل سؤالات «تدبر در قرآن» و «اندیشه و تحقیق»، با تأکید بر ارائه راهکارهای عملی.
    \end{itemize}
\end{itemize}

\vspace{\baselineskip}
\noindent
2. استراحت (15 دقیقه)
\vspace{\baselineskip}

\subsubsection*{3. واحد دوم (1 ساعت و 30 دقیقه): مرور سریع و مفهومی دروس 1 تا 6}
\begin{itemize}
    \item روش مرور:
    \begin{itemize}
        \item مرور خلاصه‌نویسی‌ها و نمودارهای مفهومی (اگر تهیه کرده‌ای).
        \item تورق سریع کتاب و یادآوری نکات کلیدی، تعاریف مهم و پیام آیات اصلی.
        \item تمرکز بر مباحثی که در مطالعه اولیه برایت دشوارتر بوده است.
    \end{itemize}
\end{itemize}

\bigskip
\hrulefill
\bigskip

\section*{شنبه، 27 اردیبهشت}

\subsection*{فرجه صبح (حدود 3 ساعت): مرور و جمع‌بندی نهایی}

\subsubsection*{1. واحد اول (1 ساعت و 30 دقیقه): مرور سریع و مفهومی دروس 7 تا 10}
\begin{itemize}
    \item روش مرور: مشابه مرور دروس 1 تا 6.
    \item تمرکز ویژه بر ارتباط مفاهیم درس 7 و 8 با زندگی عملی، و ارتباط درس 9 و 10 با مسائل کلان جامعه و جهان.
\end{itemize}

\vspace{\baselineskip}
\noindent
2. استراحت (15 دقیقه)
\vspace{\baselineskip}

\subsubsection*{3. واحد دوم (1 ساعت و 30 دقیقه): جمع‌بندی آیات و روایات مهم}
\begin{itemize}
    \item روش جمع‌بندی:
    \begin{itemize}
        \item تهیه لیستی از مهم‌ترین آیات و روایات هر درس.
        \item مرور ترجمه و پیام اصلی هر کدام.
        \item تمرین برقراری ارتباط بین آیات و روایات با مفاهیم کلیدی درس‌ها. (مثلاً کدام آیه به توحید در ربوبیت اشاره دارد؟ کدام حدیث بر اهمیت اخلاص تأکید می‌کند؟)
    \end{itemize}
\end{itemize}

\bigskip
\hrulefill
\bigskip

\subsection*{فرجه عصر (حدود 3 ساعت): حل نمونه سؤال، رفع اشکال و آمادگی نهایی}

\subsubsection*{1. واحد اول و دوم (مجموعاً حدود 3 ساعت با یک استراحت کوتاه در صورت نیاز):}
\begin{itemize}
    \item حل کامل حداقل 2 تا 3 نمونه سؤال امتحان نهایی سال‌های اخیر (با رعایت زمان‌بندی واقعی امتحان).
    \item بررسی دقیق پاسخنامه تشریحی و مقایسه با پاسخ‌های خودت.
    \item شناسایی دقیق نقاط ضعف و اشتباهات رایج در حین حل نمونه سؤالات.
    \item مراجعه مجدد به کتاب درسی برای رفع اشکال مباحثی که در آن‌ها مشکل داشته‌ای.
    \item مرور نهایی و بسیار سریع نکات کلیدی و خلاصه‌نویسی‌ها، به‌ویژه مواردی که در حل نمونه سؤالات برایت چالش‌برانگیز بوده است.
    \item ایجاد آمادگی ذهنی و روانی برای امتحان: با یادآوری تلاش‌هایت و توکل به خداوند، اعتماد به نفست را تقویت کن.
\end{itemize}

\bigskip
\hrulefill
\bigskip

\section*{اشتباهات مهلک در این روزها (تأکید مضاعف):}

\begin{enumerate}
    \item حفظ طوطی‌وار و بدون فهم: دینی درسی مفهومی و تحلیلی است. صرفاً حفظ کردن آیات و متن بدون درک عمیق، نتیجه‌بخش نخواهد بود.
    \item نادیده گرفتن بخش‌های "تدبر در قرآن" و "اندیشه و تحقیق": این بخش‌ها برای سنجش فهم و تحلیل شما بسیار مهم هستند.
    \item عدم توجه به ارتباط مفاهیم: سؤالات ترکیبی رایج هستند. باید بتوانید بین مفاهیم مختلف ارتباط برقرار کنید.
    \item مطالعه سطحی و عجولانه: برای کسب نمره کامل، نیاز به مطالعه عمیق و دقیق تمام بخش‌های کتاب است.
    \item عدم حل نمونه سؤال کافی: حل نمونه سؤال به شما کمک می‌کند با سبک سؤالات آشنا شوید و نقاط ضعف خود را پیدا کنید.
    \item ایجاد استرس و اضطراب بیش از حد: با برنامه‌ریزی صحیح و توکل به خدا، آرامش خود را حفظ کنید.
\end{enumerate}

\section*{نکات طلایی برای کسب نمره کامل (تأکید مضاعف):}

\begin{enumerate}
    \item درک عمیق آیات و روایات: ترجمه روان، پیام اصلی و ارتباط آن‌ها با مفاهیم درس را به خوبی یاد بگیرید.
    \item توانایی تحلیل و استدلال: باید بتوانید مفاهیم را به زبان خودتان توضیح دهید و برای آن‌ها استدلال بیاورید.
    \item توجه به مثال‌ها و مصادیق: سعی کنید برای هر مفهوم، مثال‌های عینی از زندگی یا از آموزه‌های دینی پیدا کنید.
    \item خلاصه‌نویسی و رسم نمودار مفهومی: این روش‌ها به تثبیت مطالب و مرور سریع کمک می‌کنند.
    \item مطالعه فعال و پرسشگرانه: در حین مطالعه از خودتان سؤال بپرسید و به دنبال پاسخ آن‌ها باشید.
    \item مرور منظم و هدفمند: مطالب خوانده شده را در فواصل زمانی مناسب مرور کنید.
\end{enumerate}
\end{document}