\documentclass[12pt,a4paper]{article}

%
% Packages to be loaded BEFORE xepersian
%
\usepackage{geometry}
\geometry{a4paper, top=2.5cm, bottom=2.5cm, left=2cm, right=2cm}
\usepackage{amsmath}
\usepackage{amssymb}
\usepackage{enumitem} % For customizing lists
\usepackage{setspace} % For line spacing
\usepackage{array}    % For better table control (though not used here, good practice)
\usepackage{hyperref} % For hyperlinks, should generally be loaded late, but before xepersian if it causes issues
\hypersetup{
    colorlinks=true,
    linkcolor=blue,
    filecolor=magenta,
    urlcolor=cyan,
    pdftitle={KPI های علمی و کاربردی},
    pdfpagemode=FullScreen,
}

%
% Load xepersian LAST
%
\usepackage{xepersian}
\settextfont{Amiri}
\setdigitfont{Amiri} % Optional: use Amiri for digits as well

% Custom command for KPI items for consistency and easier formatting
\newcommand{\kpiitem}[1]{\par\textbf{#1}}
\newcommand{\kpiseparator}{\vspace{1.5ex}\hrule\vspace{1.5ex}} % Visual separator between KPIs

\begin{document}

\onehalfspacing % Set line spacing for better readability

\begin{center}
    \Large\textbf{KPI های علمی و کاربردی}
\end{center}
\vspace{1em}

KPI های علمی و کاربردی باید بتوانند میزان دستیابی به این اهداف را به طور ملموس اندازه‌گیری کنند:

\vspace{1em}
\hrulefill
\vspace{1em}

\section*{۱. KPI های مرتبط با نتایج تحصیلی و آمادگی کنکور (اهداف کلان و موفقیت کنکور)}

\lr{\textbf{:KPI-01}} درصد افزایش میانگین تراز آزمون‌های جامع:
\kpiitem{هدف:} افزایش حداقل ۱۵٪ در میانگین تراز آزمون‌های جامع دانش‌آموزان نسبت به سال گذشته (مطابق بند ۳ سند).
\kpiitem{نحوه اندازه‌گیری:} مقایسه میانگین تراز دانش‌آموزان پایه دوازدهم در آزمون‌های جامع مشابه (مثلاً آزمون‌های آزمایشی سنجش، قلم‌چی، گاج و...) در سال تحصیلی جاری با سال گذشته.
\kpiitem{اهمیت:} نشان‌دهنده اثربخشی کلی برنامه‌های مشاوره‌ای و آموزشی در بهبود عملکرد تستی دانش‌آموزان.

\vspace{1em}
\lr{\textbf{:KPI-02}} درصد بهبود میانگین نمرات دروس نهایی:
\kpiitem{هدف:} بهبود میانگین نمرات دروس نهایی مرتبط به میزان حداقل ۰.۵ نمره.
\kpiitem{نحوه اندازه‌گیری:} مقایسه میانگین نمرات دروس نهایی دانش‌آموزان در امتحانات نهایی با سال گذشته یا با میانگین استانی/کشوری.
\kpiitem{اهمیت:} تاثیر مستقیم بر سوابق تحصیلی و کنکور، نشانگر کیفیت آموزش و مشاوره در دروس تشریحی.

\vspace{1em}
\lr{\textbf{:KPI-03}} درصد دانش‌آموزان پذیرفته شده در رشته/دانشگاه‌های هدف:
\kpiitem{هدف:} (به طور ضمنی) افزایش موفقیت در کنکور.
\kpiitem{نحوه اندازه‌گیری:} پس از اعلام نتایج کنکور، شمارش تعداد دانش‌آموزانی که در رشته‌ها و دانشگاه‌هایی که به عنوان هدف (در جلسات مشاوره) تعیین کرده بودند، پذیرفته شده‌اند.
\kpiitem{اهمیت:} شاخص نهایی موفقیت برنامه و رضایت از انتخاب رشته.

\vspace{1em}
\lr{\textbf{:KPI-04}} میانگین رتبه کنکور دانش‌آموزان:
\kpiitem{هدف:} (به طور ضمنی) ارتقای کلی عملکرد در کنکور.
\kpiitem{نحوه اندازه‌گیری:} محاسبه میانگین رتبه کشوری یا منطقه‌ای دانش‌آموزان پس از اعلام نتایج.
\kpiitem{اهمیت:} شاخص کلی عملکرد مدرسه در رقابت کنکور.

\vspace{1em}
\hrulefill
\vspace{1em}
\newpage
\section*{۲. KPI های مرتبط با مشارکت و تعامل (دانش‌آموز و والدین)}

\lr{\textbf{:KPI-05}} درصد مشارکت دانش‌آموزان در کارگاه‌های ماهانه:
\kpiitem{هدف:} مشارکت حداقل ۹۰٪ دانش‌آموزان در کارگاه‌های ماهانه (مطابق بند ۳ سند).
\kpiitem{نحوه اندازه‌گیری:} ثبت حضور و غیاب در هر کارگاه و محاسبه درصد مشارکت کل دانش‌آموزان هدف.
\kpiitem{اهمیت:} نشان‌دهنده جذابیت، اهمیت و اطلاع‌رسانی مناسب کارگاه‌ها.

\vspace{1em}
\lr{\textbf{:KPI-06}} درصد تکمیل جلسات مشاوره فردی ماهانه:
\kpiitem{هدف:} اطمینان از اجرای کامل برنامه "دو جلسه فردی ماهانه متمایز".
\kpiitem{نحوه اندازه‌گیری:} شمارش تعداد جلسات فردی برگزار شده برای هر دانش‌آموز نسبت به تعداد جلسات برنامه‌ریزی شده.
\kpiitem{اهمیت:} نشان‌دهنده پایبندی مشاوران و دانش‌آموزان به برنامه و اهمیت پایش مستمر.

\vspace{1em}
\lr{\textbf{:KPI-07}} درصد مشارکت "والد پیگیر" در همایش‌ها و جلسات تحلیل:
\kpiitem{هدف:} اطمینان از مشارکت ساختارمند والدین.
\kpiitem{نحوه اندازه‌گیری:} ثبت حضور والدین پیگیر در همایش‌های تخصصی و جلسات تحلیل عملکرد.
\kpiitem{اهمیت:} نشان‌دهنده میزان همراهی و آگاهی والدین از فرآیند تحصیلی فرزندشان.

\vspace{1em}
\hrulefill
\vspace{1em}

\section*{۳. KPI های مرتبط با رضایتمندی و سلامت روان}

\lr{\textbf{:KPI-08}} درصد رضایتمندی دانش‌آموزان از خدمات مشاوره:
\kpiitem{هدف:} کسب حداقل ۸۰٪ رضایتمندی در نظرسنجی‌های پایان دوره از دانش‌آموزان (مطابق بند ۳ سند).
\kpiitem{نحوه اندازه‌گیری:} اجرای نظرسنجی استاندارد (شامل سوالات کمی و کیفی) در پایان دوره (مثلاً پایان نیمسال یا پایان سال) در مورد کیفیت مشاور، جلسات، کارگاه‌ها و تاثیر کلی طرح.
\kpiitem{اهمیت:} سنجش مستقیم کیفیت خدمات از دیدگاه دانش‌آموز.

\vspace{1em}
\lr{\textbf{:KPI-09}} درصد رضایتمندی والدین از خدمات مشاوره و ارتباطات:
\kpiitem{هدف:} کسب حداقل ۸۰٪ رضایتمندی در نظرسنجی‌های پایان دوره از والدین (مطابق بند ۳ سند).
\kpiitem{نحوه اندازه‌گیری:} اجرای نظرسنجی استاندارد از والدین در مورد کیفیت ارتباطات، همایش‌ها، جلسات تحلیل و تاثیر طرح بر فرزندشان.
\kpiitem{اهمیت:} سنجش کیفیت خدمات و ارتباطات از دیدگاه والدین.

\vspace{1em}
\lr{\textbf{:KPI-10}} درصد کاهش گزارش‌های استرس شدید تحصیلی:
\kpiitem{هدف:} کاهش ۲۰٪ در گزارش‌های استرس شدید تحصیلی (در صورت وجود داده‌های پیشین) (مطابق بند ۳ سند).
\kpiitem{نحوه اندازه‌گیری:} از طریق پرسشنامه‌های دوره‌ای سنجش استرس یا سوالات مشخص در نظرسنجی‌های رضایت. مقایسه با داده‌های پایه (اگر موجود باشد) یا پایش روند در طول سال.
\kpiitem{اهمیت:} نشان‌دهنده تاثیر مثبت طرح بر سلامت روان و آرامش دانش‌آموزان.

\vspace{1em}
\hrulefill
\vspace{1em}

\section*{۴. KPI های مرتبط با فرآیندها و اجرای طرح}

\lr{\textbf{:KPI-11}} درصد تحقق برنامه جلسات فردی و کارگاه‌ها طبق زمان‌بندی:
\kpiitem{هدف:} اطمینان از اجرای منظم و به موقع اجزای طرح.
\kpiitem{نحوه اندازه‌گیری:} مقایسه تعداد جلسات و کارگاه‌های برگزار شده با برنامه زمان‌بندی شده اولیه.
\kpiitem{اهمیت:} پایش پایبندی به برنامه و شناسایی موانع اجرایی.

\vspace{1em}
\lr{\textbf{:KPI-12}} میانگین زمان پاسخگویی مشاور به سوالات ضروری "والد پیگیر":
\kpiitem{هدف:} اطمینان از ارتباط موثر و به موقع با والدین.
\kpiitem{نحوه اندازه‌گیری:} (می‌تواند از طریق نمونه‌گیری یا بازخورد از والدین در نظرسنجی‌ها سنجیده شود) – مثلاً "ظرف ۲۴ ساعت کاری".
\kpiitem{اهمیت:} نشان‌دهنده پاسخگو بودن سیستم مشاوره.

\vspace{1em}
\lr{\textbf{:KPI-13}} درصد ارائه گزارش‌های عملکرد ماهانه به مدیریت در موعد مقرر:
\kpiitem{هدف:} اطمینان از سیستم گزارش‌دهی منظم و استاندارد (مطابق بند ۸.۲ سند).
\kpiitem{نحوه اندازه‌گیری:} بررسی تاریخ ارسال گزارش‌های ماهانه تیم مشاوره به مدیریت.
\kpiitem{اهمیت:} تضمین شفافیت و امکان پایش روند توسط مدیریت.

\vspace{1em}
\hrulefill
\vspace{1em}
\hrulefill
\vspace{1em}

\section*{نکات مهم در استفاده از KPI ها:}
\begin{itemize}[rightmargin=1em, leftmargin=*, itemsep=0.5ex]
    \item \textbf{تعیین مقدار پایه (Baseline):} برای بسیاری از KPI ها، به‌خصوص آنهایی که به دنبال "افزایش" یا "کاهش" هستند، داشتن داده‌های سال قبل یا ابتدای دوره ضروری است.
    \item \textbf{تعیین اهداف واقع‌بینانه (Targets):} اهداف ذکر شده در سند (مثلاً ۱۵٪ افزایش تراز) باید مبنا باشند، اما برای KPI های دیگر نیز باید اهداف مشخصی تعیین شوند.
    \item \textbf{دوره زمانی اندازه‌گیری:} مشخص کنید هر KPI در چه بازه‌های زمانی اندازه‌گیری و گزارش می‌شود (مثلاً ماهانه، فصلی، سالانه).
    \item \textbf{مسئولیت‌پذیری:} برای هر KPI یک فرد یا تیم مسئول پیگیری و گزارش‌دهی تعیین شود.
    \item \textbf{بازنگری و بهبود:} KPI ها باید به طور منظم بازنگری شوند تا از مرتبط بودن و کارایی آنها اطمینان حاصل شود و در صورت نیاز اصلاح شوند.
\end{itemize}

\end{document}