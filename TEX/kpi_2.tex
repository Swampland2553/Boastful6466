\documentclass[11pt]{article}

\usepackage{amsmath}
\usepackage{array}
\usepackage[a4paper, margin=2.5cm]{geometry}
% Load xepersian after other packages as requested
\usepackage{xepersian}

% Set Amiri font
\settextfont{Amiri}
\setdigitfont{Amiri} % Optional: to ensure digits are also in Amiri

\begin{document}

\section*{نکات کلی برای فرم‌ها}

این فرم‌ها می‌توانند به صورت دیجیتال (مثلاً در اکسل یا گوگل شیتس) یا کاغذی طراحی شوند.

\textbf{نکته کلی برای فرم‌ها:} هر فرم گزارش باید شامل موارد زیر باشد:
\begin{itemize}
    \item \textbf{عنوان KPI:} (مثلاً: درصد افزایش میانگین تراز آزمون‌های جامع)
    \item \textbf{کد KPI:} (مثلاً: \lr{KPI-01})
    \item \textbf{دوره گزارش‌دهی:} (مثلاً: ماهانه - مهر ۱۴۰۴، فصلی - پاییز ۱۴۰۴، سالانه - ۱۴۰۵-۱۴۰۴)
    \item \textbf{مسئول اندازه‌گیری و گزارش:} (مثلاً: تیم مشاوره)
    \item \textbf{مقدار هدف (Target):} (مقدار از پیش تعیین شده برای KPI)
    \item \textbf{مقدار واقعی (Actual):} (مقدار اندازه‌گیری شده)
    \item \textbf{انحراف از هدف:} (تفاوت بین واقعی و هدف)
    \item \textbf{منبع داده‌ها / ابزار اندازه‌گیری:}
    \item \textbf{تحلیل و توضیحات:} (دلایل انحراف، اقدامات پیشنهادی)
    \item \textbf{تاریخ گزارش:}
\end{itemize}

\hrulefill
\bigskip

\section*{1. \lr{KPI-01}: درصد افزایش میانگین تراز آزمون‌های جامع}

\subsection*{ابزار اندازه‌گیری:}
\begin{itemize}
    \item کارنامه آزمون‌های جامع (سنجش، قلم‌چی، گاج و ...) دانش‌آموزان سال جاری.
    \item آرشیو کارنامه آزمون‌های مشابه دانش‌آموزان سال گذشته.
    \item نرم‌افزار صفحه گسترده (مانند اکسل) برای ورود داده‌ها، محاسبه میانگین و درصد افزایش.
\end{itemize}
\newpage
\subsection*{طرح فرم گزارش:}
\noindent\textbf{عنوان KPI:} درصد افزایش میانگین تراز آزمون‌های جامع \hfill \textbf{کد KPI:} \lr{KPI-01} \\
\noindent\textbf{دوره گزارش‌دهی:} پس از هر آزمون جامع / انتهای سال \hfill \textbf{مسئول:} تیم مشاوره \\

\bigskip
\begin{tabular}{ll}
\textbf{شاخص} & \textbf{سال گذشته (پایه)} \\
\hline
میانگین تراز کل دانش‌آموزان & {[عدد]} \\
\end{tabular}
\bigskip

\noindent\textbf{منبع داده‌ها:} کارنامه‌های آزمون {[نام موسسه]}، محاسبات داخلی \\
\noindent\textbf{تحلیل و توضیحات:} {[دلایل عملکرد، نقاط قوت و ضعف مشاهده شده در آزمون]} \\
\noindent\textbf{تاریخ گزارش:} \\

\hrulefill
\bigskip

\section*{2. \lr{KPI-02}: درصد بهبود میانگین نمرات دروس نهایی}

\subsection*{ابزار اندازه‌گیری:}
\begin{itemize}
    \item کارنامه‌های رسمی امتحانات نهایی دانش‌آموزان سال جاری.
    \item آرشیو کارنامه‌های رسمی امتحانات نهایی دانش‌آموزان سال گذشته.
    \item نرم‌افزار صفحه گسترده برای محاسبه میانگین و درصد بهبود.
\end{itemize}

\subsection*{طرح فرم گزارش:}
\noindent\textbf{عنوان KPI:} درصد بهبود میانگین نمرات دروس نهایی \hfill \textbf{کد KPI:} \lr{KPI-02} \\
\noindent\textbf{دوره گزارش‌دهی:} پس از اعلام نتایج امتحانات نهایی \hfill \textbf{مسئول:} تیم مشاوره / معاونت آموزشی \\

\bigskip
\begin{tabular}{ll}
\textbf{شاخص} & \textbf{سال گذشته (پایه)} \\
\hline
میانگین نمره درس {[نام درس ۱]} & {[عدد]} \\
میانگین نمره درس {[نام درس ۲]} & {[عدد]} \\
... (سایر دروس نهایی) &  \\
میانگین کل نمرات دروس نهایی & {[عدد]} \\
\end{tabular}
\bigskip

\noindent\textbf{منبع داده‌ها:} کارنامه‌های رسمی، محاسبات داخلی \\
\noindent\textbf{تحلیل و توضیحات:} {[تحلیل عملکرد در دروس مختلف]} \\
\noindent\textbf{تاریخ گزارش:} \\

\hrulefill
\bigskip
\newpage
\section*{3. \lr{KPI-03}: درصد دانش‌آموزان پذیرفته شده در رشته/دانشگاه‌های هدف}

\subsection*{ابزار اندازه‌گیری:}
\begin{itemize}
    \item فرم‌های مشاوره فردی (برای ثبت رشته/دانشگاه هدف دانش‌آموز).
    \item نتایج نهایی کنکور و انتخاب رشته دانش‌آموزان (خوداظهاری دانش‌آموز یا بررسی از طریق سازمان سنجش در صورت امکان).
    \item نرم‌افزار صفحه گسترده برای تطبیق و محاسبه درصد.
\end{itemize}

\subsection*{طرح فرم گزارش:}
\noindent\textbf{عنوان KPI:} درصد دانش‌آموزان پذیرفته شده در رشته/دانشگاه‌های هدف \hfill \textbf{کد KPI:} \lr{KPI-03} \\
\noindent\textbf{دوره گزارش‌دهی:} پس از اعلام نتایج نهایی کنکور \hfill \textbf{مسئول:} تیم مشاوره \\

\bigskip
\begin{tabular}{ll}
\textbf{شاخص} & \textbf{مقدار} \\
\hline
تعداد کل دانش‌آموزان پایه دوازدهم & {[عدد]} \\
تعداد دانش‌آموزانی که رشته/دانشگاه هدف مشخص داشته‌اند & {[عدد]} \\
تعداد دانش‌آموزان پذیرفته شده در رشته/دانشگاه هدف اولیه/ثانویه & {[عدد]} \\
درصد دانش‌آموزان پذیرفته شده در رشته/دانشگاه هدف & {[ (پذیرفته‌شدگان هدف / تعداد با هدف مشخص) * ۱۰۰ ]\%} \\
\end{tabular}
\bigskip

\noindent\textbf{مقدار هدف:} {[مثلاً ۷۰\%]} \hfill \textbf{مقدار واقعی:} {[عدد محاسبه شده]\%} \\
\noindent\textbf{منبع داده‌ها:} فرم‌های مشاوره، نتایج کنکور \\
\noindent\textbf{تحلیل و توضیحات:} {[تحلیل دلایل موفقیت یا عدم موفقیت در دستیابی به اهداف]} \\
\noindent\textbf{تاریخ گزارش:} \\

\hrulefill
\bigskip

\section*{4. \lr{KPI-04}: میانگین رتبه کنکور دانش‌آموزان}

\subsection*{ابزار اندازه‌گیری:}
\begin{itemize}
    \item نتایج کنکور دانش‌آموزان (رتبه کشوری/منطقه‌ای در هر گروه آزمایشی).
    \item نرم‌افزار صفحه گسترده برای محاسبه میانگین.
\end{itemize}

\subsection*{طرح فرم گزارش:}
\noindent\textbf{عنوان KPI:} میانگین رتبه کنکور دانش‌آموزان \hfill \textbf{کد KPI:} \lr{KPI-04} \\
\noindent\textbf{دوره گزارش‌دهی:} پس از اعلام نتایج کنکور \hfill \textbf{مسئول:} تیم مشاوره \\

\bigskip
\begin{tabular}{ll}
\textbf{گروه آزمایشی} & \textbf{تعداد شرکت‌کننده از دبیرستان} \\
\hline
ریاضی & {[عدد]} \\
تجربی & {[عدد]} \\
انسانی & {[عدد]} \\
\end{tabular}
\bigskip

\noindent\textbf{منبع داده‌ها:} نتایج کنکور دانش‌آموزان \\
\noindent\textbf{تحلیل و توضیحات:} {[مقایسه با سال‌های قبل، تحلیل عملکرد گروه‌های مختلف]} \\
\noindent\textbf{تاریخ گزارش:} \\

\hrulefill
\bigskip

\section*{5. \lr{KPI-05}: درصد مشارکت دانش‌آموزان در کارگاه‌های ماهانه}

\subsection*{ابزار اندازه‌گیری:}
\begin{itemize}
    \item لیست حضور و غیاب کارگاه‌ها (کاغذی یا دیجیتال).
    \item لیست اسامی کل دانش‌آموزان پایه دوازدهم.
    \item نرم‌افزار صفحه گسترده.
\end{itemize}

\subsection*{طرح فرم گزارش:}
\noindent\textbf{عنوان KPI:} درصد مشارکت دانش‌آموزان در کارگاه‌های ماهانه \hfill \textbf{کد KPI:} \lr{KPI-05} \\
\noindent\textbf{دوره گزارش‌دهی:} ماهانه (پس از هر کارگاه) \hfill \textbf{مسئول:} تیم مشاوره \\

\bigskip
\begin{tabular}{ll}
\textbf{عنوان کارگاه / تاریخ} & \textbf{تعداد کل دانش‌آموزان هدف} \\ % Or perhaps "تعداد شرکت کننده" if that's what [عدد] refers to
\hline
{[کارگاه ۱ / تاریخ]} & {[عدد]} \\
{[کارگاه ۲ / تاریخ]} & {[عدد]} \\
\end{tabular}
\bigskip

\noindent میانگین مشارکت ماهانه/دوره \\
\noindent\textbf{مقدار هدف:} ۹۰\% \hfill \textbf{مقدار واقعی (میانگین دوره):} {[عدد محاسبه شده]\%} \\
\noindent\textbf{منبع داده‌ها:} لیست حضور و غیاب \\
\noindent\textbf{تحلیل و توضیحات:} {[دلایل مشارکت بالا/پایین، بازخورد دانش‌آموزان از کارگاه]} \\
\noindent\textbf{تاریخ گزارش:} \\

\hrulefill
\bigskip

\section*{6. \lr{KPI-06}: درصد تکمیل جلسات مشاوره فردی ماهانه}

\subsection*{ابزار اندازه‌گیری:}
\begin{itemize}
    \item تقویم جلسات مشاوره فردی برنامه‌ریزی شده.
    \item سیستم ثبت جلسات برگزار شده (دیجیتال یا دفتر ثبت).
    \item نرم‌افزار صفحه گسترده برای محاسبه درصد تکمیل.
\end{itemize}

\subsection*{طرح فرم گزارش:}
\noindent\textbf{عنوان KPI:} درصد تکمیل جلسات مشاوره فردی ماهانه \hfill \textbf{کد KPI:} \lr{KPI-06} \\
\noindent\textbf{دوره گزارش‌دهی:} ماهانه \hfill \textbf{مسئول:} تیم مشاوره \\

\bigskip
\begin{tabular}{lll}
\textbf{شاخص} & \textbf{تعداد} & \textbf{درصد} \\
\hline
تعداد کل جلسات مشاوره فردی برنامه‌ریزی شده & {[عدد]} & 100\% \\
تعداد جلسات برگزار شده طبق برنامه & {[عدد]} & {[محاسبه]\%} \\
تعداد جلسات برگزار شده با تاخیر & {[عدد]} & {[محاسبه]\%} \\
تعداد جلسات برگزار نشده & {[عدد]} & {[محاسبه]\%} \\
\end{tabular}
\bigskip

\noindent\textbf{مقدار هدف:} 95\% \hfill \textbf{مقدار واقعی:} {[عدد محاسبه شده]\%} \\
\noindent\textbf{منبع داده‌ها:} تقویم جلسات، سیستم ثبت جلسات \\
\noindent\textbf{تحلیل و توضیحات:} {[دلایل عدم برگزاری یا تاخیر در جلسات، اقدامات اصلاحی]} \\
\noindent\textbf{تاریخ گزارش:} \\

\hrulefill
\bigskip

\section*{7. \lr{KPI-07}: درصد مشارکت "والد پیگیر" در همایش‌ها و جلسات تحلیل}

\subsection*{ابزار اندازه‌گیری:}
\begin{itemize}
    \item لیست حضور و غیاب والدین در همایش‌ها و جلسات.
    \item لیست کامل والدین پیگیر (والدینی که مسئولیت پیگیری تحصیلی دانش‌آموز را بر عهده دارند).
    \item نرم‌افزار صفحه گسترده برای محاسبه درصد مشارکت.
\end{itemize}

\subsection*{طرح فرم گزارش:}
\noindent\textbf{عنوان KPI:} درصد مشارکت "والد پیگیر" در همایش‌ها و جلسات تحلیل \hfill \textbf{کد KPI:} \lr{KPI-07} \\
\noindent\textbf{دوره گزارش‌دهی:} پس از هر همایش/جلسه \hfill \textbf{مسئول:} تیم مشاوره / روابط عمومی \\

\bigskip
\begin{tabular}{lll}
\textbf{عنوان همایش/جلسه} & \textbf{تعداد کل والدین پیگیر} & \textbf{تعداد شرکت‌کننده} \\
\hline
{[عنوان همایش/جلسه ۱]} & {[عدد]} & {[عدد]} \\
{[عنوان همایش/جلسه ۲]} & {[عدد]} & {[عدد]} \\
\end{tabular}
\bigskip

\noindent\textbf{مقدار هدف:} 80\% \hfill \textbf{مقدار واقعی:} {[عدد محاسبه شده]\%} \\
\noindent\textbf{منبع داده‌ها:} لیست حضور و غیاب، فرم‌های ثبت‌نام \\
\noindent\textbf{تحلیل و توضیحات:} {[دلایل مشارکت بالا/پایین، بازخورد والدین]} \\
\noindent\textbf{تاریخ گزارش:} \\

\hrulefill
\bigskip

\section*{8. \lr{KPI-08}: درصد رضایتمندی دانش‌آموزان از خدمات مشاوره}

\subsection*{ابزار اندازه‌گیری:}
\begin{itemize}
    \item پرسشنامه رضایتمندی دانش‌آموزان (کاغذی یا آنلاین).
    \item نرم‌افزار تحلیل داده‌های پرسشنامه.
    \item مصاحبه‌های نمونه‌ای با دانش‌آموزان (اختیاری).
\end{itemize}

\subsection*{طرح فرم گزارش:}
\noindent\textbf{عنوان KPI:} درصد رضایتمندی دانش‌آموزان از خدمات مشاوره \hfill \textbf{کد KPI:} \lr{KPI-08} \\
\noindent\textbf{دوره گزارش‌دهی:} فصلی \hfill \textbf{مسئول:} تیم مشاوره / واحد ارزیابی \\

\bigskip
\begin{tabular}{ll}
\textbf{حوزه خدمات} & \textbf{میانگین امتیاز رضایتمندی (از 5)} \\
\hline
مشاوره فردی & {[عدد]} \\
کارگاه‌های گروهی & {[عدد]} \\
مشاوره تحصیلی & {[عدد]} \\
مشاوره انتخاب رشته & {[عدد]} \\
دسترسی به مشاوران & {[عدد]} \\
میانگین کل & {[عدد]} \\
\end{tabular}
\bigskip

\noindent\textbf{مقدار هدف:} 2/4 از 5 (84\%) \hfill \textbf{مقدار واقعی:} {[عدد محاسبه شده]} ({[درصد]\%}) \\
\noindent\textbf{منبع داده‌ها:} پرسشنامه‌های رضایتمندی، مصاحبه‌ها \\
\noindent\textbf{تحلیل و توضیحات:} {[نقاط قوت و ضعف شناسایی شده، اقدامات بهبود]} \\
\noindent\textbf{تاریخ گزارش:} \\

\hrulefill
\bigskip

\section*{9. \lr{KPI-09}: درصد رضایتمندی والدین از خدمات مشاوره و ارتباطات}

\subsection*{ابزار اندازه‌گیری:}
\begin{itemize}
    \item پرسشنامه رضایتمندی والدین (کاغذی یا آنلاین).
    \item نرم‌افزار تحلیل داده‌های پرسشنامه.
    \item بازخوردهای دریافتی از والدین در جلسات.
\end{itemize}

\subsection*{طرح فرم گزارش:}
\noindent\textbf{عنوان KPI:} درصد رضایتمندی والدین از خدمات مشاوره و ارتباطات \hfill \textbf{کد KPI:} \lr{KPI-09} \\
\noindent\textbf{دوره گزارش‌دهی:} فصلی \hfill \textbf{مسئول:} تیم مشاوره / واحد ارزیابی \\

\bigskip
\begin{tabular}{ll}
\textbf{حوزه خدمات} & \textbf{میانگین امتیاز رضایتمندی (از 5)} \\
\hline
کیفیت اطلاع‌رسانی & {[عدد]} \\
جلسات مشاوره خانواده & {[عدد]} \\
همایش‌ها و کارگاه‌های والدین & {[عدد]} \\
گزارش‌دهی پیشرفت تحصیلی & {[عدد]} \\
پاسخگویی به سوالات و نگرانی‌ها & {[عدد]} \\
میانگین کل & {[عدد]} \\
\end{tabular}
\bigskip

\noindent\textbf{مقدار هدف:} 4 از 5 (80\%) \hfill \textbf{مقدار واقعی:} {[عدد محاسبه شده]} ({[درصد]\%}) \\
\noindent\textbf{منبع داده‌ها:} پرسشنامه‌های رضایتمندی، بازخوردهای جلسات \\
\noindent\textbf{تحلیل و توضیحات:} {[نقاط قوت و ضعف شناسایی شده، اقدامات بهبود]} \\
\noindent\textbf{تاریخ گزارش:} \\

\hrulefill
\bigskip

\section*{10. \lr{KPI-10}: درصد کاهش گزارش‌های استرس شدید تحصیلی}

\subsection*{ابزار اندازه‌گیری:}
\begin{itemize}
    \item پرسشنامه استاندارد سنجش استرس تحصیلی (در ابتدا و انتهای سال تحصیلی).
    \item فرم‌های ثبت مراجعات دانش‌آموزان به مشاور با شکایت استرس.
    \item نرم‌افزار تحلیل داده‌ها برای مقایسه و محاسبه درصد کاهش.
\end{itemize}

\subsection*{طرح فرم گزارش:}
\noindent\textbf{عنوان KPI:} درصد کاهش گزارش‌های استرس شدید تحصیلی \hfill \textbf{کد KPI:} \lr{KPI-10} \\
\noindent\textbf{دوره گزارش‌دهی:} نیم‌سالی (مقایسه ابتدا و انتهای هر نیم‌سال) \hfill \textbf{مسئول:} تیم مشاوره \\

\bigskip
\begin{tabular}{lll}
\textbf{شاخص} & \textbf{ابتدای دوره} & \textbf{انتهای دوره} \\
\hline
تعداد دانش‌آموزان با استرس شدید (نمره بالای X در پرسشنامه) & {[عدد]} & {[عدد]} \\
درصد دانش‌آموزان با استرس شدید & {[درصد]\%} & {[درصد]\%} \\
تعداد مراجعات مرتبط با استرس & {[عدد]} & {[عدد]} \\
\end{tabular}
\bigskip

\noindent\textbf{درصد کاهش:} {[محاسبه]\%} \\
\noindent\textbf{مقدار هدف:} 30\% کاهش \hfill \textbf{مقدار واقعی:} {[عدد محاسبه شده]\%} \\
\noindent\textbf{منبع داده‌ها:} پرسشنامه‌های استرس تحصیلی، فرم‌های مراجعه \\
\noindent\textbf{تحلیل و توضیحات:} {[عوامل موثر در کاهش/افزایش استرس، اقدامات انجام شده]} \\
\noindent\textbf{تاریخ گزارش:} \\

\hrulefill
\bigskip

\section*{11. \lr{KPI-11}: درصد تحقق برنامه جلسات فردی و کارگاه‌ها طبق زمان‌بندی}

\subsection*{ابزار اندازه‌گیری:}
\begin{itemize}
    \item برنامه زمان‌بندی سالانه جلسات فردی و کارگاه‌ها.
    \item سیستم ثبت جلسات و کارگاه‌های برگزار شده.
    \item نرم‌افزار صفحه گسترده برای محاسبه درصد تحقق.
\end{itemize}

\subsection*{طرح فرم گزارش:}
\noindent\textbf{عنوان KPI:} درصد تحقق برنامه جلسات فردی و کارگاه‌ها طبق زمان‌بندی \hfill \textbf{کد KPI:} \lr{KPI-11} \\
\noindent\textbf{دوره گزارش‌دهی:} ماهانه \hfill \textbf{مسئول:} تیم مشاوره \\

\bigskip
\begin{tabular}{lll}
\textbf{نوع برنامه} & \textbf{تعداد برنامه‌ریزی شده} & \textbf{تعداد اجرا شده طبق زمان‌بندی} \\
\hline
جلسات فردی & {[عدد]} & {[عدد]} \\
کارگاه‌های گروهی & {[عدد]} & {[عدد]} \\
همایش‌های والدین & {[عدد]} & {[عدد]} \\
جمع کل & {[عدد]} & {[عدد]} \\
\end{tabular}
\bigskip

\noindent\textbf{درصد تحقق:} {[محاسبه]\%} \\
\noindent\textbf{مقدار هدف:} 90\% \hfill \textbf{مقدار واقعی:} {[عدد محاسبه شده]\%} \\
\noindent\textbf{منبع داده‌ها:} برنامه زمان‌بندی، سیستم ثبت جلسات \\
\noindent\textbf{تحلیل و توضیحات:} {[دلایل تاخیرها (در صورت وجود)، اقدامات اصلاحی]} \\
\noindent\textbf{تاریخ گزارش:} \\

\hrulefill
\bigskip

\section*{12. \lr{KPI-12}: میانگین زمان پاسخگویی مشاور به سوالات ضروری "والد پیگیر"}

\subsection*{ابزار اندازه‌گیری:}
\begin{itemize}
    \item (ایده‌آل) سیستم ثبت تیکت یا لاگ تماس/پیام مشاوران.
    \item (عملی‌تر) سوال در پرسشنامه رضایت‌سنجی والدین در مورد سرعت پاسخگویی.
\end{itemize}

\subsection*{طرح فرم گزارش:}
\noindent\textbf{عنوان KPI:} میانگین زمان پاسخگویی مشاور به سوالات ضروری "والد پیگیر" \hfill \textbf{کد KPI:} \lr{KPI-12} \\
\noindent\textbf{دوره گزارش‌دهی:} فصلی / پایان دوره \hfill \textbf{مسئول:} سرپرست تیم مشاوره / مدیریت \\

\bigskip
\begin{tabular}{ll}
\textbf{شاخص} & \textbf{مقدار} \\
\hline
درصد والدینی که سرعت پاسخگویی را "خوب" یا "عالی" ارزیابی کرده‌اند (از پرسشنامه) & {[درصد]} \\
(در صورت وجود لاگ) میانگین زمان پاسخگویی (ساعت/روز کاری) & {[عدد] ساعت/روز} \\
\end{tabular}
\bigskip

\noindent\textbf{مقدار هدف:} {[مثلاً ۸۵\% رضایت از سرعت پاسخ / پاسخگویی ظرف ۲۴ ساعت کاری]} \hfill \textbf{مقدار واقعی:} {[عدد محاسبه شده]} \\
\noindent\textbf{منبع داده‌ها:} پرسشنامه رضایت‌سنجی والدین، لاگ ارتباطات (اختیاری) \\
\noindent\textbf{تحلیل و توضیحات:} {[چالش‌های پاسخگویی، پیشنهادات برای بهبود]} \\
\noindent\textbf{تاریخ گزارش:} \\

\hrulefill
\bigskip

\section*{13. \lr{KPI-13}: درصد ارائه گزارش‌های عملکرد ماهانه به مدیریت در موعد مقرر}

\subsection*{ابزار اندازه‌گیری:}
\begin{itemize}
    \item تقویم و لیست موعد تحویل گزارش‌ها.
    \item تاریخ واقعی تحویل گزارش‌ها به مدیریت.
\end{itemize}

\subsection*{طرح فرم گزارش:}
\noindent\textbf{عنوان KPI:} درصد ارائه گزارش‌های عملکرد ماهانه به مدیریت در موعد مقرر \hfill \textbf{کد KPI:} \lr{KPI-13} \\
\noindent\textbf{دوره گزارش‌دهی:} ماهانه (توسط مدیریت یا مسئول کنترل پروژه) \hfill \textbf{مسئول:} مدیریت / کنترل پروژه \\

\bigskip
\begin{tabular}{ll}
\textbf{ماه گزارش} & \textbf{موعد تحویل} \\
\hline
{[مهر]} & {[تاریخ]} \\
{[آبان]} & {[تاریخ]} \\
... & \\
\end{tabular}
\bigskip

\noindent درصد گزارش‌های تحویل شده به موقع در دوره \hfill {[ (تعداد به موقع / کل گزارش‌ها) * ۱۰۰ ]\%} \\
\noindent\textbf{مقدار هدف:} ۱۰۰\% \hfill \textbf{مقدار واقعی:} {[عدد محاسبه شده]\%} \\
\noindent\textbf{منبع داده‌ها:} سوابق دریافت گزارش‌ها توسط مدیریت \\
\noindent\textbf{تحلیل و توضیحات:} {[دلایل تاخیر (در صورت وجود)]} \\
\noindent\textbf{تاریخ گزارش:} \\

\hrulefill
\bigskip

این فرم‌ها به عنوان یک الگو هستند و می‌توانید بر اساس نیازها و جزئیات بیشتر طرح خودتان، آن‌ها را سفارشی‌سازی کنید. مهم این است که فرآیند اندازه‌گیری و گزارش‌دهی شفاف، منظم و قابل اتکا باشد.
\bigskip

---

\end{document}