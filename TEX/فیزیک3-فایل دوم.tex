\documentclass[12pt]{article}
\usepackage{geometry}
\geometry{a4paper, margin=1in} % Adjust margins as needed

\usepackage{amsmath} % xepersian recommends loading amsmath before it
\usepackage{enumitem} % For more list control, good to have
\usepackage{xepersian}
\settextfont{Amiri} % Or any other Persian font you have installed and prefer
% \settextfont[Scale=1.1]{XB Niloofar} % Example of another font with scaling

% For unnumbered paragraph-like headings
\newcommand{\parahead}[1]{\par\vspace{0.5ex}\noindent\textbf{#1}\par\nopagebreak[4]\vspace{0.5ex}}

\begin{document}

\begin{center}
    \Large\textbf{تحلیل درس فیزیک ۳ رشته ریاضی و آمادگی برای امتحان نهایی}
\end{center}
\vspace{0.5cm}

درس فیزیک ۳ شامل مباحث مهمی از مکانیک، موج و فیزیک جدید هست که هر کدوم ویژگی‌های خاص خودشون رو در امتحان نهایی دارن. بیا با هم فصل به فصل بررسی کنیم:

\section{فصل ۱: حرکت بر خط راست (سینماتیک)}

\parahead{مباحث اصلی:}
شناخت حرکت، مسافت و جابه‌جایی، تندی متوسط و سرعت متوسط، سرعت لحظه‌ای، شتاب متوسط و شتاب لحظه‌ای، انواع حرکت (با سرعت ثابت، با شتاب ثابت)، سقوط آزاد.

\parahead{نحوه سوالات در امتحان نهایی:}
\begin{itemize}
    \item \textbf{سوالات ساده:} تعاریف اولیه (مسافت، جابه‌جایی، سرعت، شتاب)، محاسبه تندی و سرعت متوسط از روی نمودار یا اطلاعات داده شده، تشخیص نوع حرکت از روی نمودار مکان-زمان یا سرعت-زمان. حل تمرینات مشابه کتاب درسی.
    \item \textbf{سوالات متوسط:} مسائل مربوط به حرکت با شتاب ثابت (استفاده از معادلات حرکت)، تحلیل نمودارهای سرعت-زمان و شتاب-زمان، مسائل مربوط به سقوط آزاد (اغلب با فرض مقاومت هوای ناچیز). نیاز به حل نمونه سوالات بیشتر و تمرینات کتاب‌های کمک درسی.
    \item \textbf{سوالات چالشی:} مسائل ترکیبی که نیاز به درک عمیق مفاهیم و استفاده همزمان از چند معادله دارند، تحلیل نمودارهای پیچیده‌تر، سوالاتی که نیاز به خلاقیت در راه‌حل دارند.
\end{itemize}

\parahead{بارم‌بندی تقریبی در امتحان نهایی:}
حدود ۴ تا ۵ نمره.

\section{فصل ۲: دینامیک و حرکت دایره‌ای}

\parahead{مباحث اصلی:}
قوانین حرکت نیوتون (قانون اول، دوم و سوم)، معرفی برخی از نیروهای خاص (وزن، نیروی عمودی سطح، نیروی اصطکاک، نیروی کشش نخ/فنر)، تکانه و قانون دوم نیوتون بر حسب تکانه، حرکت دایره‌ای یکنواخت (شتاب مرکزگرا، نیروی مرکزگرا)، نیروی گرانشی.

\parahead{نحوه سوالات در امتحان نهایی:}
\begin{itemize}
    \item \textbf{سوالات ساده:} بیان قوانین نیوتون، شناسایی نیروهای وارد بر جسم در حالت‌های ساده، محاسبه تکانه.
    \item \textbf{سوالات متوسط:} کاربرد قانون دوم نیوتون در حل مسائل (با نیروهای در یک راستا یا عمود بر هم)، محاسبه نیروی اصطکاک، مسائل مربوط به حرکت دایره‌ای یکنواخت (محاسبه شتاب و نیروی مرکزگرا)، مسائل ساده نیروی گرانشی.
    \item \textbf{سوالات چالشی:} مسائل ترکیبی دینامیک که نیاز به تحلیل دقیق نیروها و انتخاب دستگاه مختصات مناسب دارند، مسائل حرکت دایره‌ای در شرایط مختلف (مثلاً در پیچ جاده)، ترکیب مفاهیم دینامیک با سینماتیک.
\end{itemize}

\parahead{بارم‌بندی تقریبی در امتحان نهایی:}
حدود ۵ تا ۶ نمره.

\section{فصل ۳: نوسان و موج}

\parahead{مباحث اصلی:}
نوسان دوره‌ای، حرکت هماهنگ ساده (معادله مکان-زمان، بسامد زاویه‌ای، دوره، بسامد)، انرژی در حرکت هماهنگ ساده، تشدید، موج و انواع آن (مکانیکی، الکترومغناطیسی، طولی، عرضی)، مشخصه‌های موج (طول موج، دامنه، تندی انتشار).

\parahead{نحوه سوالات در امتحان نهایی:}
\begin{itemize}
    \item \textbf{سوالات ساده:} تعاریف اولیه (نوسان دوره‌ای، حرکت هماهنگ ساده، موج، طول موج، دامنه)، محاسبه دوره و بسامد از روی اطلاعات داده شده.
    \item \textbf{سوالات متوسط:} استفاده از معادله مکان-زمان در حرکت هماهنگ ساده، محاسبه انرژی مکانیکی نوسانگر، مسائل مربوط به رابطه تندی، طول موج و بسامد موج، تشخیص نوع موج.
    \item \textbf{سوالات چالشی:} مسائل مربوط به تشدید و کاربردهای آن، تحلیل نمودارهای مربوط به نوسان و موج، مسائل ترکیبی که نیاز به درک عمیق پدیده‌های موجی دارند.
\end{itemize}

\parahead{بارم‌بندی تقریبی در امتحان نهایی:}
حدود ۴ تا ۵ نمره.

\section{فصل ۴: برهمکنش‌های موج}

\parahead{مباحث اصلی:}
بازتاب موج، شکست موج (قانون شکست عمومی یا اسنل)، پراش موج، تداخل امواج (سازنده و ویرانگر)، موج ایستاده.

\parahead{نحوه سوالات در امتحان نهایی:}
\begin{itemize}
    \item \textbf{سوالات ساده:} تعاریف بازتاب، شکست، پراش و تداخل، بیان قانون بازتاب.
    \item \textbf{سوالات متوسط:} مسائل مربوط به قانون شکست (اسنل)، شناسایی پدیده پراش در شرایط مختلف، تشخیص تداخل سازنده و ویرانگر از روی شکل یا اطلاعات، مفاهیم اولیه موج ایستاده (گره و شکم).
    \item \textbf{سوالات چالشی:} مسائل ترکیبی برهمکنش‌های موج، تحلیل پدیده‌هایی مانند سراب با استفاده از مفهوم شکست، مسائل مربوط به تشکیل نقش‌های تداخلی.
\end{itemize}

\parahead{بارم‌بندی تقریبی در امتحان نهایی:}
حدود ۳ تا ۴ نمره.

\section{فصل ۵: آشنایی با فیزیک اتمی}

\parahead{مباحث اصلی:}
اثر فوتوالکتریک و فوتون (انرژی فوتون، تابع کار، بسامد آستانه، معادله فوتوالکتریک)، طیف خطی (گسیلی و جذبی)، مدل اتم رادرفورد-بور (اصول و مفروضات، ترازهای انرژی، گسیل و جذب فوتون)، لیزر.

\parahead{نحوه سوالات در امتحان نهایی:}
\begin{itemize}
    \item \textbf{سوالات ساده:} تعاریف اثر فوتوالکتریک، فوتون، طیف خطی، بیان اصول مدل بور.
    \item \textbf{سوالات متوسط:} مسائل مربوط به معادله فوتوالکتریک، محاسبه انرژی و طول موج فوتون گسیل شده یا جذب شده در گذارهای الکترونی، توضیح چگونگی تشکیل طیف خطی.
    \item \textbf{سوالات چالشی:} مسائل ترکیبی فیزیک اتمی، تحلیل نمودارهای مربوط به اثر فوتوالکتریک، سوالاتی که نیاز به درک عمیق مفاهیم کوانتومی اولیه دارند.
\end{itemize}

\parahead{بارم‌بندی تقریبی در امتحان نهایی:}
حدود ۳ تا ۴ نمره.

\section{فصل ۶: آشنایی با فیزیک هسته‌ای}

\parahead{مباحث اصلی:}
ساختار هسته (عدد اتمی، عدد جرمی، ایزوتوپ)، پرتوزایی طبیعی و نیمه‌عمر (واپاشی آلفا، بتا و گاما)، شکافت هسته‌ای (واکنش زنجیره‌ای)، گداخت (همجوشی) هسته‌ای.

\parahead{نحوه سوالات در امتحان نهایی:}
\begin{itemize}
    \item \textbf{سوالات ساده:} تعاریف هسته، نوکلئون، ایزوتوپ، انواع واپاشی، نیمه‌عمر.
    \item \textbf{سوالات متوسط:} نوشتن معادلات واپاشی، مسائل ساده مربوط به نیمه‌عمر، توضیح مفاهیم شکافت و گداخت.
    \item \textbf{سوالات چالشی:} مسائل ترکیبی پرتوزایی، مقایسه انرژی آزاد شده در شکافت و گداخت، سوالاتی که نیاز به درک کاربردهای فیزیک هسته‌ای دارند.
\end{itemize}

\parahead{بارم‌بندی تقریبی در امتحان نهایی:}
حدود ۳ تا ۴ نمره.

\section{راهنمای مطالعه برای سطوح مختلف نمره}

\subsection{برای گرفتن حداقل نمره (حدود ۱۰ تا ۱۲)}

\parahead{تمرکز:}
روی تعاریف اصلی، فرمول‌های پایه و حل مثال‌ها و تمرینات ساده کتاب درسی.

\parahead{روش مطالعه:}
ابتدا متن کتاب درسی را به دقت بخوانید و مفاهیم کلیدی را هایلایت کنید. سپس مثال‌های حل شده را بررسی کرده و سعی کنید خودتان آن‌ها را حل کنید. در نهایت، تمرینات ساده انتهای هر فصل را حل کنید.

\parahead{مباحث پیشنهادی برای تمرکز بیشتر:}
\begin{itemize}
    \item فصل ۱: تعاریف سرعت و شتاب، نمودارهای ساده مکان-زمان و سرعت-زمان.
    \item فصل ۲: قوانین نیوتون، تعریف وزن و نیروی عمودی.
    \item فصل ۳: تعاریف موج و نوسان، طول موج و دامنه.
    \item فصل ۵: تعریف اثر فوتوالکتریک و فوتون.
    \item فصل ۶: تعاریف واپاشی و نیمه‌عمر.
\end{itemize}

\subsection{برای گرفتن نمره قابل قبول (حدود ۱۳ تا ۱۷)}

\parahead{تمرکز:}
علاوه بر موارد بالا، تسلط بر حل مسائل متوسط کتاب درسی، فهم عمیق‌تر نمودارها و کاربرد فرمول‌ها در شرایط مختلف.

\parahead{روش مطالعه:}
پس از مطالعه کتاب درسی، به سراغ کتاب‌های کمک درسی معتبر بروید و تست‌ها و مسائل تشریحی آن‌ها را حل کنید. نمونه سوالات امتحانات نهایی سال‌های گذشته را حتماً بررسی و حل کنید.

\parahead{مباحث پیشنهادی برای تمرکز بیشتر:}
\begin{itemize}
    \item فصل ۱: مسائل حرکت با شتاب ثابت و سقوط آزاد.
    \item فصل ۲: مسائل دینامیک با نیروهای مختلف، حرکت دایره‌ای.
    \item فصل ۳: معادله حرکت هماهنگ ساده، انرژی در نوسان، رابطه تندی، طول موج و بسامد.
    \item فصل ۴: قانون شکست، مفاهیم تداخل و پراش.
    \item فصل ۵: معادله فوتوالکتریک، گذارهای الکترونی و طیف خطی.
    \item فصل ۶: معادلات واپاشی، مسائل نیمه‌عمر.
\end{itemize}

\subsection{برای گرفتن نمره کامل (۱۸ به بالا)}

\parahead{تمرکز:}
تسلط کامل بر تمام مفاهیم و جزئیات کتاب درسی، توانایی حل مسائل چالشی و ترکیبی، قدرت تجزیه و تحلیل بالا.

\parahead{روش مطالعه:}
علاوه بر موارد بالا، حل تست‌های سطح بالا و مسائل خلاقانه از منابع مختلف. سعی کنید مفاهیم را به صورت عمیق درک کرده و ارتباط بین مباحث مختلف را پیدا کنید. به نکات ریز و استثنائات توجه کنید.

\parahead{مباحث پیشنهادی برای تمرکز بیشتر:}
\begin{itemize}
    \item تمام فصول به صورت عمیق و با جزئیات.
    \item مسائل ترکیبی سینماتیک و دینامیک.
    \item تحلیل دقیق نمودارهای پیچیده.
    \item مفاهیم پیشرفته‌تر موج و برهمکنش‌های آن.
    \item درک عمیق مدل بور و پدیده‌های کوانتومی اولیه.
    \item کاربردهای فیزیک هسته‌ای و مسائل مربوط به انرژی هسته‌ای.
\end{itemize}

\section{نکات کلیدی برای موفقیت}
\begin{itemize}
    \item \textbf{مرور منظم:} مطالب را به صورت منظم مرور کنید تا در ذهنتان تثبیت شوند.
    \item \textbf{حل مسئله فراوان:} فیزیک درسی است که با حل مسئله یاد گرفته می‌شود. تا می‌توانید مسئله حل کنید.
    \item \textbf{توجه به نمودارها:} نمودارها بخش مهمی از سوالات فیزیک هستند. توانایی خواندن، تفسیر و رسم نمودارها را در خود تقویت کنید.
    \item \textbf{مدیریت زمان:} در جلسه امتحان، زمان خود را به خوبی مدیریت کنید تا به تمام سوالات برسید.
    \item \textbf{دقت در محاسبات:} در حل مسائل، دقت کافی در محاسبات داشته باشید تا به خاطر اشتباهات محاسباتی نمره از دست ندهید.
    \item \textbf{استفاده از راهنمای تصحیح:} پس از حل نمونه سوالات، حتماً پاسخ‌های خود را با راهنمای تصحیح مقایسه کنید تا نقاط ضعف خود را پیدا کنید.
\end{itemize}

\end{document}