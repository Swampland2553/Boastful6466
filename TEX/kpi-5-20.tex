\documentclass[12pt,a4paper]{article}

\usepackage{amsmath}
\usepackage{graphicx}

\usepackage[margin=2.5cm]{geometry} % Adjust margins as needed
\usepackage{enumitem}
\usepackage{xepersian}
\settextfont{Amiri} % A widely available and suitable Persian font
% \settextfont[Scale=1.1]{XB Niloofar} % Another option, if available and preferred

 % For more control over lists, if needed

% For better looking sections (optional)
% \usepackage{titlesec}
% \titleformat{\section}{\normalfont\Large\bfseries}{\RL{\thesection.}}{1em}{}
% \titleformat{\subsection}{\normalfont\large\bfseries}{\RL{\thesubsection.}}{1em}{}

\title{KPI های مطرح درباره مشاوره تحصیلی دبیرستان \\
و بررسی تفصیلی آنها}
\author{} % Author can be added here if known
\date{}    % Date can be added or left empty for current date

\begin{document}
\maketitle
\begin{center}
    \rule{0.8\textwidth}{0.4pt} % Optional horizontal rule after title
\end{center}
\vspace{1em}



\section{افزایش حداقل ۱۵٪ در میانگین تراز آزمون‌های جامع دانش‌آموزان نسبت به سال گذشته}

\subsection*{راه ثبت و ضبط}
\begin{itemize}
    \item \textbf{نیاز به داده‌های پایه:} دسترسی به میانگین تراز آزمون‌های جامع سال گذشته دانش‌آموزان مشابه (مثلاً پایه دوازدهم همان مدرسه یا مدارس هم‌تراز).
    \item \textbf{ثبت منظم:} ایجاد یک فایل اکسل یا استفاده از نرم‌افزارهای مدیریت آموزشی برای ثبت تراز هر دانش‌آموز در هر آزمون جامع (مانند آزمون‌های گزینه‌دو یا آزمون‌های شبیه‌ساز داخلی).
    \item \textbf{محاسبه میانگین:} محاسبه میانگین تراز کل دانش‌آموزان شرکت‌کننده در هر آزمون.
    \item \textbf{نگهداری:} ذخیره‌سازی امن فایل‌ها با قابلیت پشتیبان‌گیری منظم.
\end{itemize}

\subsection*{مقایسه علمی و به‌روز}
\begin{itemize}
    \item \textbf{مقایسه با سال پایه:} مقایسه مستقیم میانگین تراز فعلی با میانگین تراز سال گذشته.
    \item \textbf{تحلیل روند:} بررسی روند تغییرات میانگین تراز در طول سال تحصیلی جاری. آیا رشد مشاهده می‌شود؟ آیا این رشد پایدار است؟
    \item \textbf{مقایسه با گروه‌های کنترل (در صورت امکان):} اگر مدرسه‌ای مشابه با برنامه‌ای متفاوت وجود دارد، مقایسه می‌تواند مفید باشد (البته این مورد پیچیده‌تر است).
    \item \textbf{تحلیل آماری:} استفاده از آزمون‌های آماری ساده (مانند \lr{t-test}) برای بررسی معنادار بودن تفاوت میانگین‌ها (در صورت داشتن داده‌های کافی).
\end{itemize}

\section{کسب حداقل ۸۰٪ رضایتمندی در نظرسنجی‌های پایان دوره از دانش‌آموزان و والدین}

\subsection*{راه ثبت و ضبط}
\begin{itemize}
    \item \textbf{طراحی پرسشنامه استاندارد:} تهیه پرسشنامه‌ای جامع و معتبر که ابعاد مختلف طرح (کیفیت کلاس‌ها، مشاوره، برنامه‌ریزی، امکانات، ارتباطات و ...) را بسنجد. سوالات باید هم کمی (مثلاً طیف لیکرت ۱ تا ۵) و هم کیفی (سوالات باز) باشند.
    \item \textbf{اجرای نظرسنجی:} به صورت آنلاین (مثلاً با \lr{Google Forms}) یا کاغذی در پایان هر دوره (مثلاً پایان دوره تابستان، پایان ترم اول، پایان سال).
    \item \textbf{تحلیل داده‌ها:} محاسبه درصد رضایتمندی کلی و تفکیکی برای هر بخش. تحلیل محتوای پاسخ‌های کیفی برای شناسایی نقاط قوت و ضعف.
    \item \textbf{نگهداری:} آرشیو کردن پرسشنامه‌ها و نتایج تحلیل‌ها به صورت دیجیتال.
\end{itemize}

\subsection*{مقایسه علمی و به‌روز}
\begin{itemize}
    \item \textbf{مقایسه با اهداف:} مقایسه درصد رضایتمندی با هدف ۸۰٪.
    \item \textbf{مقایسه دوره‌ای:} بررسی تغییرات رضایتمندی در طول زمان (مثلاً از ترم اول به ترم دوم).
    \item \textbf{تحلیل کیفی:} شناسایی دلایل نارضایتی و برنامه‌ریزی برای رفع آنها.
    \item \textbf{استفاده از بنچمارک (در صورت امکان):} مقایسه با سطح رضایتمندی در مدارس مشابه.
\end{itemize}

\section{کاهش ۲۰٪ در گزارش‌های استرس شدید تحصیلی (در صورت وجود داده‌های پیشین)}

\subsection*{راه ثبت و ضبط}
\begin{itemize}
    \item \textbf{نیاز به داده‌های پایه:} دسترسی به گزارش‌های استرس سال گذشته یا انجام یک ارزیابی اولیه استرس در ابتدای طرح.
    \item \textbf{ابزار سنجش استرس:} استفاده از پرسشنامه‌های استاندارد سنجش استرس تحصیلی (مانند پرسشنامه استرس تحصیلی یا مقیاس‌های اضطراب امتحان) یا مصاحبه‌های ساختاریافته با دانش‌آموزان و مشاوران.
    \item \textbf{سنجش دوره‌ای:} انجام سنجش در مقاطع زمانی مشخص (مثلاً هر فصل یا هر ترم).
    \item \textbf{نگهداری:} ثبت نتایج سنجش‌ها به صورت محرمانه و تحلیل \lr{aggregated} (تجمیعی).
\end{itemize}

\subsection*{مقایسه علمی و به‌روز}
\begin{itemize}
    \item \textbf{مقایسه با داده‌های پایه:} بررسی میزان کاهش استرس نسبت به ابتدای طرح یا سال گذشته.
    \item \textbf{تحلیل روند:} بررسی تغییرات سطح استرس در طول اجرای طرح.
    \item \textbf{شناسایی عوامل استرس‌زا:} از طریق مصاحبه و سوالات کیفی، عواملی که باعث استرس می‌شوند را شناسایی و برای کاهش آنها برنامه‌ریزی کنید.
\end{itemize}

\section{مشارکت حداقل ۹۰٪ دانش‌آموزان در کارگاه‌های ماهانه}

\subsection*{راه ثبت و ضبط}
\begin{itemize}
    \item \textbf{لیست حضور و غیاب:} ثبت دقیق حضور دانش‌آموزان در هر کارگاه.
    \item \textbf{محاسبه درصد مشارکت:} برای هر کارگاه و به طور کلی.
    \item \textbf{نگهداری:} آرشیو لیست‌های حضور و غیاب.
\end{itemize}

\subsection*{مقایسه علمی و به‌روز}
\begin{itemize}
    \item \textbf{مقایسه با هدف:} مقایسه درصد مشارکت با هدف ۹۰٪.
    \item \textbf{تحلیل دلایل عدم مشارکت:} در صورت پایین بودن مشارکت، دلایل آن بررسی و رفع شود (مثلاً جذاب نبودن موضوع، زمان نامناسب و ...).
    \item \textbf{بررسی روند مشارکت:} آیا مشارکت در طول زمان افزایش یا کاهش می‌یابد؟
\end{itemize}

\section{بهبود میانگین نمرات دروس نهایی به میزان حداقل 0/5 نمره}

\subsection*{راه ثبت و ضبط}
\begin{itemize}
    \item \textbf{نیاز به داده‌های پایه:} دسترسی به میانگین نمرات دروس نهایی سال گذشته دانش‌آموزان مشابه.
    \item \textbf{ثبت نمرات:} جمع‌آوری نمرات دروس نهایی دانش‌آموزان پس از اعلام نتایج.
    \item \textbf{محاسبه میانگین:} محاسبه میانگین نمرات برای هر درس و میانگین کلی.
    \item \textbf{نگهداری:} ثبت نمرات به صورت امن و محرمانه.
\end{itemize}

\subsection*{مقایسه علمی و به‌روز}
\begin{itemize}
    \item \textbf{مقایسه با سال پایه:} بررسی میزان افزایش میانگین نمرات نسبت به سال گذشته.
    \item \textbf{تحلیل تفکیکی دروس:} کدام دروس بیشترین و کمترین رشد را داشته‌اند؟ دلایل آن چیست؟
    \item \textbf{مقایسه با میانگین کشوری یا استانی (در صورت دسترسی):} برای ارزیابی عملکرد در مقیاس بزرگتر.
\end{itemize}

\section{سیستم گزارش‌دهی و تحلیل نتایج \lr{KPI} ها}
\begin{itemize}
    \item \textbf{داشبورد مدیریتی:} ایجاد یک داشبورد ساده (می‌تواند در اکسل یا نرم‌افزارهای تخصصی‌تر باشد) که وضعیت هر \lr{KPI} را به صورت بصری و به‌روز نشان دهد. استفاده از نمودارها و شاخص‌های رنگی (سبز، زرد، قرمز) برای نمایش سریع وضعیت.
    \item \textbf{گزارش‌های دوره‌ای:} تهیه گزارش‌های منظم (مثلاً ماهانه یا فصلی) برای تیم مدیریت مدرسه، دبیران و والدین (بخش‌های مرتبط با آنها) که شامل:
    \begin{itemize}
        \item وضعیت فعلی هر \lr{KPI}.
        \item مقایسه با اهداف و داده‌های پایه.
        \item تحلیل روند و شناسایی نقاط قوت و ضعف.
        \item اقدامات پیشنهادی برای بهبود.
    \end{itemize}
    \item \textbf{جلسات بررسی عملکرد:} برگزاری جلسات منظم با حضور مسئولین طرح، مشاوران و نمایندگان دبیران برای بررسی نتایج \lr{KPI}ها و تصمیم‌گیری در مورد اقدامات اصلاحی.
\end{itemize}

\section{راهکارهای علمی و به‌روز برای مدیریت \lr{KPI}ها}
\begin{itemize}
    \item \textbf{استفاده از نرم‌افزارهای مدیریت آموزشی (\lr{LMS}) یا \lr{CRM} آموزشی:} این نرم‌افزارها می‌توانند در ثبت، نگهداری و تحلیل داده‌ها بسیار کمک‌کننده باشند و گزارش‌های خودکار تولید کنند.
    \item \textbf{استانداردسازی فرآیندها:} برای جمع‌آوری داده‌ها (مثلاً پرسشنامه‌ها) از روش‌های استاندارد و یکسان استفاده شود تا قابلیت مقایسه وجود داشته باشد.
    \item \textbf{محرمانگی داده‌ها:} اطمینان از حفظ حریم خصوصی دانش‌آموزان و محرمانه بودن اطلاعات فردی آنها.
    \item \textbf{آموزش تیم:} افرادی که مسئول جمع‌آوری و تحلیل داده‌ها هستند باید آموزش لازم را دیده باشند.
    \item \textbf{فرهنگ بهبود مستمر:} نتایج \lr{KPI}ها باید به عنوان ابزاری برای یادگیری و بهبود مداوم طرح استفاده شوند، نه فقط برای قضاوت.
    \item \textbf{درگیر کردن ذی‌نفعان:} در فرآیند تعریف، پایش و تحلیل \lr{KPI}ها از نظرات دانش‌آموزان، والدین و دبیران نیز استفاده شود تا احساس مالکیت و همکاری افزایش یابد.
\end{itemize}

با پیاده‌سازی یک سیستم مدون و دقیق برای پایش این \lr{KPI}ها، می‌توان به طور مستمر اثربخشی طرح "همراهی تا اوج موفقیت" را ارزیابی کرده و در جهت بهبود آن گام برداشت. این رویکرد داده‌محور، به تصمیم‌گیری‌های آگاهانه‌تر و نتایج بهتر منجر خواهد شد.

\end{document}